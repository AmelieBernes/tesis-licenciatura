\section{Bases ortonormales de espacios de Hilbert}

En general, dado $V$ un $F-$ espacio vectorial,
definimos una base como \TODO{AAA}. La importancia 
de este concepto es obvia; una base permite representar
de forma única a un elemento $v \in V$ por medio, simplemente,
de una colección finita de escalares. Puede demostrarse
que, dado $A \subseteq V$, el que $A$ sea base de $V$
equivale a que $A$ sea un maximal l.i.; esto último significa
que todo subconjunto l.i. $A'$ de $V$ que contenga
a $A$ de hecho coincide con $A$.

El tener además en $V$ definido un producto punto
dota de estructura extra al espacio, en particular, 
nos provee de una noción de longitud y también 
de ortogonalidad entre vectores, luego, en este 
contexto podemos también hablar de subconjuntos $B$
ortonormales maximales. Puesto que la ortogonalidad implica
trivialmente la independencia lineal, es natural plantearse
la siguiente pregunta: en un $F-$espacio vectorial $V$
con producto punto
cualquiera, ¿es cierto que todo ortonormal maximal es también
maximal l.i.? Como se establece 
fácilmente en la siguiente proposición,
en el caso finito dimensional, la respuesta a esta
pregunta es positiva.

\begin{prop}
Sea $V$ un espacio finito dimensional
con producto punto. Si $A$
es un subconjunto ortonormal maximal de $V$, entonces
también es maximal l.i. (o sea, base de $V$)
\end{prop}
\noindent
\textbf{Demostración.}
Supongamos que $A \subseteq V$ es maximal 
respecto a la propiedad de ser ortonormal pero
no a la de ser l.i., es decir, que existe
un subconjunto $A'$ l.i. que contiene a $A$ propiamente.
Digamos que $A' = A \cup B$ (es decir, ordenamos a 
los elementos de $A'$ poniendo todos los de $A$ al principio).
Podemos ortonormalizar a este con el proceso de Gram-Schmidt
para obtener un subconjunto ortonormal $B$ (que genera
al mismo espacio que $A'$);
por las fórmulas del teorema \ref{Teo:Gram-Schmidt}
y el hecho de que los elementos de $A$ sean ortogonales
entre sí es claro que los primeros $|A|$ elementos
son los elementos de $A$ (``intactos''); así $B$
es un subconjunto ortonormal de $V$ que contiene propiamente
a $A$ (contradicción).
\QEDB
\vspace{0.2cm}

Como veremos en lo que sigue de esta sección,
la situación cambia radicalmente cuando no se supone
que la dimensión del espacio sea finita.

\begin{defi} \label{defi: BON y base de Hamel}
Sea $V$ un espacio vectorial con producto punto.
Si $A$ es un subconjunto de $V$ que es 
maximal con respecto a la propiedad de
\begin{itemize}
\item ser linealmente independiente, entonces lo llamaremos
\textbf{base de Hamel} de $V$
\item ser ortonormal, entonces lo llamaremos
\textbf{base ortonormal} de $V$ (y abreviamos este
nombre como \textbf{BON}).
\end{itemize}
\end{defi}

En base a la teoría que exponemos a continuación
(que proviene principalmente de \TODO{Conway}),
veremos que, para los espacios vectoriales con producto
punto, conviene mucho más trabajar con BONs que con
bases de Hamel; estamos dispuestos a sacrificar el poder
representar a todo vector de forma única como combinación lineal
finita de elementos de $A$ \TODO{AAA}. Es por eso que en la mayoría
de los libros de análisis funcional, el nombre ``base''
suele reservarse para las BONs (pudiendo llegar a confundir
a un lector primerizo, y llevarlo a pensar en
lo que nosotros hemos llamado bases de Hamel). Puesto que
nosotros no queremos dar lugar a confusiones de este estilo,
nos abstenemos de usar el nombre ``base'', y en su lugar ocupamos
los nombres introducidos en la definición \ref{defi: BON y base de Hamel}.

\[
----
\]

Incluimos el siguiente teorema (que es 
\TODO{[CON, THm 4.13]}) porque en repetidas ocasiones
usaremos las equivalencias establecidas en él. Nos basamos
en la demostración dada en el libro, pero completamos todos
los detalles que no se dan en este. El inciso $c)'$
es uno añadido por nosotros.

\TODO{AAAAA}

En el siguiente teorema (que es 
\TODO{[CON, THm 4.13]}) damos algunas equivalencias
a ser BON; en repetidas ocasiones usaremos las equivalencias
en él establecidas. Además, gracias a este podremos
dar un ejemplo de una BON que no es base de Hamel (justificando
así la distinción que hacemos entre uno y otro concepto).
Por todas las propiedades expresadas en el teorema, vemos
que, en el contexto de espacios de Hilbert, es 
mucho más conveniente usar BONs que bases de Hamel.
Es por eso que el nombre ``base'' se reserva para
estos primeros objetos; sin embargo, como
nosotros no queremos conjundirnos nunca sobre qué bases
hablamos, evitaremos el nombre ambiguo de ``base'' y en su lugar
usaremos los nombres completos de ``base de Hamel''
(entiendiendo por esta un subconjunto l.i. maximal
de un espacio vectorial) y ``base ortonormal''
(siendo esta un subconjunto ortonormal maximal
de un espacio vectorial con producto punto)

\begin{teo} \label{thm: Coway, 4.13}
Sea $\cali{E}$ un subconjunto ortonormal del espacio
de hilbert $\cali{H}$.
Las siguiente proposiciones son equivalentes entre sí:
\begin{itemize}
\item[a)] $\cali{E}$ es una BON de $\cali{H}$.
\item[b)] Si $h \in \cali{H}$ es ortogonal a todo elemento de 
$\cali{E}$ entonces es cero.
\item[c)] $\overline{span}(E)= \cali{H} $.
\item[c)'] $\overline{E}=\cali{H}$. \TODO{???}
\item[d)] Para todo $h \in \cali{H}$ se cumple que
\[
h = \suma{}{}{\{ <h,e> e | e \in \cali{E} \}}
\]
\item[e)] Para cualesquiera $g, h \in \cali{H}$,
\[
<g,h>= \suma{}{}{\{ <g,e> <e,h> | e \in \cali{E} \}}
\]
\item[f)](\textbf{Identidad de Parseval})
Para todo $h \in \cali{H}$ se cumple que
\[
||h||^{2}= \suma{}{}{\{ <h,e>^{2} | e \in \cali{E} \}}.
\]
\end{itemize}
\end{teo}
\begin{nota}
El inciso $c)$ del listado corresponde a la definición
de BON dada en \TODO{Kolmogorov}; así, una BON
es también un subconjunto ortonormal tal que el menor
subespacio cerrado de $\cali{H}$ que contiene a $E$ es 
todo $\cali{H}$.
\end{nota}
\noindent
\textbf{Demostración.}
\begin{itemize}
\item[$a) \Rightarrow b)$] Suponer la existencia de un 
$h$ no cero ortogonal a todo elemento de $\cali{E}$
nos permite considerar al subconjunto ortonormal
\[
\cali{E} \cup \{ \frac{h}{||h||} \}
\]
de $\cali{H}$
que contiene propiamente a $\cali{E}$,
contradiciendo así la maximalidad de $\cali{E}$.

\item[$b) \Rightarrow c)$] Sea el subespacio
$W := \overline{span}(E)$ de $\cali{H}$. Por ser
$W$ un subespacio cerrado de $\cali{H}$, podemos proyectar
sobre él. Para mostrar que tenemos la igualdad entre
estos dos espacios, tomemos un $h \in \cali{H}$ cualquiera.
Según el teorema de la proyección ortogonal 
\ref{Teo:proyOrt}, $(h-\Pi_{W}(h))$
es ortogonal a todo elemento de $W$, 
a todo elemento de $E$;
por hipótesis, esto implica que $h-\Pi_{W}(h)$ sea el vector cero.
Así,
\[
h = \Pi_{W}(h) \in W.
\]
\item[$c) \iff c)'$]  \TODO{???}

\item[$c) \Rightarrow b)$] Sea $h \in \cali{H}$
ortogonal a todo elemento de $\cali{E}$; mostremos que
$h$ es cero. Como $h \in \overline{span}(E)$ (hipótesis),
existe $(b_{n})_{n \in \IN}$ una sucesión de
$span(E)$ tal que
$\lim_{n \rightarrow \infty}b_{n}=h$.
Usando el que para toda $n$ el producto punto $<b_{n}, h>$
sea cero y la continuidad
establecida en la proposición \ref{prop: continuidad del producto punto},
llegamos a que
\[
<h,h>= <h, \lim_{n \rightarrow \infty}b_{n}>= 
\lim_{n \rightarrow \infty}b_{n}<h, b_{n}>= 
\lim_{n \rightarrow \infty} 0 = 0,
\]
de donde concluimos que $h=0$.


\item[$b) \Rightarrow d)$] Según el \TODO{Lema 4.12},
la red $\suma{}{}{\{ <h,e>e | e \in \cali{E} \}}$
en efecto converge a un vector; resta ver que tal 
vector es $h$
o, equivalentemente, que
\begin{equation} \label{eq 0: 8Agosto}
h- \suma{}{}{\{ <h,e>e | e \in \cali{E} \}}
\end{equation}
es el vector cero; según nuestra hipótesis, podremos
concluir esto si demostramos que para todo $e \in \cali{E}$
ocurre
\[
<e, \suma{}{}{\{ <h,e>e | e \in \cali{E} -h \}} > =0.
\]
Haremos esto mostrando que la norma de 
\eqref{eq 0: 8Agosto} es cero \\
Sea $\epsilon >0$.
Por definición de convergencia de redes, sabemos que
existe $F \subseteq \cali{E}$ finito con $e \in F$
tal que
\begin{equation} \label{eq 1: 8Agosto}
|L_{F}| < \epsilon,
\end{equation}
donde $L_{F}:= \suma{f \in F}{}{<h,f>f} - 
\suma{}{}{\{ <h,e>e| e \in \cali{E} \}}$. \\

Por la bilinealidad del producto punto y por
ser $e$ ortogonal a todo elemento de $\cali{E}-\{ e \}$
tenemos que
\begin{equation} \label{eq 2: 8Agosto}
<\suma{f \in F}{}{<h,f>f}, e> = \suma{f \in F}{}{<h,f><f,e>}
= <h,e> <e,e>=<h,e>.
\end{equation}

Así,
\begin{align*}
0 \leq |<e, \suma{}{}{\{ <h,e>e | e \in \cali{E} \}} -h > | = &
| <e, \suma{}{}{\{ <h,e>e | e \in \cali{E} \}} > - <e,h>| = \\
=  | <e, \suma{}{}{\{ <h,e>e | e \in \cali{E} \}} - \suma{f \in F}{}{<h,f>f}
+ & \suma{f \in F}{}{<h,f>f} > - <h,e> | \\
= & | <e,  \suma{f \in F}{}{<h,f>f}> + <e, L_{F}> - <h,e>  | \\
(\text{por \eqref{eq 2: 8Agosto} }) = & |<e, L_{F}>| \\
(\text{Cauchy-Schwarz})&   \leq |e| |L_{F}| = |L_{F} | \\
(\text{por \eqref{eq 1: 8Agosto}})& < \epsilon.  
\end{align*}


Demostramos entonces que
\[
(\forall \epsilon > 0): \hspace{0.5cm} 
0 \leq |<e, \suma{}{}{\{ <h,e>e | e \in \cali{E} \}} -h > |< \epsilon;
\]
de esto conlcuimos que $|<e, \suma{}{}{\{ <h,e>e | e \in \cali{E} \}} -h > |$
es cero.

\item[$d) \Rightarrow e)$] Según nuestra hipótesis,
\[
h= \suma{}{}{\{ <h,e>e | e \in \cali{E} \} }, \hspace{0.5cm}
g= \suma{}{}{\{ <g,e>e | e \in \cali{E} \} }.
\]

Para $F \subseteq \cali{E}$ finito, sean
\[
H_{F}:= \suma{e \in F}{}{<h, e>e-h},
\hspace{0.5cm}
G_{F}:= \suma{e \in F}{}{<g, e>e-g}.
\]

Por la ortonormalidad supuesta en el conjunto $\cali{E}$,
\begin{equation}
\label{eq 3: 8Agosto}
<\suma{e \in F}{}{<h, e>e}, \suma{e \in F}{}{<g, e>e} >
= \suma{e \in F}{}{<h,e><g,e>};
\end{equation}
así,
\begin{align*}
|\suma{e \in F}{}{<h,e><g,e>} - <h,g>| \underbrace{=}_{\text{por 
\eqref{eq 3: 8Agosto}}} & |<\suma{e \in F}{}{<h, e>e}, \suma{e \in F}{}{<g, e>e} >
- <h,g>| \\
= & |<H_{F}+h, G_{F}+g > - <h,g>| \\
= & |<H_{F}, G_{F}> -<H_{F}, g> + <h,G_{h}>| \\
(\text{desigualdad triangular}) \leq & |<H_{F}, G_{F}> | + |<H_{F}, g>|
+ |<h,G_{H}>| \\
(\text{Cauchy-Schwarz}) \leq & ||H_{F}|| \cdot ||G_{F}|| +
||H_{F}|| \cdot ||g|| + ||h|| \cdot ||G_{H}||;
\end{align*}
puesto que $||g||$ y $||h||$ son constantes
y $||H_{F}||$ y $||G_{F}||$ pueden hacerse tan pequeñas como
se quiera (escogiendo apropiadamente al conjunto finito $F$),
concluimos que la red 
$\suma{}{}{\ <h,e> <g,e> | e \in \cali{E}\}}$
converge a $<h,g>$.

\item[$e) \Rightarrow f)$] Para $h \in \cali{H}$,
\[
||h||^{2}=<h,h> = \suma{}{}{\{ <h,e>^{2}| e \in \cali{E} \}}.
\]

\item[$e) \Rightarrow f)$]
Supongamos que $\cali{E}$ puede extenderse aún más
para ser ortonormal, es decir, que existe 
$e_{0} \in \cali{H}$ unitario tal que
para todo $e \in \cali{E}$, $<e_{0}, e>=0$. 
Llegamos entonces a la siguiente contradicción:
\[
1=||e_{0}||^{2}=\suma{}{}{\{ <e_{0}, e>^{2}| e \in \cali{E} \}}=
\suma{}{}{\{ 0\}}=0.
\]
\QEDB
\end{itemize}
\vspace{0.2cm}

Ya estamos listos par dar un ejemplo de una BON que
no es una base de Hamel.

\begin{ejemplo}
(de un subconjunto de un espacio
con producto punto que sea maximal ortonormal pero no maximal l.i.)
\TODO{Dónde introduzco al espacio de sucesiones $\ell^{2}$??
Yo creo que justo después de G-S, o sea, justo antes de esta.}

Considere al espacio de Hilbert
\[
\ell^{2}= \{ x: \IN \longrightarrow \IR | \hspace{0.2cm} 
\suma{k=1}{\infty}{|x_{k}|^{2}}< \infty \}
\]

con el producto punto
\[
<x,y>= \suma{k=1}{\infty}{x_{k}y_{k}}.
\]

Sea el subconjunto de este
\[
\cali{B}:= \{e_{i}: \IN \longrightarrow \IR| \hspace{0.2cm} i \in \IN \},
\]

donde $e_{i}$ es la sucesión dada por la regla
\[
e_{i}(j)=\delta_{i,j}, \hspace{0.4cm} j \in \IN.
\]


\begin{itemize}
\item[i)]($\cali{B}$ es un subconjunto maximal ortonormal) 
Claro que todos los elementos de $\cali{B}$ tienen norma uno;
además, si $x \in \ell^{2}$ es tal que para todo índice $i$
se tiene la igualdad
\[
x_{i}= \suma{k=1}{\infty}{x(k)e_{i}(k)}=<x,e_{i}>=0,
\]
entonces $x$ es la sucesión cero, o sea,
el elemento cero del espacio
$\ell^{2}$. Según la equivalencia
$a) \iff b)$ del teorema \ref{thm: Coway, 4.13}, esto basta
para demostrar que $\cali{B}$ es BON de $\ell^{2}$.

\item

\end{itemize}


\TODO{AAAAA}

\end{ejemplo}

\begin{teo} \label{teo: Kol 6, p.149}
[Kol, 6 p. 149] Si $\{ v_{k} |k \in \IN \}$ es un sistema
ortonormal \footnote{aquí consideramos la situación de un subespacio
con BON numerable, pero en el argumento la demostración del resultado ya
se incluye el caso de una BON finita.} de un espacio con producto
punto $V$ y $v \in V$, entonces la expresión
\[
|| v - \suma{k=1}{\infty}{a_{k}v_{k}}  ||
\]
alcanza su mínimo para la elección de coeficientes
$a_{k}= <v , v_{k} > $.
\end{teo}

\noindent
\textbf{Demostración.}
Si $(a_{k})_{k \in \IN}$ es una sucesión de coeficientes,
como es costumbre,
en caso de existir el límite de sumas parciales
$S_{n}:= \suma{k=1}{n}{a_{k}v_{k}}$, este es un vector que denotamos
por $\suma{k=1}{\infty}{a_{k}v_{k}}$. Además,
por la continuidad de la norma,
\[
||v - \suma{k=1}{\infty}{a_{k}v_{k}} || = \limite{n \rightarrow \infty }{||v-S_{n}||}
\]
por lo que, si encontramos una elección de coeficientes $(a_{k})_{k \in \IN}$
que haga que
la sucesión $(S_{n})_{n \in \IN}$ converja y tal que las normas
$||v-S_{n}||$ (o, equivalentemente, los números $||v-S_{n}||^{2}$)
sean míminas, esa será la elección buscada. \\
Observe que, para toda $n$,
\begin{align*}
||v-S_{n}||^{2} & = <v-S_{n} , v-S_{n} > \\
\text{(bilinealidad)} & =  <v, v > - 2 < v , v_{n} > + <S_{n} , S_{n}  > \\
\text{(ortonormalidad)} & =  ||v||^{2} - 2 < v ,\suma{k=1}{n}{a_{k}v_{k}} > + 
\suma{k=1}{n}{a_{k}^{2}} \\
= & ||v||^{2} - 2\suma{k=1}{n}{a_{k}c_{k}}  + 
\suma{k=1}{n}{a_{k}^{2}},
\end{align*}
donde $c_{k}:= < v , v_{k} >$. Así,
\[
||v-S_{n}||^{2} = ||v||^{2} - \suma{k=1}{n}{c_{k}^{2}}  + 
\suma{k=1}{n}{(a_{k}-c_{k})^{2}};
\]
claro que la sucesión de coeficientes que minimiza
esta expresión es $(c_{k})_{k \in \IN}$; además, no es difícil ver
que esta candidata hace que la sucesión $(S_{n})_{n \in \IN}$
de sumas parciales converja, pues tenemos la siguiente acotación
válida para toda $n$:

\begin{equation} \label{ecuacion teor kolmogorov}
\suma{k=1}{n}{c_{k}^{2}} \leq ||v||^{2}-||v-S_{n}||^{2} \leq 
\underbrace{||v||^{2}}_{cte}.
\end{equation}

\QEDB
\vspace{0.2cm}


\begin{cor} \label{cor: proyeccion en terminos de una BON}
\TODO{Tal vez sería bueno poner una versión de este
para BONs finitas.}
(dando explícitamente a la proyección de un vector
a un subespacio cerrado respecto a una BON de este último)
Con la notación del teorema \ref{teo: Kol 6, p.149},
para todo $v \in V$
tenemos que
\[
\Pi_{W}(v) = \suma{k=1}{\infty}{<v, v_{k}>v_{k}},
\]
donde $W := \overline{\{v_{k}| k \in \IN \}}$. \TODO{de hecho,
Conway demuestra esto en p. 15.}
\end{cor}


\begin{cor} \label{cor: representacion de un vector respecto a una BON}
Si $V$ es un espacio con producto punto 
y $B=(v_{k})_{k \in \Delta}$ es una BON de este
a lo más numerable, entonces, para todo
$v \in V$, $v = \suma{k \in \Delta}{}{<v, v_{k}>v_{k}}$
\end{cor}
\noindent
\textbf{Demostración.}
La proyección de un $v \in V$ sobre $V$ es trivialmente $v$
($v$ es el elemento de $V$ más cercano a sí mismo); según el corolario
\ref{cor: proyeccion en terminos de una BON}, esta
proyección es la serie propuesta. \QEDB
\vspace{0.2cm}

Note que, en el caso en el que $\Delta$ sea infinito numerable,
¡el orden en que se tome la serie de arriba no importa! esto porque,
independientemente del orden que se elija, siempre se tratará con
una BON del espacio. \\

\begin{cor}(\textbf{desigualdad de Bessel})
Si $\{ v_{k} | k \in \IN \}$ es un sistema ortonormal en 
el espacio con producto punto $V$, entonces, para todo $v\in V$,
\[
\suma{k=1}{\infty}{<v , v_{k} >^{2}} \leq ||v||^{2}.
\]
\end{cor}
\noindent
\textbf{Demostración.}
Esto es una consecuencia inmediata de la acotación
\eqref{ecuacion teor kolmogorov} establecida en la demostración
del teorema \ref{teo: Kol 6, p.149}.
\QEDB
\vspace{0.2cm}