\section{Revisión de literatura y artículos relacionados}

Hicimos una búsqueda en la literatura
para saber si bases parecidas a la
de Legendre discreta
(i.e. a la definida en 
\ref{def: base de Legendre discreta})
ya habían sido 
consideradas antes y, en caso afirmativo, las propiedades
conocidas sobre ellas. 


\begin{itemize}
\item Encontramos el artículo \textit{``Discrete (Legendre) 
orthogonal polynomials -a survey''} ~\cite{Neuman}, en el que se habla de
ciertos polinomios ortogonales de variable independiente
discreta; fijado un entero $n$,
se afirma (pero no demuestra) la existencia de únicas funciones
\[
P_{k}(\cdot ,n)=P_{k}(m,n), \hspace{1cm} 0 \leq k \leq n
\]
de variable discreta $0 \leq m \leq n $, tales que


\begin{equation}
\suma{m=0}{n}{P_{k}(m,n) P_{l}(m,n)} =0
\hspace{0.3cm} \text{si} \hspace{0.2cm} m \neq l \tag{DLOP-1[n]} \label{eq:dlop1}
\end{equation}
y 
\begin{equation}
\text{para toda} \hspace{0.3cm} 0 \leq k \leq n, \hspace{0.3cm} P_{k}(0, n)=1.   \tag{DLOP-2[n]} \label{eq:dlop2}
\end{equation}


Pensando en los 
$n+1$ vectores de $\IR^{n+1}$ que resultan
de evaluar estas funciones discretas en su rango,
es decir, en los vectores
\begin{equation}
y_{k}:= (P_{k}(m, n))_{m=0}^{m=n}, \hspace{0.2cm}
0 \leq k \leq m,
\end{equation}
las condiciones \eqref{eq:dlop1} y \eqref{eq:dlop2}
se reinterpretan como
\begin{equation}
\{ y_{k}: \hspace{0.2cm} 0 \leq k \leq n\}
\hspace{0.3cm} \text{es una
base ortogonal de} \hspace{0.2cm} \IR^{n}
\tag{DLOP'-1[n]} \label{eq:dlop'1}
\end{equation}
y 
\begin{equation}
\text{la primera
entrada de todo} \hspace{0.2cm} y_{k} \hspace{0.2cm}
\text{es uno.} 
\tag{DLOP'-2[n]} \label{eq:dlop'2}
\end{equation}


Los autores de este estudio recopilan y derivan expresiones
analíticas (que dependen de $n$ y $k$) para estas funciones
discretas $P_{k}(\cdot ,n)$ y, del notar que los coeficientes
numéricos de estos son, salvo por posibles cambios de signo,
los de los polinomios de Legendre trasladados 
(c.f. \cite{leg})
se refieren a 
estos como \textbf{polinomios ortogonales
discretos de Legendre} (
``discrete Legendre orthogonal
polynomials'' en inglés), usando para designarlos la abreviatura
``DLOP's''. \\

Aparte de mencionar en la introducción que
`` variantes de estos polinomios fueron consideradas por primera
vez por Chebyshev en 1858 y más tarde por Gram en 1915 '', no se
da un marco teórico como el desarrollado por nosotros, sólo 
se dan expresiones analíticas para la aplicación de estos
(expresiones que nos interesa, después de efectuar los
cambios necesarios, utilizar; volveremos a esta cuestión más
adelante). Los trabajos originales de Chebyshev y Gram
no están citados en las referencias del artículo, y 
nosotros no fuimos capaces de 
encontrarlos. 


\item  Una segunda mención importante a los DLOP's se encuentra en 
la tesis ``Cross directional control of basis weight on paper machines
using Gram polynomials'', tesis en la que la motivación para la consideración de
polinomios ortogonales en un conjunto finito de puntos se
introduce como sigue: \\


Si $\hat{x}$ es un vector de $n$ entradas, digamos
\[
\hat{x}=(x(1), \ldots , x(n)),
\]
donde las entradas representan tiempos de medición,
si $\hat{y}=\hat{y}(\hat{x})$ es el vector de las medidas 
observadas en los tiempos dados por las entradas del vector $\hat{x}$,
fijado un entero $m$, si para $j=0, \ldots , n-1$
$p_{j}$ es un polinomio de grado $j$, se busca determinar escalares
$c_{j}$ tales que, asociando pesos $w(x(i))$ a los tiempos $x(i)$,
la función polinomial
\[
\suma{j=0}{m}{c_{j}P_{j}}
\]
aproxime ``lo mejor posible'' a $\hat{y}$ en los tiempos $x(i)$, 
entiendo por esto el que la expresión

\[
\suma{i=0}{n-1}{w(x(i))}\left[ y(x(i))- 
\suma{j=0}{m}{c_{j}P_{j}}(x(i)) \right]
\]

alcance un mínimo. \\
Se observa que la determinación de estos escalares se
simplifica si los polinomios $P_{j}$ son ortogonales
entre sí en los tiempos $x(1), \ldots , x(n)$ (entendiendo
por esto que los vectores de $\IR^{n}$ que resultan de evaluar
a las funciones polinomiales en ellos son ortogonales entre sí). \\
La obtención de polinomios ortogonales en un
conjunto de puntos $x(1), \ldots , x(n)$ se hace en el 
libro `` Rabinowitz, a first course in numerical analysis ''
p. 255: lo que hacen los autores es
considerar $m$ polinomios
\[
q_{j}=q_{j}(x),\hspace{1cm} j=0, \ldots , m-1,
\]

siendo $q_{j}$ un polinomio de grado $j$, 
tales que los vectores de $\IR^{n}$ que resultan de evaluar
a estos en los puntos $x(1), \ldots , x(n)$ son linealmente
independientes. \TODO{luego hacen lo mismo que nosotros, verdad?
Después derivan unas fórmulas por recurrencia que no entendí.}
\end{itemize}