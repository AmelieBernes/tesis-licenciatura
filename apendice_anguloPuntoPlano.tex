\subsection{Definición de ángulo entre un punto y un plano en un espacio euclideo}

Como vimos en \TODO{(ref: justo después de C-S di 
que es gracias a esta desigualdad que puedes definir el
ángulo entre vectores)}, una de las grandes ventajas de tener
definido en un espacio vectorial $V$ un producto punto es
que este conlleva una definición natural de ángulo entre
vectores.

Si además $W$ es un subespacio cerrado de $V$
(lo que, recuerde, siempre ocurre en caso de que
$V$ sea finito dimensional \TODO{ref.}), entonces también
es posible definir el ángulo entre un punto $x \in V$
del espacio y $W$.

\begin{defi} \label{def: angulo punto subespacio}
Sea $(V, \langle \cdot , \cdot \rangle)$ un espacio con 
producto punto. Sean $W \leq V$ un subespacio cerrado de $V$
y $x \in V$. Definimos el \textbf{ángulo entre $x$ y $W$}
como el ángulo que forma 
$x$ con su proyección a $W$, es decir,
\[
\measuredangle (x, W):= \measuredangle(x, \Pi_{W}(x)).
\]
\end{defi}

Una caracterización del ángulo entre un punto y un subespacio
se da a continuación.

\begin{prop}
Si $V$, $W$ y $x$ son como en la definición 
\ref{def: angulo punto subespacio}, entonces
\[
\measuredangle (x, W) = min \{ \measuredangle(x,w) \hspace{0.1cm} :
 \hspace{0.1cm} w \in W \}.
\]
\end{prop}
\noindent
\textbf{Demostración.}
Sea $w$ un elemento cualquiera de $W$.
Puesto que el ángulo entre dos vectores es
preservado bajo multiplicación por escalares
(c.f. \TODO{debes poner en el apéndice este resultado, 
luego lo citas en el ejemplo}), sin pérdida de generalidad
podemos suponer que 
\begin{equation}
\label{eq1: 9Feb}
||w||= || \Pi_{W}(x)||.
\end{equation}

De la definición del vector $\Pi_{W}(x)$ se sigue que
\[
|| x -\Pi_{W}(x) ||^{2} \leq  || x - w||^{2};
\]
expresando ambos lados de la desigualdad como un producto
punto (c.f. \TODO{referencia}) y aplicando la
bilinealidad del producto punto, llegamos a que
\[
\langle x , x \rangle -2 \langle x, \Pi_{W}(x) \rangle +
\langle \Pi_{W}(x) , \Pi_{W}(x) \rangle \leq 
\langle x , x \rangle  -2 \langle x , w \rangle 
+ \langle w , w \rangle ;
\]
usando \eqref{eq1: 9Feb}, podemos simplificar esta
última desigualdad para llegar a 
\[
\langle x, w \rangle \leq \langle x, \Pi_{W}(x) \rangle,
\]
de donde se sigue, usando nuevamente \eqref{eq1: 9Feb},
que 
\[
cos \left( \measuredangle (x,w) \right) =
\frac{\langle x , w \rangle}{||x||\cdot ||w||} \leq
\frac{\langle x ,  \Pi_{W}(x)  \rangle}{||x||\cdot ||w||} =
cos \left( \measuredangle \left(x, \Pi_{W}(x) \right) \right);
\]
del que la función coseno sea decreciente en el intervalo
$[0, \pi]$ se concluye de esta última desigualdad que
$ \measuredangle \left(x, \Pi_{W}(x) \right) \leq 
\measuredangle (x,w)$.
\QEDB
\vspace{0.2cm}



\TODO{Pon ejemplos gráficos en dimensión 2 y 3.
Creo que en dimensión 2 el ángulo entre $x$ y un vector
de $W$ es siempre el mismo.}