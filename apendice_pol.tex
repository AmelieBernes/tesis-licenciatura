\section{Polinomios y teorema fundamental del álgebra}

En general, si $R$ es un anillo, se define a partir de 
él un nuevo anillo denominado en \textbf{anillo de polinomios
con coeficientes en $R$} como sigue:


En el contexto del álgebra, la definición inicial se hace

el objetivo de hacerlo de esta forma es, a partir de los
elementos de $K$ construir elementos $f$ tales que
(pues la idea es de tal forma que los coeficientes sean 
casi todos cero (o sea, que el conjunto de coeficientes no cero 
sea finito) y que estos determinen unívocamente al polinomio).
(comentarios de Jacobson!) \TODO{aquí define
evaluación de un polinomio en un elemento del anillo.}

\begin{defi}
Sean $K$ un anillo, $f(x)= \suma{k=0}{n}{a_{k}x^{k}} \in K[x]$
con $a_{n}$ un elemento no cero del anillo $K$.
\begin{itemize}
\item Al elemento $a_{n} \in K$ se le llama el \textbf{coeficiente
principal} del polinomio $f$, y
\item a $n \in \overline{\IN}$ se le llama el \textbf{grado de $f$}.
\end{itemize}
Todo elemento $r \in K$ tal que $f(r)=0$ se llamará 
una \textbf{raíz de $f$}. \TODO{Aquí di que, por lo general,
se dice que el grado del polinomio cero no está definido o que es $- \infty$,
pero que nosotros, al igual que muchos autores, vamos a convenir en
pensar que el polinomio cero, al igual que todo polinomio
constante, tiene grado cero. Esto lo hacemos simplemente para alsfj
y da referencias al trabajo principal en el que esta convención sea útil.}
\end{defi}

\TODO{Sólo usa campos ahora!}

A partir de ahora, nos interesa sólo tratar con 
campos $K$. De hecho, para este trabajo
sólo usaremos $K$ igual a $\IR$.

\TODO{Regla de suma de los grados de los factores ! La uso
muchas veces.}

Es una consecuencia inmediata del algoritmo de la división
(c.f. Rotman p. 131) el que, si $K$ es un campo
y $f(x) \in K[x]$ es un polinomio de grado $n>0$, 
entonces $f$ tiene a lo más $n$ raíces (el argumento es
relacionar raíces de un polinomio con divisores lineales
de este, y argumentar que, puesto que cuando el anillo de
coeficientes $K$ es un dominio entero, el grado de un 
producto de polinomios es igual a la suma de los grados de los
factores (c.f Rotman \TODO{ref}), si el grado de $f$
es $n$ entonces este no se puede factorizar como el producto
de más de $n$ factores lineales, luego, no puede tener más de 
$n$ raíces).

Se tiene pues una cota superior para la cantidad
de raíces de un polinomio basada en 
su grado, sin embargo, esta cota bien puede no
alcanzarse. De hecho, como muestra el siguiente ejemplo,
puede ser que un polinomio no constante con coeficientes en un campo
no tenga ninguna raíz.

\begin{ejemplo}
Considere al anillo $\IZ_{5}$ de enteros módulo $5$. Como
$5$ es un número primo, $\IZ_{5}$ es un campo (c.f. \TODO{ref.}).
Sea $f(x)=x^{2}-2 \in \IZ_{5}[x]$; evaluar a este polinomio
en cada uno de los cinco elementos de $\IZ_{5}$
nunca da lugar al cero del campo, luego, $f$ es un polinomio
de grado dos con coeficientes en $\IZ_{5}$ sin raíces.
\final
\end{ejemplo}

\begin{ejemplo}
El ejemplo canónico de polinomio con coeficientes reales
sin raíces (reales) es $p(x)=x^{2}+1$. No puede tener raíces por
ser el cuadrado de todo número real no negativo. 
\final 
\end{ejemplo}

%\marginpar{amelei}

\begin{teo}
\label{teo: fundamental del algebra}
\textbf{(Teorema fundamental del álgebra)}:
\TODO{Mejor usa a $t$ y no a $x$ como
argumento del polinomio} Todo polinomio
$f(x)$ de coeficientes complejos
y grado al menos uno
tiene al menos una raíz compleja.
(\TODO{Kurosh, p. 143})
\end{teo}

Como se hace notar en la referencia, es difícil exagerar
la importancia que tiene este teorema no sólo en el álgebra, 
sino en la matemática en general; se resalta también el hecho
de que, hasta el momento, todas las pruebas de este teorema
hacen uso no sólo de la estructura algebráica del dominio entero
$\IC[x]$, sino de propiedades topológicas de este (obtenidas
al definir en él una norma) o de otra índole, agregando estructura
extra a $\IC[x]$. Hay, sin embargo, construcciones puramente algebraicas
del campo $\IC$ basadas en la idea de ``completar'' al campo
de los números reales \TODO{cita a Godement.}


De un simple argumento inductivo, usando el teorema 
\ref{teo: fundamental del algebra} se demuestra que 
el que 
todo polinomio no constante con coeficientes en $\IC$
tenga al menos una raíz compleja \textbf{equivale} a que
tenga tantas raíces como su grado.

\begin{teo}
\label{teo: alt del fundamental del algebra}
Todo polinomio $f(x) \in \IC[x]$ 
de grado $n>0$
tiene exactamente $n$ raíces complejas
(contando multiplicidades).
\end{teo}
\noindent
\textbf{Demostración.}
Procedemos por inducción sobre $n$, el grado del polinomio.
Sea $f$ un polinomio con coeficientes
complejos y de grado $n=1$. Según el teorema 
\ref{teo: fundamental del algebra}, $f$ tiene al menos
una raíz $r_{1}$, luego,
el polinomio lineal $x-r_{1}$
divide a $f$. Ahora bien, si $f$ tuviese una segunda
raíz $r_{2}$, el polinomio 
de grado dos $(x-r_{1})(x-r_{2})$
dividiría a $f$, pero esto contradice el que 
todos los divisores de un polinomio $f$ con coeficientes
en un campo tengan grado menor o igual al de $f$. 
Con esto comprobamos la validez del teorema para $n=1$.

Supongamos ahora el teorema cierto para $n \geq 1$.
Sea $f \in \IC[x]$ un polinomio de grado uno; según 
\ref{teo: fundamental del algebra}, $f$ tiene al menos
una raíz $r_{1}$, luego,
\begin{equation}
\label{eq0: 11Dic}
f(x)= (x-r_{1}) g(x), 
\end{equation}
con $g(x)$ algún polinomio de coeficientes complejos y grado
$n$. Por hipótesis de inducción, $g$ tiene exactamente
$n$ raíces complejas (contando multiplicidades); puesto que,
según la ecuación \eqref{eq0: 11Dic}, las raíces de $f$
son $r_{1}$ y las raíces de $g$ (pues, en un campo, el producto
de dos elementos del campo es cero sí y sólo si alguno de 
estos es cero), concluimos, como queríamos, que, salvo
multiplicidades, $f$ tiene $n+1$ raíces complejas.
\null\nobreak\hfill\ensuremath{\square} %final dem
\vspace{0.2cm}

La siguiente consecuencia inmediata del teorema fundamental del
álgebra es (como se puede apreciar en el mapa \TODO{referencia}),
una de las piezas angulares del trabajo desarrollado en esta tesis.

\begin{prop}
\label{prop: cita TFA}
Sea $f(x) \in \IR[x]$ un polinomio con coeficientes reales
de grado a lo más $n$. Si $f$ tiene más de $n$ raíces reales, entonces
$f$ es el polinomio cero.
\end{prop}
\noindent
\textbf{Demostración.}
En efecto, si $f$ tuviese grado mayor a cero, entonces, 
según el teorema \ref{teo: alt del fundamental del algebra},
$f$ tendría exactamente $n$ raíces reales,
pero, por hipótesis, $f$ tiene más de $n$; así, $f$
debe tener grado cero, o sea, debe  ser un polinomio constante.
Puesto que el único polinomio constante con al menos una raíz
es el polinomio cero, concluimos que $f=0$.
\null\nobreak\hfill\ensuremath{\square} %final dem
\vspace{0.2cm}