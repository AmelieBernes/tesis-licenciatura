\section{Hiperplanos de espacios con producto punto de dimensión finita}
\label{section: hiperplanos}
En general, si $V$ es cualquier $F-$espacio vectorial (con $F$
un campo cualquiera) y $W$ es un subespacio de $V$
cuya dimensión difere de la del espacio ambiente $V$
por uno, $W$ se denomina un ''hiperplano`` de $V$.

Como a nosotros sólo nos interesa el escenario
en el que $V$ es finito dimensional y 
en él se ha definido un producto punto, damos la
definición para este caso
particular y algunas de sus consecuencias a continuación.

\begin{defi}
Sea $V$ un $\IR-$espacio vectorial,
con $dim(V)=n < \infty$ y con un producto punto $\langle \cdot, \cdot \rangle$
definido en él \footnote{\TODO{Por lo tanto un espacio
de Hilbert, i.e. completo, verdad?}}. A todo subespacio
$W$ de $V$ con $dim(W)=n-1$ le llamaremos un
\textbf{hiperplano de $V$.}
\end{defi}



Si $W$ es un hiperplano de $V$, puesto que
V=$W + W^{\perp}$
(c.f. \cite{friedberg}, p.355, ejercicio 13),
$W^{\perp}$ tiene dimensión uno; fijando pues un
vector $u$ de $W^{\perp}$, tenemos que
$W^{\perp}=span\{ u\}$. En base a este 
podemos definir (gracias a la linealidad
del producto punto) al siguiente funcional:

\begin{center}
\aplica{h_{u}}
{V}
{\IR }
{x}{\langle x, u \rangle.}
\end{center}
Observe que
\begin{itemize}
\item $h_{u}(x)>0$ si y sólo si $x= a u$ para algún $a>0$,
\item $h_{u}(x)=0$ si y sólo si $x \in W$,
\item $h_{u}(x)<0$ si y sólo si $x= a u$ para algún $a<0$.
\end{itemize}
Así, en base al hiperplano $W$, via
el funcional $h_{u}$ podemos dividir
al espacio ambiente $V$ en tres subconjuntos ajenos
dos a dos, a saber, 
\begin{equation}
\label{eq1: 29Nov}
R_{I}= \{ x \in V: \hspace{0.1cm} h_{u}(x)> 0 \},
\hspace{0.3cm}
R_{II}=W, 
\hspace{0.3cm}
R_{III}= \{ x \in V: \hspace{0.1cm} h_{u}(x)<0 \},
\end{equation}
siendo el funcional $h_{u}$ el criterio usado para
determinar la pertenencia de un $x \in V$ a alguna de estas regiones.

Es fácil comprobar que,
si se escoge otro vector $\tilde{u} \in W^{\perp}$,
las regiones análogas a las
\eqref{eq1: 29Nov} obtenidas ahora con el funcional
$h_{\tilde{u}}$ coinciden con las dadas en \eqref{eq1: 29Nov}
(aunque,
obviamente, pueden diferir en el orden en que estas se escriban),
pues todos los elementos de $W^{\perp}$ son múltiplos
escalares uno del otro.
En este sentido decimos que \textbf{todo hiperplano
divide en tres regiones ajenas al espacio ambiente.}



