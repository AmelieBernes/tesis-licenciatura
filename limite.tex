\section{PROVISIONAL; límites}

En \TODO{rojo} se resaltan las fórmulas que YA se han
verificado calculándolas dos veces. En 
\textcolor{blue}{azul} cuando la aproximación ha sido
simulada exitosamente.


\TODO{
\[
\frac{sen(2 \pi \omega) cos(2 \pi \omega \frac{n-1}{n})}{sen
(2 \pi \frac{\omega}{n})}
\sim
\frac{
2 \pi \omega - \frac{4 \pi^{3}}{3n^{2}}
(4n^2-6n+3) \omega^{3} + o(\omega^{5})
}{
\frac{2\pi}{n} \omega -
\frac{4 \pi^{3}}{3 n^{3}} \omega^{3} + o(\omega^{5})
},
\]
}
por lo que

\textcolor{blue}{
\begin{align*}
\xi_{n, \omega} \sim &
\sqrt{2} 
\left(
4 \pi
\frac{                                                                                                                                          
\omega - \frac{2\pi^{2}}{3n^{2}}(2n^2-3n+2)\omega^{3} + o(\omega^{5})
}{
\frac{2\pi}{n} \omega -
\frac{4 \pi^{3}}{3 n^{3}} \omega^{3} + o(\omega^{5})
}
\right)^{-1/2} \\
\sim &
\sqrt{2} 
\left(
2n
\frac{                                                                                                                                          
\omega - \frac{2\pi^{2}}{3n^{2}}(2n^2-3n+2)\omega^{3} + o(\omega^{5})
}{
\omega -
\frac{2 \pi^{2}}{3 n^{2}} \omega^{3} + o(\omega^{5})
}
\right)^{-1/2} 
\rightarrow \frac{1}{\sqrt{n}},
\end{align*}
}

\textcolor{blue}{
\begin{align*}
\eta_{n, \omega} \sim &
\sqrt{2} 
\left(
\frac{
8 \pi^{3} (2n-1)(n-1)\omega^{3} + o(\omega^{5})
}{
6 \pi n \omega -
\frac{4 \pi^{3}}{n} \omega^{3} + o(\omega^{5})
}
\right)^{-1/2} \\
\sim & 
\sqrt{2} 
\left(
4\pi^{2}
\frac{
(2n-1)(n-1)\omega^{3} + o(\omega^{5})
}{
3 n \omega -
\frac{2 \pi^{2}}{n} \omega^{3} + o(\omega^{5})
}
\right)^{-1/2} 
 \rightarrow \infty.
\end{align*}
}

\textcolor{blue}{
\begin{align*}
\langle
c_{n, \omega}, s_{n, \omega}
\rangle \sim &
\frac{2\pi}{n}
\xi_{n, \omega} \eta_{n, \omega}
(n-1)  
\left(
\frac{n}{2} \omega - \frac{2\pi^{2}}{3} (n-1) \omega^{3} + o(\omega^{5})
\right) \\
= & 
\frac{2 \sqrt{2}}{n \sqrt{n}} \pi (n-1)
\left(
\frac{
\frac{3}{2} \pi n^{3}\omega^{3} + o(\omega^{5})
}{
8 \pi^{3}(2n-1)(n-1)\omega^{3} + o(\omega^{5})
}
\right)^{1/2}
\rightarrow \frac{
\sqrt{6(n-1)}
}{2 \sqrt{2n-1}}.
\end{align*} 
}
Tuve que usar la expresión asintótica
de $\eta_{n, \omega}$ para poder determinar este
último límite, pues con la expresión de la
primera linea tenía una indeterminación
de tipo $\infty \cdot 0$.

\textcolor{blue}{
\[
\langle x,
c_{n, \omega}
\rangle = 
\xi_{n, \omega} 
\left(
X_{0} - \frac{2 \pi^{2}}{n^{2}}X_{2} \omega^{2} 
+ \frac{2 \pi^{4}}{3n^{4}} X_{4} \omega^{4} + o(\omega^{5})
\right) \rightarrow \frac{X_{0}}{\sqrt{n}} ,
\]
}

\textcolor{blue}{
\begin{align*}
\langle x,
s_{n, \omega}
\rangle = &
\eta_{n, \omega} 
\left(
\frac{2 \pi}{n} X_{1} \omega - \frac{4 \pi^{3}}{3n^{3}}X_{3} \omega^{3} 
 + o(\omega^{5})
\right)\\
= &
\begin{cases}
\sqrt{2} \left(
\frac{
\frac{24 \pi^{3}}{n}X_{1}^{2}\omega^{3} + o(\omega^{5}) }{
8 \pi^{3} (2n-1)(n-1)\omega^{3} + o(\omega^{5})
}
\right)^{1/2}
\rightarrow \left(
\frac{6 X_{1}^{2}}{(2n-1)(n-1)n}
\right)^{1/2} & \textit{si } \alpha_{n, \omega}(x) \geq 0 \\
-\sqrt{2} \left(
\frac{
\frac{24 \pi^{3}}{n}X_{1}^{2}\omega^{3} + o(\omega^{5}) }{
8 \pi^{3} (2n-1)(n-1)\omega^{3} + o(\omega^{5})
}
\right)^{1/2}
\rightarrow -\left(
\frac{6 X_{1}^{2}}{(2n-1)(n-1)n}
\right)^{1/2} & \textit{si } \alpha_{n, \omega}(x) < 0,\\
\end{cases}
\end{align*}
}
donde
\[
\alpha_{n, \omega}(x) := 
\frac{2 \pi}{n} X_{1} \omega - \frac{4 \pi^{3}}{3n^{3}}X_{3} \omega^{3}.
\] 

\TODO{Ya lo tengo. Lo único no tan lindo de la fórmula
es que se requiere calcular el signo del coeficiente alpha. No sé si 
sea siempre fácil de determinar-o si quiera
si sea constante cerca de cero. Lo que yo hice en el programa es determiar
el signo para cuando $\omega = 0.001$}





\begin{equation}
\label{ec: limite x, cnw}
\limite{\omega \rightarrow 0^{+}}{\langle
x, c_{n, \omega}
\rangle }
= \frac{X_{0}}{\sqrt{n}}
\end{equation}

\begin{equation}
\label{ec: limite x, snw}
\limite{\omega \rightarrow 0^{+}}{\langle
x, s_{n, \omega}
\rangle }
\begin{cases}
\left(
\frac{6 X_{1}^{2}}{(2n-1)(n-1)n}
\right)^{1/2} & \textit{si } \alpha_{n, \omega}(x) \geq 0 \\
-\left(
\frac{6 X_{1}^{2}}{(2n-1)(n-1)n}
\right)^{1/2} & \textit{si } \alpha_{n, \omega}(x) < 0,\\
\end{cases}
\end{equation}

\begin{equation}
\label{ec: limite -2abc}
\limite{\omega \rightarrow 0^{+}}{
-2 \langle x, c_{n, \omega} \rangle
\langle x, s_{n, \omega} \rangle
\langle c_{n, \omega}, s_{n, \omega} \rangle
= 
}
\end{equation}

Así, los elementos que parecen en la fórmula
para $\sigma_{n}(x, \omega)$ cuando $\omega \not\in \frac{n}{2} \IZ$ son
\begin{itemize}
\item 
\TODO{
\[
\langle
c_{n, \omega}, s_{n, \omega}
\rangle^{2} \sim
\frac{4\pi^{2}}{n^{2}}(n-1)^{2} \xi_{n, \omega}^{2} \eta_{n, \omega}^{2}
\left(
\frac{n^{2}}{4} \omega^{2} - \frac{2n}{3} \pi^{2} (n-1) \omega^{4} + o(\omega^{5})
\right)
\]
}
sigue desarrollando el eta  !!!
\item
\TODO{
\[
\langle x,
c_{n, \omega}
\rangle^{2} = 
\xi_{n, \omega}^{2}
\left(
X_{0}^{2} - \frac{4 \pi^{2}}{n^{2}}X_{0}X_{2} \omega^{2} 
+ 
\frac{4 \pi^{4}}{n^{4}}
\left(
\frac{1}{3} X_{0}X_{4} + X_{2}^{2}
\right) \omega^{4} 
+ o(\omega^{5})
\right)
\]
}

\item
\TODO{
\[
\langle x,
s_{n, \omega}
\rangle^{2} = 
\frac{4 \pi^{2}}{n^{2}}
\eta_{n, \omega}^{2}
\left(
X_{1}^{2}\omega^{2} - \frac{4 \pi^{2}}{3n^{2}}X_{1}X_{3} \omega^{4} 
+  o(\omega^{5})
\right)
\]
}

\item
\[
\langle
x, c_{n, \omega}
\rangle
\langle
x, s_{n, \omega}
\rangle
\langle
c_{n, \omega}, s_{n, \omega}
\rangle \sim
\frac{2 \pi^{2}}{n}
\xi_{n, \omega}^{2} \eta_{n, \omega}^{2}(n-1)
\left(
X_{0}X_{1}\omega^{2} 
- \frac{2 \pi^{2}}{n}
\left(
\frac{1}{3n} X_{0}X_{3} + \frac{1}{n} X_{2}X_{1}
+ \frac{2}{3}(n-1)X_{0}X_{1}
\right) \omega^{4}
\right).
\]
\end{itemize}

\TODO{Cambia los iguales por $\sim$.}
Según estos cálculos, si
$a_{n, \omega} = \langle x, c_{n, \omega} \rangle$, 
$b_{n, \omega} = \langle x, s_{n, \omega} \rangle$, 
$c_{n, \omega} = \langle c_{n, \omega}, s_{n, \omega} \rangle$, 
entonces

\[
a_{n, \omega}^{2} + b_{n, \omega}^{2} - 
2a_{n, \omega}b_{n, \omega}c_{n, \omega}
= N_{1} + N_{2} +N_{3},
\]
donde

\[
N_{1} =
\xi_{n, \omega}^{2} \left(
X_{0}^{2} - \frac{4\pi^{2}}{n^{2}} X_{0}X_{2}
\omega^{2} + \frac{4 \pi^{4}}{n^{4}}
\left(
\frac{1}{3} X_{0}X_{4} + X_{2}^{2}
\right) \omega^{4}
\right)
\rightarrow 
\frac{X_{0}}{n}
\left(
X_{0}  - \frac{4 \pi^{2}}{n^{2}}X_{2}
\right),
\]

\[
N_{2} = \frac{4 \pi^{2}}{n^{2}} \eta_{n, \omega}^{2}
\left(
X_{1}^{2} \omega^{2} - \frac{4 \pi^{2}}{3n^{2}}X_{1}X_{3} \omega^{4}
+ o(\omega^{5})
\right)
=
\frac{16 \pi^{3}}{n^{2}}
\frac{3n \omega - \frac{2 \pi^{2}}{n} \omega^{3} + o(\omega^{5})}{
8 \pi^{3}(2n-1)(n-1) \omega^{3} + o(\omega^{5})
}
\rightarrow \infty,
\]

\[
N_{3} =
\frac{-4 \pi^{2}}{n}
\xi_{n, \omega}^{2} \eta_{n, \omega}^{2}(n-1)
\left(
X_{0}X_{1}\omega^{2} 
- \frac{2 \pi^{2}}{n}
\left(
\frac{1}{3n} X_{0}X_{3} + \frac{1}{n} X_{2}X_{1}
+ \frac{2}{3}(n-1)X_{0}X_{1}
\right) \omega^{4}
\right)
\rightarrow ?,
\]
luego, no puedo determinar este último límite, por lo tanto,
tampoco hablar solbre el límite de
de $a_{n, \omega}^{2} + b_{n, \omega}^{2} - 
2a_{n, \omega}b_{n, \omega}c_{n, \omega}$, pues
tengo una indeterminación del tipo 
$0 \cdot \infty$. \\

El denominador es
\[
1-c^{2} \sim
1 - \frac{4\pi^{2}}{n^{2}}(n-1)^{2} \xi_{n, \omega}^{2} \eta_{n, \omega}^{2}
\left(
\frac{n^{2}}{4} \omega^{2} - \frac{2n}{3} \pi^{2} (n-1) \omega^{4} + o(\omega^{5})
\right) \rightarrow ?,
\]
aquí también tengo una indeterminación de tipo
$0 \cdot \infty$.