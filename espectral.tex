\chapter{Estudio espectral}
\label{chap: estudio espectral}


\section{Motivación para realizar un estudio espectral de los PDL}
No es difícil convencerse de que las condiciones de ortogonalidad
impuestas en la definición de la base discreta de Legendre
\marginnote{En esta motivación informal, por ``oscilación''
de una señal $x = (x_{m})_{m=0}^{n-1}$
entendemos
tres cambios consecutivos de signo en sus entradas.}
$\cali{L}^{n}$
forzan a las entradas de los polinomios discretos $\cali{L}^{n,k}$
a cambiar más frecuentemente de signo conforme aumenta
el grado $k$, luego, conforme $k$ tiende a $n-1$,
la cantidad de oscilaciones aumenta; revisemos, 
por ejemplo, el caso $n=4$.



\begin{minipage}{0.5\textwidth}
\begin{figure}[H]
\includegraphics[scale=0.25]{oscil1}
\end{figure}
\end{minipage} \hfill
\begin{minipage}{0.45\textwidth}
1.- Por definición,
$\mathcal{L}^{4,0}$ se obtiene al normalizar 
al vector constante uno de $\IR^{4}$, o sea, 

\[
\cali{L}^{4,0} = \left(
\frac{1}{2}, \frac{1}{2}, \frac{1}{2}, \frac{1}{2}
\right).
\]
\end{minipage} 


\begin{minipage}{0.5\textwidth}
2.- La señal $\cali{L}^{4,1} \in \IR^{4}$ es un polinomio discreto de
dimensión 4 y grado 1 que se obtiene exigiendo las
siguientes condiciones

\[
\langle \cali{L}^{4,1} , \cali{L}^{4,0} \rangle=0
\hspace{0.2cm} \text{y} \hspace{0.2cm}
\langle \cali{L}^{4,1} , \cali{L}^{4,1} \rangle=1;
\]
esta primera condición se refleja en que 
las alturas de los puntos de la gráfica de 
$\cali{L}^{4,1}$ deben sumar cero;
esto implica un cambio de signo (y sólo uno,
pues el polinomio es lineal).

\end{minipage} \hfill
\begin{minipage}{0.45\textwidth}
\begin{figure}[H]
\includegraphics[scale=0.3]{oscil2}
\end{figure}
\end{minipage}

\noindent
3.- El vector
$\cali{L}^{4,2} \in \IR^{4}$,
que es un polinomio discreto
de grado $2$, satisface las siguientes
tres condiciones:

\[
\langle \cali{L}^{4,2} , \cali{L}^{4,0} \rangle=0,
\hspace{0.2cm}
\langle \cali{L}^{4,2} , \cali{L}^{4,1} \rangle=0,
\hspace{0.2cm} \text{y} \hspace{0.2cm}
\langle \cali{L}^{4,2} , \cali{L}^{4,2} \rangle=1.
\]
La segunda condición no da información sobre más
requerimientos que deba cumplir
$\cali{L}^{4,2}$, pues,
como $\cali{L}^{4,1} \in S_{n,-}$
y $\cali{L}^{4,2} \in S_{n,+}$
(c.f. teorema 
\ref{prop: simetrias en dimensiones pares}),
ya del lema
\ref{lema: ortogonalidad entre sim y antisim}
se deducía la ortogonalidad de estas señales; 
observe sin embargo que, si las entradas de 
$\cali{L}^{4,2}$ fuesen todas positivas o todas negativas,
entonces no se tendría la ortogonalidad
con la señal constante $\cali{L}^{4,0}$.


\begin{figure}[H]
\includegraphics[scale=0.6]{oscil3}
\end{figure}

4.- Por último, $\cali{L}^{4,3} \in \IR^{4}$ satisface
las siguientes cuatro condiciones: 

\[
\langle \cali{L}^{4,3} , \cali{L}^{4,0} \rangle=0,
\hspace{0.2cm}
\langle \cali{L}^{4,3} , \cali{L}^{4,1} \rangle=0,
\langle \cali{L}^{4,3} , \cali{L}^{4,2} \rangle=0,
\hspace{0.2cm} \text{y} \hspace{0.2cm}
\langle \cali{L}^{4,3} , \cali{L}^{4,3} \rangle=1.
\]

Según los teoremas 
\ref{cor: propiedades importantes de espacios Wi}
y \ref{prop: simetrias en dimensiones pares},
$\cali{L}^{4,3}$ es la discretización de un polinomio
cúbico y además es una señal antisimétrica (en el sentido de la definición
\ref{def: espacios de seniales simetricas y antisimetricas}); dos gráficas de señales que cumplen esto se ilustran abajo:

\begin{figure}[H]
	\includegraphics[scale=0.6]{oscil4}
	\sidecaption{Observe que la señal cúbica de la izquierda tiene
	un sólo cambio de signo, mientras que la de la derecha (que es 
	$\cali{L}^{4,3}$) tiene tres.}
\end{figure}

La señal cúbica de la izquierda, a pesar de 
ser ortogonal a $\mathcal{L}^{4,0}$ y a $\cali{L}^{4,2}$ por 
simetría (c.f. lema
\ref{lema: ortogonalidad entre sim y antisim}), 
definitivamente no puede ser ortogonal
a $\cali{L}^{4,1}$, pues las entradas de estos dos vectores de 
$\IR^{4}$ tienen el mismo signo. Sin embargo, la señal cúbica 
de la derecha (que de hecho es $\cali{L}^{4,3}$) sí cumple el ser
ortogonal a $\cali{L}^{4,1}$ pero, para lograr esto, sus
entradas deben cambiar de signo tres veces. \\


Después de graficar los PDL de otras dimensiones
puede apreciarse que esta tendencia a aumentar el número
de oscilaciones en las gráficas conforme el grado $k$
tiende a $n-1$ (su cota superior)
parece presentarse en todas las dimensiones.
Por ejemplo, abajo se muestran las gráficas de los 
PDL de dimensión $7$.

\begin{figure}[H]
	\centering
	\includegraphics[scale = 0.55]{oscilaciones_dim7} 
\end{figure}	

\section{Hipótesis sobre las oscilaciones de los PDL}

Con el objetivo de analizar la forma
de oscilación de los PDL, nos proponemos
hacer un análisis espectral. Antes
de proceder sistemáticamente, hagamos un análisis empírico.
\begin{figure}[H]
	\sidecaption{
	Con ``sinusoide'' nos referimos a una función 
	continua cuya forma se ilustra en la figura. Este queda
	completamente determinado al fijar una
	\textbf{amplitud}, una \textbf{frecuencia} y un
	\textbf{desfase}, que nosotros preferimos normalizar
	para que sea un número entre $0$ y $1$.
	\label{fig: sinusoide}
	}
	\centering
	\includegraphics[scale = 0.8]{sinusoide} 
\end{figure}	

Pongamos una dimensión de $n = 60$; a continuación,
graficamos los PDL de dimensión $60$ para los primeros grados,
e intentamos aproximar tales gráficas con sinusoides continuos.


\begin{figure}[H]
	\sidecaption{
	Aproximando las gráficas
	de $\cali{L}^{60,0}$ y $\cali{L}^{60,1}$
	con sinusoides. 
	\label{fig: hip_0,1}
	}
	\centering
	\includegraphics[scale = 0.6]{hip_0,1} 
\end{figure}	

\begin{figure}[H]
	\sidecaption{
	Aproximando las gráficas
	de $\cali{L}^{60,2}$ y $\cali{L}^{60,3}$
	con sinusoides. 
	\label{fig: hip_2,3}
	}
	\centering
	\includegraphics[scale = 0.6]{hip_2,3} 
\end{figure}	

\begin{figure}[H]
	\sidecaption{
	Aproximando las gráficas
	de $\cali{L}^{60,4}$ y $\cali{L}^{60,5}$
	con sinusoides. 
	\label{fig: hip_4,5}
	}
	\centering
	\includegraphics[scale = 0.6]{hip_4,5} 
\end{figure}	


Después de dibujar gráficas parecidas para varias dimensiones, 
establecemos la siguiente hipótesis de trabajo.

\begin{hip}
\label{ref: hipotesis}
Sean $n \geq 2$, $0 \leq k \leq n-1$ entero. 
Sea $\cali{L}^{n,k}$ el PDL de dimensión $n$ y grado $k$.
El espectro de $\cali{L}^{n,k}$ se concentra alrededor
de la frecuencia $\omega = k/2$.
\end{hip}

Se ha formulado de forma intencional la hipótesis 
\ref{ref: hipotesis} en términos vagos, para tener libertad
después de definir las herramientas con las que abordaremos el problema
de dar forma concreta a la hipótesis y mejorarla o refutarla con simulaciones
numéricas. \\ 

Observe que, para analizar la presencia de oscilaciones
en las gráficas de las señales, estamos hablando de un ``espectro''.
se cuenta ya con una herramienta clásica para hacer esta clase de análisis,
cuyo producto final es 
una función que
da información sobre
la importancia que cierta frecuencia tiene para
modelar la gráfica analizada. Hablamos de la transformada 
discreta de Fourier, cuyas bases teóricas comentamos en la sección
\ref{sec: TDF},
para usarla posteriormente para hacer un primer análisis espectral. Motivados
por algunas limitaciones de esta herramienta que comentamos más adelante,
nosotros proponemos una alternativa de metodología para realizar un análisis espectral 
en la sección 
\ref{sec: metodologia para realizar un analisis espectral que considere frecuencias arbitrarias}.

Algunos de los espectros así obtenidos se imprimen en
el capítulo
\ref{chap: resultados numericos analisis espectrales}.
El código para calcular tales espectros para cualquier
señal de $\IR^{n}$ se comparte en
\TODO{referencia cuaderno jupyter.}


\section{La transformada discreta de Fourier y estudios espectrales de señales finitas}

\TODO{aquí una introducción}
\TODO{La teoría se basa en el libro de Algoritmos.}

La teoría de esta sección se apoya en la existencia de las raíces
$n-$ésimas de la unidad (garantizada por el teorema fundamental
del álgebra \ref{teo: fundamental del algebra}) 
y propiedades de estas que hacen posible
tener un método eficiente de calcular lo que después llamaremos
la ``transformada discreta'' de un elemento de $\IC^{n}$.

\subsection{Raíces $n-$ésimas de la unidad y propiedades de estas}

\TODO{Habla sobre la exponencial compleja; tal vez pon su definición,
pero di que no entras en detalles (estos pueden consultarse, por ejemplo, en Marsden), sólo citamos unas propiedades de esta función de $\IC$ a $\IC$
que necesitaremos}

\TODO{Tal vez puedes hablar de cómo esta forma involucra al ángulo y
a la norma para representar a un punto, y por qué esto es útil
para hablar de mulitplicación y conjugados! También tienes que definir
la norma en $\IC^{n}$, esto no es trivial y de hecho me estaba dando
problemas.}

\begin{prop}
\label{prop: propiedades exp compleja}
(\textbf{Propiedades de la exponencial compleja}) a
	\begin{itemize}
	\item $exp(z) = 1$ si y sólo si $z= 2K \pi i$ para algún $K \in \IZ$
	\item Para todo $\omega \in \IZ$ y todo $z \in \IC$, $(exp(z))^{\omega} = 			exp(\omega z)$ \TODO{ve si $\omega$ puede de hecho ser cualquier complejo? 			tiene sentido esto?}
	\item Para cualesquiera $z_{1}, z_{2} \in \IC$, 
	$\frac{exp(z_{1})}{exp(z_{2})} = exp(z_{1} - z_{2})$.
	\item Para todo entero $n$ y todo real $b$,
	$(exp(bi))^{m} = exp (mb i)$
	\item \TODO{pon lo de exponencial de la suma.}
	\end{itemize}
\end{prop}


\begin{defi}
\label{defi: raices n esimas de la unidad}
Sea $n \in \IN$. A las $n$ raíces del polinomio
$p_{n}(t)= t^{n}-1$ se les denominará las \textbf{raíces $n-$ésimas de la unidad.}
\end{defi}


Las raíces $n-$ésimas de la unidad son pues los números complejos
tales que, elevados a la potencia $n$, son iguales a 1; según el 
teorema fundamental
del álgebra \ref{teo: fundamental del algebra}, sí hay números complejos
que satisfacen la definición \ref{defi: raices n esimas de la unidad}, y además
son a lo más $n$. Es fácil establecer, como hacemos a continuación, 
fórmulas explícitas \TODO{continua.}

\begin{prop}
Sea $n \in \IN$, $n \geq 2$. Hay exactamente $n$ raíces $n-$ésimas de la
unidad, y estas son los números complejos
 	\begin{equation}
	\label{eq3: 8ab}
	z_{n, \omega} : = exp \left( \frac{2 \pi i }{n} \omega
	\right), \hspace*{0.2cm} \textit{con} 
	\hspace*{0.2cm} \omega \in \{0, 1, \ldots, n-1 \}.
	\end{equation}
	
\end{prop}
\noindent
\textbf{Demostración.}
Por las propiedades expresadas en la proposición
\ref{prop: propiedades exp compleja}, es fácil ver que 
$z_{n,1} :=  exp \left( \frac{2 \pi i }{n} \right)$ es raíz $n-$ésima
de la unidad, pues
\[
(z_{n,1})^{n} = exp(2 \pi i ) = 1.
\]
Además, para todo $\omega \in \{ 0, \cdots , n-1 \}$, el número
\[
z_{n, \omega} : = (z_{n,1})^{\omega} = exp \left( \frac{2 \pi i }{n} \omega \right)
\]
también es es raíz $n-$ésima de la unidad, ya que

\[
(z_{n, \omega})^{n} = ((z_{n,1})^{\omega} )^{n} = 
((z_{n,1})^{n} )^{\omega} = 1^{\omega}=1. 
\]
Note ahora que los $n$ números complejos $z_{n, \omega}$ son todos 
distintos entre sí, pues si $\omega_{1}$ y $\omega_{2}$ son enteros
entre $0$ y $n-1$ tales que $z_{n, \omega_{1}} = z_{n, \omega_{2}}$,
o sea, tales que 
$exp \left( \frac{2 \pi i }{n} \omega_{1} \right) = 
exp \left( \frac{2 \pi i }{n} \omega_{1} \right)$, entonces, según el tercer
punto de la proposición \ref{prop: propiedades exp compleja},
$1 = exp \left( \frac{2 \pi i }{n} (\omega_{1}-\omega_{2}) \right)$, luego, 
según el primer punto de esta misma proposición, $\frac{\omega_{1}-\omega_{2}}{n}$
es entero, o sea, $n$ divide a $\omega_{1}-\omega_{2}$; por el rango de 
$\omega_{1}$ y $\omega_{2}$, esto sólo ocurre si $\omega_{1}-\omega_{2}$ es
cero, o sea, si $\omega_{1}$ y $\omega_{2}$
son iguales.
\QEDB
\vspace{0.2cm}

\TODO{Aquí la figura de siempre:)}



\subsection{DFT}




\begin{prop}
Sea $n \in \IN$. El conjunto

\begin{equation}
\label{eq2: 8ab}
\cali{B}_{n} : = \{
e_{\omega} = \left(
\frac{1}{\sqrt{n}} exp \left(
2 \pi i \omega \frac{m}{n}
\right)
\right)_{0 \leq m \leq n-1}
: \hspace{0.2cm} 0 \leq \omega \leq n-1
 \}
\end{equation}
es una base ortonormal del $\IC-$espacio
vectorial $\IC^{n}$.
\end{prop}

\noindent
\textbf{Demostración.}
Calculemos el producto punto de dos elementos
$e_{\omega_{1}}$ y $e_{\omega_{2}}$ del conjunto \eqref{eq2: 8ab};
si $\omega := \omega_{1}-\omega_{2}$,
\begin{align*}
\langle e_{w_{1}}, e_{w_{2}} \rangle = &
\frac{1}{n}
\suma{m=0}{n-1}{exp \left( 2 \pi i \frac{m}{n} \omega_{1} \right)
\cdot \overline{ exp \left( 2 \pi i \frac{m}{n} \omega_{2} \right) }} \\
= & \frac{1}{n}
\suma{m=0}{n-1}{\left( 2 \pi i \frac{m}{n} (\omega_{1}-\omega_{2}) \right)} \\
= & \frac{1}{n}\suma{m=0}{n-1}{exp\left( 2 \pi i \frac{\omega}{n} m \right)} \\
= & \frac{1}{n}\suma{m=0}{n-1}{exp\left( 2 \pi i \frac{\omega}{n}  \right)^{m}} \\
= & \frac{1}{n}\suma{m=0}{n-1}{(z_{n, \omega})^{m}};
\end{align*}

\noindent
esta última es una suma geométrica. 
\begin{itemize}
	\item Si $\omega_{1} \neq \omega_{2}$, entonces $n$ no puede dividir 
	a $\omega = \omega_{1}-\omega_{2}$ (pues, por el rango en el que se encuentran
	$\omega_{1}$ y $\omega_{2}$, $w \in [-(n-1), n-1]$, y el único múltiplo
	de $n$ en este intervalo es cero), luego, $z_{n, \omega} \neq 1$.
	En este caso se tiene entonces que 
	\[
	\langle e_{w_{1}}, e_{w_{2}} \rangle = 
	\frac{1}{n}\suma{m=0}{n-1}{(z_{n, \omega})^{m}}
	= \frac{1}{n} \cdot \frac{(z_{n, \omega})^{n}-1}{z_{n, \omega}-1}=
	\frac{1}{n} \cdot \frac{1-1}{z_{n, \omega}-1}=0.
	\]
	
	\item SI $\omega_{1} = \omega_{2}$, entonces $\omega = 0$, y
	\[
	\langle e_{w_{1}}, e_{w_{2}} \rangle = 
	\frac{1}{n}\suma{m=0}{n-1}{(z_{n, 0})^{m}}
	= \frac{1}{n}\suma{m=0}{n-1}{1} = \frac{1}{n} \cdot n = 1.
	\]
\end{itemize}

Demostramos así que los elementos de $\cali{B}_{n}$
tienen norma uno (c.f. \TODO{ref ec. norma en $\IC^{n}$}) y que además
son ortogonales
dos a dos, luego, según \TODO{ref}, $\cali{B}_{n}$ es un subconjunto l.i. 
de $\IC^{n}$; como $\IC^{n}$ es un $\IC-$ espacio vectorial de 
dimensión $n$, concluimos lo deseado.
\QEDB
\vspace{0.2cm}

Por ser \eqref{eq2: 8ab} una BON de $\IC^{n}$, siempre es
posible expresar a un vector $x = (x_{m})_{0 \leq m \leq n-1} \in \IC^{n}$
como combinación lineal de los elementos de \eqref{eq2: 8ab}
y además los coeficientes están dados por los productos puntos
de $x$ y los elementos de \eqref{eq2: 8ab}, que son

\begin{align*}
\langle x, e_{\omega} \rangle = & 
\frac{1}{\sqrt{n}} \suma{m=0}{n-1}{x_{m} exp \left(
2 \pi i \omega \frac{m}{n}
\right)} \\
= & 
\frac{1}{\sqrt{n}} \suma{m=0}{n-1}{x_{m} 
\left(
exp \left( \frac{2 \pi i }{n} \omega
\right) \right)^{m}} \\
= & A_{x}(z_{n, \omega}),
\end{align*}


\noindent
donde $z_{n, \omega}$ es como en \eqref{eq3: 8ab} y 
$A_{x} = A_{x}(t) \in \IC[t]$ es el polinomio de 
coeficientes complejos definido 
a partir de $x$ como sigue:

	\begin{equation}
		\label{eq4: 8ab}
		A_{x}(t) = \suma{m=0}{n-1}{\frac{x_{m}}{\sqrt{n}} t }\in \IC[t];
	\end{equation}

\noindent
así, \textbf{calcular los coeficientes de $x \in \IC^{n}$ respecto
a la BON $\cali{B}_{n}$ es lo mismo que evaluar al polinomio 
$A_{x}$ de grado $n-1$ definido en \eqref{eq4: 8ab} en todas las raíces
$n-$ésimas de la unidad.} Un algoritmo para evaluar eficientemente
polinomios es pues necesario.\TODO{cita el FFT}

\begin{defi}
Al proceso de calcular los coeficientes de $x$
respecto a $\cali{B}_{n}$
se le conoce como el \textbf{cálculo de la 
transformada discreta de $x$}
\end{defi}

{\Huge{\textcolor{red}{Dominio: tiempo}}} 


{\Huge{\textcolor{red}{Dominio: frecuencia}}}

{\Huge{ $x = (x_{m})_{0 \leq m \leq n-1}$ }}

{\Huge{ $\langle x, e_{\omega} \rangle$, $0 \leq \omega \leq n-1 $ }}



Dada la motivación de antes, es claro cómo usar la transformada
discreta de 

\textbf{a esto se le llama un análisis espectral.}


\subsection{Versión real de la DFT}

En el caso en el que todas las entradas de un vector
$x = (x_{m})_{0 \leq m \leq n-1}$ sean reales, se puede definir
una base ortonormal de $\IR^{n}$ que se defina en base a muestreos uniformes
de sinusoides de frecuencias enteras.
Esta base será entonces, así como lo era \TODO{ref} para el 
caso complejo, un sistema de representación en el que 

\TODO{sintetizar una señal en términos de frecuencias.}

\begin{prop}
\label{prop: base de fourier version real}
Sean $n \in \IN$ mayor a uno, $M = \lceil \frac{n}{2} \rceil$.
Para cualquier $\omega >0$, sean los vectores 

	\begin{equation}
	\label{eq0: 10ab}
	c_{n, \omega} := \left( \sqrt{\frac{2}{n}} cos
	\left(2 \pi \omega \frac{m}{n}
	\right) \right)_{0 \leq m \leq n-1}
	\hspace{0.2cm} \textit{y} \hspace{0.2cm} 
	s_{n, \omega} := \left( \sqrt{\frac{2}{n}} sin
	\left(2 \pi \omega \frac{m}{n}
	\right) \right)_{0 \leq m \leq n-1}.
	\end{equation}

El subconjunto $\cali{F}_{n}$ de $\IR^{n}$ definido como

	\begin{itemize}
	\item $\cali{F}_{n} : = \{ c_{n,0}, c_{n,1}, s_{n,1},
	\ldots , c_{n,M-1}, s_{n,M-1}, c_{n,M} \}$ si $n$ es par
	(o sea, si $n=2M$), y como
	\item $\cali{F}_{n} : = \{ c_{n,0}, c_{n,1}, s_{n,1},
	\ldots , c_{n,M-1}, s_{n,M-1} \}$ si $n$ es impar
	(o sea, si $n=2M-1$)
	\end{itemize}
	
es una base ortonormal del $\IR-$espacio vectorial $\IR^{n}$.
\end{prop}

\TODO{Deberías poner una imagen de cómo muestreas sinusoides para
obtener los vectores de la base. Tal vez deberías introducir el 
término ``vector de frecuencia''.}

\noindent
\textbf{Demostración.}
Supongamos $n$ par. Si $0 \leq \omega_{1}, \omega_{2} \leq M$
son enteros, entonces
$\omega_{1} + \omega_{2}$ sólo es divisible por $n$ si ambos números
son iguales a $M$. Si suponemos a $\omega_{1}$ y $\omega_{2}$ distintos, 
entonces

\begin{align*}
\langle c_{, \omega_{1}} , c_{n, \omega_{2}} \rangle = &
\frac{1}{n} \suma{m=0}{n-1}{cos \left(2 \pi \omega_{1} \frac{m}{n} \right) \cdot 
cos \left(2 \pi \omega_{2} \frac{m}{n} \right)} \\
= &\frac{1}{2n} \left(
cos \left(2 \pi (\omega_{1} + \omega_{2}) \frac{m}{n} \right) +
cos \left(2 \pi (\omega_{1} - \omega_{2}) \frac{m}{n} \right)
\right) \\
= & \frac{1}{4n} (
\suma{m=0}{n-1}{
(exp(2 \pi m(\omega_{1}+\omega_{2})i/n) +
exp(-2 \pi m(\omega_{1}+\omega_{2})i) } \\
&  + exp(2 \pi m(\omega_{1}-\omega_{2})i/n) +
exp(-2 \pi m(\omega_{1}-\omega_{2})i)) )\\
\textit{(suma geométrica)} = & 
\frac{exp(2 \pi i (\omega_{1}+\omega_{2}))-1}{4n (exp(2 \pi i (\omega_{1}+\omega_{2})/n)-1)} +
\frac{exp(- 2 \pi i (\omega_{1}+\omega_{2}))-1}{4n (exp(-2 \pi i (\omega_{1}+\omega_{2})/n)-1)}
\\
& + 
\frac{exp(2 \pi i (\omega_{1}-\omega_{2}))-1}{4n (exp(2 \pi i (\omega_{1}-\omega_{2})/n)-1)} +
\frac{exp(- 2 \pi i (\omega_{1}-\omega_{2}))-1}{4n (exp(-2 \pi i (\omega_{1}-\omega_{2})/n)-1)};
\\
\end{align*}

\noindent
puesto que $\omega_{1}+\omega_{2}$ y $\omega_{1}-\omega_{2}$
son ambos enteros, según la proposición 
\ref{prop: propiedades exp compleja} las exponenciales de los numeradores
de esta última expresión son todas iguales a uno, luego, 
$\langle c_{n, \omega_{1}} , c_{n, \omega_{2}} \rangle  =0$. 


Con argumentos similares se prueba 
que todos los elementos de $\cali{F}_{n}$ tienen norma uno, así como
la ortogonalidad entre dos elementos
distintos del conjunto $\cali{F}_{n}$, por lo tanto, la independencia lineal de
este conjunto, luego, el que $\cali{F}_{n}$ sea base 
(ortonormal) de $\IR^{n}$.


\QEDB
\vspace{0.2cm}



\begin{defi}
Sea $n \in \IN$, $n \geq 2$. Llamaremos a la BON
$\cali{F}_{n}$ de $\IR^{n}$ definida en \ref{prop: base de fourier version real}
la \textbf{base de Fourier real de dimensión $n$}.
\end{defi}

Observe que $\cali{F}_{n}$, a diferencia de $\cali{B}_{n} \subseteq \IC^{n}$, 
considera frecuencias enteras no mayores a $M := \lceil \frac{n}{2} \rceil$
(cuando $n$ es par) o a $M-1$ (cuando $n$ es impar), mientras que
en $\cali{B}_{n}$ se consideran las frecuencias enteras entre $0$
y $n-1$ (inclusivo). Es decir, si 
\TODO{vamos a sintentizar a una señal respecto a menos frecuencias enteras.} 



\textbf{Ejemplo:} Consideremos a la señal 
\begin{equation}
\label{eq2: 10ab}
x=(-0.5,-8,-5.3,15,-0.3,6,4) \in \IR^{7}.
\end{equation}

Según la construcción de $\cali{F}_{7}$ (c.f. 
proposición \ref{prop: base de fourier version real}),
una expresión de $x$ respecto a $\cali{F}_{7}$ 
es una síntesis de $x$ a partir de señales 
de frecuencias $\omega = 0,1,2,3$. En la imágen de abajo
se muestran los coeficientes de $x$ respecto a $\cali{F}_{7}$.

\begin{figure}[H]
	\sidecaption{
	Se muestran la gráfica de $x$ junto con la gráfica de los
	coeficientes de $x$ respecto a la BON $\cali{F}_{7}$. Observe 
	que, por definición, sólo un vector de $\cali{F}_{7}$ tiene frecuencia
	cero (i.e. es constante), mientras que para las otras frecuencias
	tenemos dos vectores de la misma frecuencia, uno construido a partir de un 			coseno y otro a partir de un seno.
	\label{fig: ejFrecuencia 1}
	}
	\centering
	\includegraphics[scale=0.4]{ejFrecuencia_1} 
\end{figure}	

Se tiene la siguiente descomposición
\sidenote{Se redondearon los coeficientes.} de $x$;

\[
x = 4.12 c_{0} - 8.76c_{1} -7.35s_{1}+
4.77c_{2}-10s_{2}+0.14c_{3}+9.91s_{3}.
\]
A continuación mostramos las gráficas
de los sinusoides que fueron discretizados
para obtener los vectores de frecuencia
$0,1,2$ y $3$ en los que descompusimos a $x$.

\begin{figure}[H]
	\sidecaption{
	Aporte de frecuencia $0$.
	\label{fig: ejFrecuencia 2}
	}
	\centering
	\includegraphics[scale=0.4]{ejFrecuencia_2} 
\end{figure}	

\begin{figure}[H]
	\sidecaption{
	Aporte de frecuencia $1$.
	\label{fig: ejFrecuencia 3}
	}
	\centering
	\includegraphics[scale=0.4]{ejFrecuencia_3} 
\end{figure}	

\begin{figure}[H]
	\sidecaption{
	Aporte de frecuencia $2$.
	\label{fig: ejFrecuencia 4}
	}
	\centering
	\includegraphics[scale=0.4]{ejFrecuencia_4} 
\end{figure}	


\begin{figure}[H]
	\sidecaption{
	Aporte de frecuencia $3$.
	\label{fig: ejFrecuencia 5}
	}
	\centering
	\includegraphics[scale=0.4]{ejFrecuencia_5} 
\end{figure}	

Sumando todas las gráficas de la derecha, obviamente
obtenemos una función de cosenos y senos tal que,
al muestrearla uniformemente en $[0,1]$, obtenemos
al vector $x$ \eqref{eq2: 10ab}.

\begin{figure}[H]
	\sidecaption{
	En morado se muestra la gráfica de la función suma
	de las gráficas derechas en las figuras anteriores.
	\label{fig: ejFrecuencia 6}
	}
	\centering
	\includegraphics[scale=0.45]{ejFrecuencia_6} 
\end{figure}	


\TODO{Tal vez el algoritmos de la FFT no merezca una subsección. Mejor sólo 
exboza los detalles y cite al libro.}
\TODO{Para esto puedes apoyarte mucho en las notas del libro 
de Algorithms. Eso sí lo entendí bien.}























\section{Metodología para realizar un análisis espectral que considere frecuencias arbitrarias}
\label{sec: metodologia para realizar un analisis espectral que considere frecuencias arbitrarias}

Ya podemos usar la base de Fourier real $\cali{F}_{n}$
definida en la proposición \ref{prop: base de fourier version real}
para hacer un estudio espectral de los PDL. \\

Puesto que, por la construcción de $\cali{F}_{n}$, 
hacer un análisis espectral de una señal $x \in \IR^{n}$
via su análisis respecto a la BON $\cali{F}_{n}$ nos lleva
a considerar sólo ciertas frecuencias enteras
(c.f. nota \ref{nota: frecuencias en las bases de fourier}),
queremos no sólo usar la TDF para realizar
nuestro estudio espectral, pues
no queremos restringirnos
al estudio de frecuencias enteras
(después de todo, según la hipótesis planteada en 
\ref{ref: hipotesis}, 
creemos que la frecuencia que mejor aproxima al PDL
$\cali{L}^{n,k}$ es $\frac{k}{2}$, y este último número no siempre
es un entero), sino que nos gustaría
\begin{enumerate}
	\item poder elegir una frecuencia $\omega \geq 0$ respecto
a la cual comparar a la señal y,
	\item una vez fijada una frecuencia, buscar el desfase $\phi \in [0,1]$
	que mejor ajuste la gráfica de $x$.
\end{enumerate}

\begin{figure}[H]
	\sidecaption{
	Aquí se grafica una misma señal $x \in \IR^{16}$ y se 
	compara con dos sinusoides de frecuencia $3.6$, una con 
	desfase (normalizado) 0.8 y otra con 0.32. Observe que
	la primera parece ajustar mucho mejor la gráfica de $x$.
	\label{fig: ejemplo desfase}
	}
	\centering
	\includegraphics[scale=0.45]{desfase_ejemplo} 
\end{figure}	


Vamos a seguir
una linea de razonamiento totalmente análoga a la empleada 
en el ejemplo \ref{subs: ejm 3}, pues aquí también abordamos el problema
definiendo subconjuntos (de hecho, subespacios)
de $\IR^{n}$ que consten de elementos que cumplan
determinada propiedad (en el caso del ejemplo \ref{subs: ejm 3}, la propiedad
era ser elemento de determinado espacio 
de polinomios discretos $W_{n,k}$, mientras que 
en esta sección la propiedad de nuestro interés es ``ser la discretización
de un sinusoide de frecuencia $\omega$'') y usando el coseno del ángulo que
una señal $x$ forma con dichos subconjuntos para dar una medida de qué tanto
tiene $x$ la propiedad considerada.


\subsection{Espacios monofrecuenciales}

\begin{notacion}
Para simplificar la notación, denotamos por $I_{n}$ al intervalo
$\{ \frac{m}{n}  : 0 \leq m \leq n-1 \}$.
\end{notacion}

Digamos qué es lo que 
entendemos por ``señal de frecuencia pura $\omega$''.

\begin{defi}
Sean $n \in \IN$,  $\omega>0$, $\phi \in [0,1[$.  
A toda señal $n-$dimensional  
de la forma

\begin{equation}
A \left(
cos \left(  2 \pi \omega t + 2 \pi \phi
\right)
\right)_{t \in I_{n}}
\end{equation}

\noindent
con $A \in \IR$, se le llamará
\textbf{señal $n-$dimensional de frecuencia
pura $\omega$}. En este contexto,
a $\phi$ se le llama el \textbf{desfase normalizado}
de la señal, y a $A$ la \textbf{amplitud}.
\end{defi}

Note que los vectores
$c_{n, \omega}$ y $s_{n, \omega}$
definidos en la proposición
\ref{prop: base de fourier version real} son
señales $n-$dimensionales de frecuencia pura $\omega$.

\begin{nota}
Observe que toda señal de la forma
\begin{equation*}
A \left(
sin \left(  2 \pi \omega t + 2 \pi \phi
\right)
\right)_{t \in I_{n}},
\end{equation*}
con $A \in \IR$, también es una señal $n-$dimensional
de frecuencia pura $\omega$, pues, como 
\[
sen(x) = - cos (x+ \pi/2) \hspace{0.2cm}
\textit{para toda } x \in \IR,
\]
entonces
\begin{equation*}
A \left(
sin \left(  2 \pi \omega t + 2 \pi \phi
\right)
\right)_{t \in I_{n}} =
-A \left(
cos \left(  2 \pi \omega t + 2 \pi \phi^{'}
\right)
\right)_{t \in I_{n}},
\end{equation*}
donde $\phi^{'}= \phi + 1/4$.
\end{nota}


\begin{figure}[H]
	\sidecaption{
	Se grafica a la función 
	$f(t) = cos(2 \pi \cdot \frac{5}{2} t + 2 \pi \cdot 0.3)$;
	muestreando este sinusoide de forma uniforme con $n$
	puntos en el 
	intervalo [0,1] obtenemos una señal $n-$dimensional
	de frecuencia pura
	$\omega = \frac{5}{2}$. En la figura, $n=20$.
	\label{fig: desfasee ejemplo grafico}
	}
	\centering
	\includegraphics[scale= 0.55]{muestreo_coseno} 
\end{figure}	

\begin{prop}
\label{prop: para que frecuencias omega vector seno es cero}
	Sean $n \geq 2$, $\omega \geq 0$.
	\begin{itemize}
		\item El vector 
		\begin{equation}
		\label{eq: coseno omega}
		\tilde{c}_{n, \omega} = \left(cos(2 \pi \omega m/n) \right)_{m=0}^{n-1} \in \IR^{n}
		\end{equation}
		no es cero, y 
		\item el vector 
		\begin{equation}
		\label{eq: seno omega}
		\tilde{s}_{n, \omega} = \left(sen(2 \pi \omega m/n) \right)_{m=0}^{n-1} \in \IR^{n}
		\end{equation}
		es cero si y sólo si $\omega \in \frac{n}{2} \IZ$.
	\end{itemize}
\end{prop}
\noindent
\textbf{Demostración.}
El primer punto es fácil de probar, pues la primera entrada del
vector \eqref{eq: coseno omega} es 
$cos(0)=1$.

Supogamos ahora que $\omega>0$ es tal que \eqref{eq: seno omega}
es el vector cero, o sea, que
para toda $0 \leq m \leq n-1$, se tiene que 
$sen(2 \pi \omega m/n)=0$. En particular, ocurre
$sen(2 \pi \omega /n)=0$; esto implica la igualdad 
$2 \pi \omega /n = \pi K$ para algún entero $K$. Despejando
a $\omega$ de la ecuación tenemos que 
$\omega = \frac{n}{2}K \in \frac{n}{2} \IZ$. Recíprocamente,
todo $\omega$ de la forma 
$\frac{n}{2}K$, con $K \in \IZ$ hace que el vector 
\eqref{eq: seno omega} sea cero, pues, para toda $0 \leq m \leq n-1$,
$sen\left(2 \pi \frac{n}{2}K \frac{m}{n}\right)=
sen((Km)\pi)=0$.
\QEDB
\vspace{0.2cm}

Nos interesará considerar al subespacio
de $\IR^{n}$ generado por las señales
de frecuencia $\omega$ $\tilde{c}_{n, \omega}$
y $\tilde{s}_{n, \omega}$, o sea, a 

\begin{equation}
\label{eq: espacio Pnw}
P_{n, \omega} := span(\tilde{c}_{n, \omega}, \tilde{s}_{n, \omega}).
\end{equation} 
caracterizamos a los elementos de 
espacios de la forma
$P_{n, \omega}$ a continuación.

\begin{teo}
\label{prop: Pw consta de las señales de frecuencia omega}
Sean $n \in \IN$, $\omega \geq 0$ 
con 
$\omega \not\in \frac{n}{2} \IZ$.
El espacio $P_{n, \omega}$ definido en \eqref{eq: espacio Pnw} consta exactamente
de las señales $n$ dimensionales de frecuencia $\omega$.
\end{teo}

\noindent
\textbf{Demostración.}

Sea $\phi \in [0,1]$ un desfase cualquiera y $A \in \IR$
una amplitud cualquiera; por la regla
del coseno de la suma de dos ángulos, tenemos que
\[
A(cos (2 \pi \omega t + 2 \pi \phi))_{t \in I_{n}}
= Aa  (cos(2 \pi \omega t))_{t \in I_{n}} +
Ab  (sen(2 \pi \omega t))_{t \in I_{n}} \in P_{n, \omega} 
\]
donde
\[
a := cos (2 \pi \phi) \hspace{0.2cm} \text{y}
\hspace{0.2cm} b := sin (2 \pi  \phi).
\]


Recíprocamente, si $a, b \in \IR$ son escalares cualesquiera, 
el elemento genérico
$x=  a \left( cos \left(2 \pi \omega t \right) \right)_{t \in I_{n}} +
b ( sen (2 \pi \omega t ))_{t \in I_{n}} $ de $P_{n,w}$ puede
expresarse como sigue:

\begin{equation}
\label{eq1: 28Mar23}
x = \sqrt{a^{2}+b^{2}} \left(
A  \left( cos \left(2 \pi \omega t \right) \right)_{t \in I_{n}} +
B  \left( sin \left(2 \pi \omega t \right) \right)_{t \in I_{n}}
\right),
\end{equation}

\noindent
donde
\[
A := \frac{a}{\sqrt{a^{2}+b^{2}}} \hspace{0.2cm} \text{y} \hspace{0.2cm}
B := \frac{b}{\sqrt{a^{2}+b^{2}}}.
\]
Como $A^{2}+ B^{2}=1$, existe $\phi \in [0,1]$ tal que
\begin{equation}
\label{eq0: 28Mar23}
A = cos (2 \pi \phi) \hspace{0.2cm} \text{y}  \hspace{0.2cm}
B = sin (2 \pi \phi);
\end{equation}
sustituyendo \eqref{eq0: 28Mar23} en \eqref{eq1: 28Mar23}, llegamos
a que

\begin{align*}
x = &  \sqrt{a^{2}+b^{2}} (
cos(2 \pi \phi) \cdot (cos (2 \pi \omega t))_{t \in I_{n}} + 
sin(2 \pi \phi) \cdot (sin (2 \pi \omega t))_{t \in I_{n}} 
) \\
= & \sqrt{a^{2}+b^{2}} (
cos(2 \pi \phi) \cdot cos (2 \pi \omega t) +
sin(2 \pi \phi) \cdot sin (2 \pi \omega t) 
)_{t \in I_{n}}  \\
= &  \sqrt{a^{2}+b^{2}} (
cos (2 \pi \omega t - 2 \pi \phi)
)_{t \in I_{n}}.
\end{align*}

\QEDB
\vspace{0.2cm}

\begin{defi}
Si $n \geq 2$ y $\omega>0$, entonces
al subespacio $P_{n,\omega}$ 
de $\IR^{n}$
definido en \eqref{eq: espacio Pnw} 
le llamaremos el \textbf{espacio monofrecuencial
$n$ dimensional} de frecuencia $\omega$.
\end{defi}

\begin{obs}
\label{obs aa: f y g son l.i. y de norma uno}
Sean $n \geq 2$ entero, $\omega \geq 0$ con $\omega \not\in \frac{n}{2} \IZ$.
Los vectores \eqref{eq: coseno omega} y \eqref{eq: seno omega}
de $\IR^{n}$ son linealmente independientes.
\end{obs}
\noindent
\textbf{Demostración.}
Sólo note que 
la primera entrada de \eqref{eq: coseno omega} es $1$, mientras que  
la primera entrada de $\tilde{s}_{n, \omega}$ es cero pero no
todas sus entradas lo son (c.f. proposición 
\ref{prop: para que frecuencias omega vector seno es cero}). 
\QEDB
\vspace{0.2cm}


Según la observación 
\ref{obs aa: f y g son l.i. y de norma uno}, si $\omega \not\in \frac{n}{2} \IZ$,
el espacio $P_{n,\omega}$ que generan los vectores 
\eqref{eq: coseno omega} y \eqref{eq: seno omega}

\begin{align}
\label{eq6: 23Ap}
P_{n,\omega}:= & span( \tilde{c}_{n, \omega}, 
\tilde{s}_{n, \omega}) \notag  \\  
= &
\{ a \left( cos \left(2 \pi \omega t \right) \right)_{t \in I_{n}} +
b ( sen (2 \pi \omega t ))_{t \in I_{n}} : 
\hspace{0.2cm} a, b \in \IR \},
\hspace{0.1cm} \omega \not\in \frac{n}{2} \IZ
\end{align}

\noindent
es un plano (i.e. un subespacio de dimensión $2$) de $\IR^{n}$
que además, según el teorema 
\ref{prop: Pw consta de las señales de frecuencia omega},
consta exactamente de las señales
de dimensión $n$ y frecuencia (pura) $\omega$. \\

Si $\omega \in \frac{n}{2} \IZ$, entonces, según 
la proposición 
\ref{prop: para que frecuencias omega vector seno es cero}, el vector
$\tilde{s}_{n, \omega}$ es el vector cero y 
$\tilde{c}_{n, \omega}$ no, luego, el espacio
\begin{align}
\label{eq0: 23Ap}
P_{n,\omega}:= & span(c_{n, \omega}, s_{n, \omega}) \notag  \\  
= &
\{ a \left( cos \left(2 \pi \omega t \right) \right)_{t \in I_{n}} : 
\hspace{0.2cm} a \in \IR \},
\hspace{0.1cm} \omega \in \frac{n}{2} \IZ
\end{align}
es una recta (i.e. un subespacio de dimensión $1$)
de $\IR^{n}$.

\begin{nota}
Sean $n \geq 2$, $\omega \in \frac{n}{2} \IZ$ 
una frecuencia
mayor o igual a cero; digamos que 
$\omega = \frac{n}{2}K$. Entonces, 
según la proposición 
\ref{prop: para que frecuencias omega vector seno es cero},
$s_{n, \omega} =0$ y
\[
\tilde{c}_{n, \omega} = \left(cos\left( 2 \pi \frac{n}{2}K \frac{m}{n}\right)
\right)_{m=0}^{n-1}
= (cos(mK \pi))_{m = 0}^{n-1} = ((-1)^{mK})_{m=0}^{n-1},
\]
luego, 
fijada una dimensión $n$, sólo hay dos espacios
$P_{n, \omega}$ cuando 
$\omega \in \frac{n}{2} \IZ$,
a saber,
\[
\{ (a)_{m=0}^{n-1} : \hspace{0.2cm} a \in \IR \} \subseteq \IR^{n}
\]
y
\[
\{ ((-1)^{m}a)_{m=0}^{n-1} : \hspace{0.2cm} a \in \IR \} \subseteq \IR^{n}.
\]

\end{nota}

\subsection{Fórmulas para el coseno del ángulo de un punto a un plano}
\label{ap: Caso particular en el que el subespacio en cuestión es un plano}

Para proponer una 
metodología 
alternativa al uso de la TDF
para realizar un análisis espectral,
necesitaremos medir ángulos de señales a planos
(i.e. subespacios de dimensión $2$).
En esta subsección nos dedicamos a 
obtener expresiones que usaremos después 
\TODO{aaa}. Vamos pues a 
desarrollar la teoría de la sección
para este caso particular.

La situación es la siguiente: $V$ es un $\IR-$espacio
de Hilbert, $u$ y $v$ son elementos de $V$,
unitarios y linealmente
independientes entre sí. El espacio que ellos generan
es pues un plano, digamos,


\[
P := span \{ u, v \}.
\]

\noindent
Dado $x \in V$,
el coseno del ángulo entre $x$ y $W$ es,
según la proposición
\ref{prop: algunos hechos sobre el angulo entre un vector y un subespacio},

\begin{equation}
\label{eq0: 19Marzo}
cos \left( \measuredangle (x, P) \right) = 
\frac{|| \Pi_{P}(x) ||}{||x||};
\end{equation}
para lograr expresar el lado derecho de la igualdad en términos
sólo de $u$, $v$ y $x$ (que son los elementos básicos de
nuestra discusión), conviene primero obtener, a partir 
de estos elementos, una base
ortonormal del espacio $P$.


\begin{obs}
Si $u, v \in V$ son unitarios y linealmente independientes, y $P$
es el plano que generan, entonces
$\{ u, z \}$, donde

\begin{equation}
\label{eq2: 19Marzo}
z:= \frac{v- \langle u, v \rangle u}{||v- \langle u, v \rangle u||}
\end{equation}
es una BON de $P$
\end{obs}
\noindent
\textbf{Demostración.}
Basta aplicar el teorema de Gram-Schmidt 
\ref{Teo:Gram-Schmidt}.
\QEDB
\vspace{0.2cm}

Teniendo una BON de $P$, según el 
corolario 
\ref{cor: proyeccion en terminos de BON}, se tiene la siguiente
expresión para la proyección de $x$ en $P$;

\begin{equation}
\label{eq1: 19Marzo}
\Pi_{P}(x)= \langle x, u \rangle u + \langle x, z \rangle z;
\end{equation}

\noindent
puesto que, según la definición \eqref{eq2: 19Marzo} de 
$z$ este vector es función de $u$ y $v$, fácilmente se
puede derivar, a partir de \eqref{eq1: 19Marzo},
una expresión de $\Pi_{P}(x)$ en función sólo
de $x$, $u$ y $v$. Se plasman las fórmulas 
concretas a continuación.
	\begin{prop}
	\label{prop: formulas 20Marzo}
	Sean $V$ un espacio de Hilbert, $x \in V$,
	$u,v \in V$ linealmente independientes
	y unitarios. Si $P$ es el plano
	que generan $u$ y $v$, entonces,

		\begin{equation}
		\label{eq0: 24ap}
		\Pi_{P}(x)= \frac{\langle x, u \rangle -\langle u, v \rangle \langle x, v \rangle }{1-\langle u, v \rangle^{2}} u + \frac{\langle x, v \rangle -\langle u, v \rangle \langle x, u \rangle }{1-\langle u, v \rangle^{2}} v
		\end{equation}
	y 
		\begin{equation}
		\label{eq3: 19Marzo}
		  || \Pi_{P}(x) ||^{2}=
		  \frac{\langle x, u \rangle^{2} +  \langle x, v \rangle^{2}	
	       -2  \langle x, u \rangle \langle x, v \rangle \langle u, v \rangle	}{1- \langle u, v 		\rangle^{2}}.
		\end{equation}
 
	\end{prop}

\noindent
\textbf{Demostración.}
La demostración consiste de simples manipulaciones aritméticas.
Según \eqref{eq1: 19Marzo},
\begin{align*}
\Pi_{P}(x) = & \langle x, u \rangle u + \langle x, z \rangle z \\
 = & \langle x, u \rangle u
 + \frac{\langle x, v \rangle - \langle u, v \rangle \langle x, u \rangle}{|| v -\langle u,v \rangle u ||^{2}}
(v - \langle u,v \rangle u);\\
\end{align*}

\noindent
puesto que $u$ y $v$ son unitarios, 
tenemos que
\begin{align}
\label{eq3: 23ap}
|| v -\langle u,v \rangle u ||^{2} = & 
\langle v,v \rangle^{2} -2
\langle u,v \rangle^{2} +\langle u,v \rangle^{2}\langle u,u \rangle \notag  \\
= & 1 -\langle u,v \rangle^{2}; 
\end{align}
sustituyendo \eqref{eq3: 23ap} en la última expresión para 
$\Pi_{P}(x)$ llegamos a \eqref{eq3: 19Marzo}. \\

Finalmente, 
\begin{align*}
|| \Pi_{P}(x) ||^{2} = & 
\langle x,u \rangle^{2} + \langle x,z \rangle^{2} \\
= & \langle x,u \rangle^{2} + 
\left(
\frac{\langle x,v \rangle - \langle u,v \rangle
\langle x,u \rangle}{||v -\langle u,v \rangle u ||}
\right)^{2};\\
\end{align*}

\noindent
sustituyendo \eqref{eq3: 23ap} en esta última expresión
llegamos a \eqref{eq3: 19Marzo}.

\QEDB
\vspace{0.2cm}

Usando las expresiones
\eqref{eq: coseno a subespacio}
y \eqref{eq3: 19Marzo} es fácil establecer
la siguiente proposición.

\begin{prop}
Sean $V$ un espacio de Hilbert, $x \in V$,
	$u,v \in V$ linealmente independientes
	y unitarios. Si $P$ es el plano
	que generan $u$ y $v$, entonces,
	
	
\begin{equation}
\label{eq: coseno a plano}
cos (\measuredangle (x, P)) = 
\sqrt{
\frac{\langle x, u \rangle^{2} +  \langle x, v \rangle^{2}	
	       -2  \langle x, u \rangle \langle x, v \rangle \langle u, v \rangle	}{
	       ||x||^{2} \cdot 
	       (1- \langle u, v 	\rangle^{2})  }}.
\end{equation}
\end{prop}

\subsection{Estudio espectral basado en ángulos a espacios monofrecuenciales}

Definidos los espacios monofrecuenciales
$P_{n, \omega} \subseteq \IR^{n}$ 
(c.f. \eqref{eq: espacio Pnw})
y caracterizados sus elementos como las señales
$n-$dimensionales de frecuencia pura $\omega$,
parece razonable
medir la cercanía de una señal $n-$dimensional $x \in \IR^{n}$
a tener frecuencia $\omega$
con el ángulo que $x$ forma con el subespacio $P_{n, \omega}$,
cuyo coseno, según la proposición
\ref{prop: algunos hechos sobre el angulo entre un vector y un subespacio}
es
\begin{equation}
\label{eq0: 20Mar}
cos \left( \measuredangle (x, P_{n, \omega}) \right) = 
\frac{|| \Pi_{P_{n, \omega}}(x) ||}{||x||}
\in [0,1].
\end{equation}


\begin{figure}[H]
	\sidecaption{
	Según la relación \eqref{eq0: 20Mar}, 
	si $\frac{||\Pi_{P_{\omega}}(x)||}{||x||}$ es cercano 
	a uno (resp. a cero), entonces $x$ es muy parecido a una señal de frecuencia $\omega$
	(resp. se aleja de ser una señal de frecuencia $\omega$).
	\label{fig: 20Mar23_1}
	}
	\centering
	\includegraphics[scale= 1]{20Mar23_1} 
\end{figure}	

Si $x$ es unitaria,
tenemos la relación simplificada 

\begin{equation}
\label{eq1: 20Mar}
cos \left( \measuredangle (x, W) \right) = || \Pi_{W}(x) || 
\hspace{0.5cm} (x \hspace{0.1cm} \text{unitario).}
\end{equation}


\textbf{Usaremos pues, para dar una medida de qué tanto
reacciona una señal $x \in \IR^{n}$ a una frecuencia
$\omega >0$
el número 
\[\frac{||\Pi_{P_{n, \omega}}(x)||}{||x||} \in [0,1].\]} \\

\begin{defi}
\label{def: final de sigmas}
Sean $n \geq 2$, $\omega \geq 0$. 
Definimos la función $\sigma_{n}(\cdot, \omega)$ para todo 
elemento de $\IR^{n} - \{ 0\}$
como sigue;
\begin{equation}
\label{eq: def sigmas}
	\forall x \in \IR^{n}-\{ 0\}: \hspace{0.2cm}
	\sigma_{n}(x, \omega) =
	cos \left( \measuredangle (x, P_{n, \omega}) \right) = 
	\frac{||\Pi_{P_{n, \omega}}(x)||}{||x||} .
\end{equation}
\end{defi}

\begin{nota}
\label{nota: significado de los sigma en AE}
Fijada una frecuencia $\omega$, 
\begin{itemize}
\item si $\sigma_{n, \omega}(x)$ es ``cercano'' a cero, $\omega$ no
es una frecuencia con la que es razonable aproximar a $x$ (pues $x$ será
cercano a ser ortogonal a toda señal de dimensión $n$ y frecuencia 
$\omega$),  mientras que

\item si $\sigma_{n, \omega}(x)$ es ``cercano'' a uno, también es muy cercano
(hablando en términos de distancia euclídea) a su proyección al espacio
$P_{n, \omega}$, luego $x$ es muy parecido a tener frecuencia $\omega$.
\end{itemize}
\end{nota}



Para poder usar las fórmulas
derivadas en la subsección 
\ref{ap: Caso particular en el que el subespacio en cuestión es un plano},
debemos de dar una BON del espacio $P_{n, \omega}$.

\begin{prop}
\label{prop: aaa}
Sean $n \in \IN$, $\omega>0$. Sean los vectores 
$\tilde{c}_{n, \omega}, 
\tilde{s}_{n ,\omega} \in \IR^{n}$ 
como se definieron en 
\eqref{eq: coseno omega} y \eqref{eq: seno omega}, 
respectivamente.
\begin{itemize}
	\item Si $\omega \not\in \frac{n}{2} \IZ$, entonces 
	$\{ c_{n, \omega}, s_{n, \omega} \}$, donde

	\begin{equation}
	\label{eq5: 19Marzo}
	c_{n, \omega}=\xi_{n, \omega} \tilde{c}_{n, \omega}
	\in \IR^{n}
	\end{equation}
y 

	\begin{equation}
	\label{eq6: 19Marzo}
	s_{n, \omega}= \eta_{n, \omega} \tilde{s}_{n, \omega}
	\in \IR^{n},
	\end{equation}
con 
\begin{equation}
\label{eq7: 19Marzo}
	\xi_{n, \omega}= 
	\sqrt{2} \cdot \left( n + \frac{sen(2 \pi \omega)
	cos(2 \pi \omega \left(\frac{n-1}{n} \right))}{sen \left(2 \pi 
	\frac{\omega}{n} \right)} \right)^{-\frac{1}{2}} 
\end{equation}
y

	\begin{equation}
	\label{eq8: 19Marzo}
	\eta_{n, \omega}= \sqrt{2} \cdot \left( n - \frac{sen(2 \pi \omega)
	cos(2 \pi \omega \left(\frac{n-1}{n} \right))}{sen \left(2 \pi 
	\frac{\omega}{n} \right)} \right)^{-\frac{1}{2}}
	\end{equation}

\noindent
es una base normalizada del subespacio $P_{n, \omega} \leq \IR^{n}$
definido en \eqref{eq6: 23Ap}, y,
\item si $\omega \in \frac{n}{2} \IZ$, entonces 
$\{ c_{n, \omega} \}$, con
\begin{equation}
\label{ec: 4: 23ap}
	c_{n, \omega} := \frac{1}{\sqrt{n}} \tilde{c}_{n, \omega}
\end{equation}
es una base normalizada 
del subespacio $P_{n, \omega} \leq \IR^{n}$
definido en \eqref{eq0: 23Ap}.
\end{itemize}
\end{prop}
\noindent
\textbf{Demostración.}
En efecto, por definición del espacio
$P_{n, \omega}$, $\{ \tilde{c}_{n, \omega}, 
\tilde{s}_{n, \omega} \}$
es una base de este cuando
$\omega \not\in \frac{n}{2}\IZ$, y
$\{ \tilde{c}_{n, \omega}\}$ es base si 
$\omega \in \frac{n}{2}\IZ$,
luego, en ambos casos el conjunto propuesto en efecto
es una base de $P_{n, \omega}$.
En el segundo caso, puesto que
$c_{n, \omega}$ será un vector cuyas entradas serán $1$, o $-1$,
en efecto es un vector unitario. 
En el primer caso, se han calculado las constantes 
$\xi_{n, \omega}$ y $\eta_{n, \omega}$
dadas en 
\eqref{eq7: 19Marzo} y \eqref{eq8: 19Marzo}
para que $c_{n, \omega}$ y $s_{n, \omega}$
tengan normal uno; puesto que los cálculos son muy
similares a los realizados en la demostración de la proposición
\ref{prop: producto punto entre f y g}, los omitimos.
\QEDB
\vspace{0.2cm}


\begin{nota}
\label{nota: notacion cnw, snw}
Observe que las notaciones 
``$c_{n, \omega}$'' y ``$s_{n, \omega}$'' ya las
habíamos empleado antes en la proposición 
\ref{prop: base de fourier version real} cuando se definía
la base ortonormal en función de la cual se calcula la TDF; no hay problema
en usar esta notación aquí también, pues para los valores 
$\omega \in Dom_{TDF, n}$, los vectores
$c_{n, \omega}$ y $s_{n, \omega}$ definidos en 
la proposición \ref{prop: base de fourier version real}
coinciden con los que acabamos de defininir en la proposición 
\ref{prop: aaa}.
\end{nota}

Conviene también establecer una fórmula para
el producto punto entre 
los vectores $c_{n, \omega}$ y $s_{n, \omega}$
definidos en la proposición \ref{prop: aaa}
cuando $\omega \not\in \frac{n}{2} \IZ$.
Hacemos esto a continuación.

\begin{prop}
\label{prop: producto punto entre f y g}
Fijados $n \geq 2$ y $\omega \geq 0$ con 
$\omega \not\in \frac{n}{2}\IZ$, 
el producto punto entre 
los vectores
$c_{n, \omega}$ y $s_{n, \omega}$, definidos 
\eqref{eq5: 19Marzo} y \eqref{eq6: 19Marzo}
respectivamente, es

\begin{equation}
\label{eq9: 19Marzo}
\langle c_{n, \omega} , s_{n, \omega} \rangle =
\frac{\xi_{n, w} \eta_{n, \omega}}{2} \cdot 
\frac{sen(2 \pi \omega)
sen(2 \pi \omega \left( 1- \frac{1}{n} \right))}{sen \left(2 \pi 
\frac{\omega}{n} \right)}
\end{equation}

\end{prop}
\noindent
\textbf{Demostración.}
Aquí usaremos las siguientes tres igualdades:

\begin{equation}
\label{eq10: 19Marzo}
\forall \alpha \in \IR: \hspace{0.2cm}
sen(2 \alpha) = 2 sen(\alpha) cos(\alpha),
\end{equation}



\begin{equation}
\label{eq11: 19Marzo}
\forall z\in \IR: \hspace{0.2cm}
sen(z)= \frac{e^{iz}-e^{-iz}}{2i},
\end{equation}



\begin{equation}
\label{eq12: 19Marzo}
\forall a \in \IR-\{ 1 \}: \hspace{0.2cm}
\suma{m=0}{n-1}{a^{r}}= \frac{1-a^{n}}{1-a}.
\end{equation}

\noindent
Tenemos que

\begin{align*}
\langle c_{n,\omega} , s_{n, \omega} \rangle = &
\xi_{n, \omega} \eta_{n, \omega} \left\langle 
\left( cos \left( 2 \pi \omega \frac{m }{n} \right) \right)_{0 \leq m \leq N-1} ,  
\left( sen \left( 2 \pi \omega \frac{m }{n}\right) \right)_{0 \leq m \leq N-1} \right\rangle \\
= & \xi_{n, \omega} \eta_{n, \omega} \suma{m=0}{n-1}{
cos \left(2 \pi \omega \frac{m}{n}\right) sen\left( 2 \pi \omega \frac{m}{n}\right)} \\
= & \frac{\xi_{n, \omega} \eta_{n, \omega}}{2}
\suma{m=0}{n-1}{
\left( sen\left( 4 \pi \omega \frac{m}{n}\right) \right)} \\
= & \frac{\xi_{n, \omega} \eta_{n, \omega}}{4i} \suma{m=0}{n-1}{
\left( e^{4 \pi \omega i m/n} - 
e^{-4 \pi \omega i m/n} \right) } \\
= & \frac{\xi_{n, \omega} \eta_{n, \omega}}{4i} 
\left(
\frac{1-e^{4 \pi \omega i }}{1-e^{4 \pi \omega i /N}} - 
\frac{1-e^{-4 \pi \omega i }}{1-e^{-4 \pi \omega i /N}} 
\right) \\
= & \frac{\xi_{n, \omega} \eta_{n, \omega}}{4i} 
\left(
\frac{e^{2 \pi \omega i }}{e^{2 \pi \omega i/n }}
\frac{e^{-2 \pi \omega i }-e^{2 \pi \omega i }}{e^{-2 \pi \omega i/n }-e^{2 \pi \omega i /N}} - 
\frac{e^{-2 \pi \omega i }}{e^{-2 \pi \omega i/n }}
\frac{e^{2 \pi \omega i }-e^{-2 \pi \omega i }}{e^{2 \pi \omega i/n }-e^{2 \pi \omega i /N}} 
\right) \\
= & 
\frac{\xi_{n, \omega} \eta_{n, \omega}}{4i} 
\left(
e^{2 \pi \omega i \left( 1-1/n \right)}
\frac{sen(2 \pi \omega)}{sen(2 \pi \omega /n)} - 
e^{-2 \pi \omega i \left( 1-1/n \right)}
\frac{sen(2 \pi \omega)}{sen(2 \pi \omega /n)}
\right) 
\\
= & 
\frac{\xi_{n, \omega} \eta_{n, \omega}}{4i} 
\frac{sen(2 \pi \omega)}{sen(2 \pi \omega /n)}
\left(
e^{2 \pi \omega i \left( 1-1/n \right)} - e^{-2 \pi \omega i \left( 1-1/n \right)}
\right) \\
= &
\frac{\xi_{n, \omega} \eta_{n, \omega}}{4i} 
\frac{sen(2 \pi \omega)}{sen(2 \pi \omega /n)}
\left(
2i \cdot  sen \left( 2 \pi \omega  \left( 1- \frac{1}{n} \right) \right)
\right)\\
= & 
\frac{\xi_{n, \omega} \eta_{n, \omega}}{2} 
\frac{sen(2 \pi \omega)}{sen(2 \pi \omega /n)}
sen \left( 2 \pi \omega  \left( 1- \frac{1}{n} \right) \right). \\
\end{align*}
\QEDB
\vspace{0.2cm}



\begin{prop}
\label{prp: ammm}
Sean $n \geq 2$, $\omega \geq 0$. 
Sea $\sigma_{n}(\cdot,\omega): \IR^{n} \longrightarrow [0,1]$
la función definida en \ref{def: final de sigmas}.
Para todo $x \in \IR^{n}-\{ 0 \}$
se tiene que
\begin{itemize}
	\item Si $\omega \not\in \frac{n}{2} \IZ$, entonces
	\begin{equation}
	\label{eq: pi ommm 1}
	 \Pi_{P_{n, \omega}}(x) = 
\frac{
\langle x, c_{n, \omega} \rangle - \langle c_{n, \omega}, s_{n, \omega} \rangle 
\langle x, s_{n, \omega} \rangle
}
{1-|\langle c_{n, \omega}, s_{n, \omega} \rangle |^{2}  }
c_{n, \omega} +
\frac{
\langle x, s_{n, \omega} \rangle - \langle c_{n, \omega}, s_{n, \omega} \rangle 
\langle x, c_{n, \omega} \rangle
}
{1-|\langle c_{n, \omega}, s_{n, \omega} \rangle |^{2}  }
s_{n, \omega}
	\end{equation}
	y 
	\begin{equation}
	\label{eq: coef sigma caso 1}
	\sigma_{n}(x, \omega) =
	\left(		  
		  \frac{\langle x, c_{n, \omega } \rangle^{2} +  \langle x, s_{n, \omega } \rangle^{2}	
	       -2  \langle x, c_{n, \omega } \rangle \langle x, s_{n, \omega } \rangle \langle c_{n, \omega }, s_{n, \omega } \rangle}{ || x ||^{2} \cdot
	       (1- \langle c_{n, \omega }, s_{n, \omega } \rangle^{2})}	  
\right) ^{1/2},
	\end{equation}
donde $c_{n, \omega}$ y $s_{n, \omega}$ son como en 
\eqref{eq5: 19Marzo} y \eqref{eq6: 19Marzo}, y

\item si $\omega \in \frac{n}{2} \IZ$, entonces 
\begin{equation}
\label{eq: pi ommm 2}
\Pi_{P_{n, \omega}}(x) = \langle x, c_{n, \omega} \rangle c_{n, \omega}
\end{equation}
y 
\begin{equation}
\label{eq: sfklmslsfl}
\sigma_{n}(x, \omega) = \frac{|\langle x, c_{n, \omega} \rangle |}{||x||},
\end{equation}
donde $c_{n, \omega}$ es como en \eqref{ec: 4: 23ap}.
\end{itemize}
\end{prop}



Ya tenemos todo lo necesario para dar una definición alternativa 
del espectro de una señal 
(c.f. definición
\ref{def: espectro DFT} para ver la definición de espectro
basada en la TDF).

\begin{defi}
\label{def: espectro monofrecuenciales inicial}
Sean $n \geq 2$, $x \in \IR^{n}$. 

Si $x \neq 0$, definimos a su \textbf{espectro basado
en espacios monofrecuenciales} como la función 
$\Sigma_{x}: \IR^{+}_{0} \longrightarrow [0,1]$
dada por
\[
\Sigma(x) = \sigma_{n, \omega} (x) \hspace{0.2cm}
\text{ para toda }
\hspace{0.2cm} \omega \in Dom_{\omega},
\]
donde los coeficientes
$\sigma_{n}(x, \omega)$ son como se definieron en 
la proposición \ref{prp: ammm}. \\

Si $x = 0$, definimos su espectro como la 
función
constante cero.
\end{defi}


\begin{comment}
\begin{nota}
Observe lo siguiente; fijadas una dimensión $n$
y una señal $x \in \IR^{n}$, si 
$\omega \in [0, \frac{n}{2}]$, entonces
\begin{equation}
\label{eq0: 1May}
\sigma_{n}(x, \omega) = \sigma_{n}(x, \omega + n/2).
\end{equation}

\noindent
En efecto, para toda $0 \leq m \leq n-1$,
por la regla del coseno de la suma de dos ángulos,
\[
cos\left(2 \pi \left( \omega + \frac{n}{2} \right) \frac{m}{n} \right)
= (-1)^{m} cos \left( 2 \pi \omega \frac{m}{n} \right)
\] 
y, similarmente, 
\[
sen \left(2 \pi \left( \omega + \frac{n}{2} \right) \frac{m}{n} \right)
= (-1)^{m} sen \left( 2 \pi \omega \frac{m}{n} \right),
\] 
luego, se tienen las igualdades \[
c_{n, \omega} =  2_{n, \omega + n/2}
\]
y \[
s_{n, \omega} = s_{n, \omega + n/2},
\] 
por lo tanto, 
\begin{align*}
P_{\omega + \frac{n}{1}} := & span(c_{n, \omega}, s_{n, \omega}) \\
= &  span(c_{n, \omega}, s_{n, \omega})
\end{align*}
\TODO{general al mismo P omega, por eso los sigmas son iguales.}
\end{nota}
\end{comment}

\subsection{Desfase de la proyección de una señal a espacios monofrecuenciales}


Fijada una dimensión $n$ y 
una frecuencia $\omega \geq 0$,
dado cualquier 
$x \in \IR^{n}-\{ 0 \}$
ya tenemos una fórmula para calcular la
proyección $\Pi_{P_{n, \omega}}(x)$ 
(c.f. proposición \ref{prp: ammm}).
Sin embargo, 
como
$\Pi_{P_{n, \omega}}(x) \in P_{n, \omega}$, 
según el teorema 
\ref{prop: Pw consta de las señales de frecuencia omega}, 
es posible expresar a la señal $\Pi_{P_{n, \omega}}(x)$
como el resultado de muestrear uniformemente con $n$ mediciones
a un sinusoide de frecuencia pura $\omega$. Lo que queremos hacer
en esta sección es dar explícitamente a los dos elementos que
faltan para determinar univocamente este sinusoide continuo,
a saber, el parámetro de amplitud $A \in \IR$ y el 
desfase normalizado $\phi \in [0,1[$. Buscamos entonces $A$
y $\phi$ tales que
\[
\Pi_{P_{n, \omega}}(x) = A (cos(2 \pi \omega t -  2 \pi \phi ))_{t \in I_{n}}.
\]
 
\marginnote{Buscamos pues la amplitud y el desfase de la señal
de frecuencia $\omega$ más cercana a $x$.}

Puesto que
$\Pi_{P_{n, \omega}}(x)$ (donde $P_{n, \omega}$ es como se definió en 
\eqref{eq: espacio Pnw}) 
es la señal de frecuencia $\omega$ que está a menor
distancia euclidea de $x$, podremos interpretar este
desfase $\phi$ como el desfase que mejor se ajusta a $x$
(c.f. figura \ref{fig: desfasee ejemplo grafico}). \\

Primero abordemos el caso en el que
$\omega \not\in \frac{n}{2} \IZ$. 

Como los vectores $c_{n, \omega}$ y $s_{n, \omega}$ 
definidos en \eqref{eq5: 19Marzo} y \eqref{eq6: 19Marzo}
son unitarios y linealmente independientes (c.f. proposición
\ref{prop: aaa}),
podemos usar la ecuación \eqref{eq0: 24ap}
para escribir a la proyección de $x$ en $P_{\omega}$ como sigue


\begin{equation}
\label{eq3: 20Marzo}
\Pi_{P_{\omega}}(x)= c (cos (2 \pi \omega t))_{t \in I_{n}} + d 
(sin (2 \pi \omega t))_{t \in I_{n}},
\end{equation}
donde

\begin{equation}
\label{eq4: 20Marzo}
c= \frac{
\langle x, c_{n, \omega} \rangle - \langle c_{n, \omega}, s_{n, \omega} \rangle
\langle x, s_{n, \omega} \rangle
}{1-\langle c_{n, \omega}, s_{n, \omega} \rangle^{2}} \xi_{n, \omega}
\end{equation}
y
\begin{equation}
\label{eq5: 20Marzo}
d= \frac{
\langle x, s_{n, \omega} \rangle - \langle c_{n, \omega}, s_{n, \omega} \rangle
\langle x, c_{n, \omega} \rangle
}{1-\langle c_{n, \omega}, s_{n, \omega} \rangle^{2}} \eta_{n, \omega}.
\end{equation}

\noindent 
Nos conviene más reescribir a \eqref{eq3: 20Marzo} como
\begin{equation}
\label{eq6: 20Marzo}
\Pi_{P_{n, \omega}}(x)= 
\sqrt{c^{2}+d^{2}}
\left[
C (cos (2 \pi \omega t))_{t \in I_{n}} +
D (sen (2 \pi \omega t))_{t \in I_{n}} 
\right],
\end{equation}

\noindent 
donde

\begin{equation}
\label{eq3: 28Marz23}
C:= \frac{c}{\sqrt{c^{2}+d^{2}}} \hspace{0.2cm} \text{y}
\hspace{0.2cm} D:= \frac{d}{\sqrt{c^{2}+d^{2}}},
\end{equation}
\noindent 
pues, como $C^{2} + D^{2}=1$, existe un único
$\phi \in [0,1[$ tal que
\begin{equation}
\label{eq7: 20Marzo}
C= cos(2 \pi \phi), \hspace{0.2cm} 
D= sin(2 \pi \phi).
\end{equation}

\noindent 
Sustituyendo \eqref{eq7: 20Marzo} en \eqref{eq6: 20Marzo},
llegamos a que

\begin{align*}
\Pi_{P_{n, \omega}}(x) = & 
\sqrt{c^{2}+d^{2}} \left[
cos(2 \pi \phi) \cdot (cos (2 \pi \omega t))_{t \in I_{n}} +
sin(2 \pi \phi) \cdot (sin (2 \pi \omega t))_{t \in I_{n}} 
\right] \\
= & 
\sqrt{c^{2}+d^{2}} 
(cos(2 \pi \phi) \cdot cos (2 \pi \omega t) +
sin(2 \pi \phi) \cdot sin (2 \pi \omega t) )_{t \in I_{n}} \\
= & 
\sqrt{c^{2}+d^{2}} 
(cos(2 \pi \omega t - 2 \pi \phi))_{t \in I_{n}}.
\end{align*}

\noindent
Además, de \eqref{eq7: 20Marzo} y \eqref{eq3: 28Marz23}
se deduce que
\begin{marginfigure}
\includegraphics[scale= 0.8]{encontrado_desfase} 
\end{marginfigure}
\begin{equation}
\label{eq: desfase phi 1}
\phi =
\begin{cases}
\frac{tan^{-1}(d/c) }{2 \pi}  \hspace{0.4cm}    \text{   si }   d, c > 0,  \\
\frac{tan^{-1}(d/c) + \pi }{2 \pi} \hspace{0.2cm}  \text{si }  d, c < 0
\text{ o } d<0, c>0, \\
\frac{tan^{-1}(d/c) + 2\pi }{2 \pi} \hspace{0.2cm}  \text{si }  d>0,  c < 0. 
\end{cases}
\end{equation}


Hemos probado el siguiente
\begin{teo}
\label{teo: amelie1}
Sean $n \geq 2$ y $\omega > 0$ con $\omega \not\in \frac{n}{2}\IZ$.
Si $P_{n, \omega}$ es el subespacio de $\IR^{n}$ definido como 
en \eqref{eq6: 23Ap}, entonces, para todo 
$x \in \IR^{n}$ no cero, se tiene que
\begin{equation}
\label{ec: desfase explicito 1}
\Pi_{P_{n, \omega}} (x) = \sqrt{c^{2}+d^{2}} \cdot (
cos (2 \pi \omega t - 2 \pi \phi)
)_{t \in I_{n}} \in \IR^{n},
\end{equation}

\noindent
donde 
$c$ y $d$ son como en \eqref{eq4: 20Marzo} y 
\eqref{eq5: 20Marzo}, resp., y $\phi$ está 
dado por \eqref{eq: desfase phi 1}.
\end{teo}
Observe que tenemos una fórmula para obtener a
la frecuencia y la amplitud de $\Pi_{P_{n, \omega}}(x)$
usando sólamente los datos
\[
\langle x, c_{n, \omega} \rangle, \hspace{0.2cm}
\langle x, s_{n, \omega} \rangle \hspace{0.1cm} \text{y} \hspace{0.1cm}
\langle c_{n, \omega}, s_{n, \omega} \rangle.
\]

El resultado análogo para cuando $\omega \in \frac{n}{2}\IZ$
es más fácil de establecer, pues en este caso
el espacio monofrecuencia $P_{n, \omega}$ es una recta.

\begin{teo}
\label{teo: amelie2}
Sean $n \geq 2$ y $\omega > 0$ con $\omega \in \frac{n}{2}\IZ$.
Si $P_{n, \omega}$ es el subespacio de $\IR^{n}$ definido como 
en \eqref{eq0: 23Ap}, entonces, para todo 
$x \in \IR^{n}$ no cero, se tiene que
\begin{equation}
\label{ec: desfase explicito 2}
\Pi_{P_{n, \omega}} (x) = 
\frac{1}{\sqrt{n}} \langle x, c_{n, \omega} \rangle
\cdot (cos (2 \pi \omega t))_{t \in I_{n}} \in \IR^{n}.
\end{equation}
\end{teo}
\noindent
\textbf{Demostración.}
En efecto, según \eqref{eq: pi ommm 2} y 
\eqref{ec: 4: 23ap},
se tiene que
\begin{align*}
\Pi_{P_{n, \omega}} (x) = & 
\langle x, c_{n, \omega} \rangle c_{n, \omega} \\
= & \langle x, c_{n, \omega} \rangle \frac{1}{\sqrt{n}} \tilde{c}_{n, \omega} \\
= & \frac{1}{\sqrt{n}} \langle x, c_{n, \omega} \rangle
\cdot (cos (2 \pi \omega t))_{t \in I_{n}} \in \IR^{n}.
\end{align*}

\QEDB
\vspace{0.2cm}

\section{Simetría, periodicidad y continuidad del espectro $\Sigma_{x}$}
\label{sec: simetria, periodicidad, continuidad}
Fijada una dimensión $n \geq 2$,
si $x \in \IR^{n}$ es cualquier señal y 
$\Sigma_{x}$
es su espectro como se definió en
\ref{def: espectro monofrecuenciales},
en esta sección vamos a demostrar 
algunos resultados sobre la periodicidad 
de esta función y su simetría 
respecto a puntos de la forma
$\frac{n}{2} \IZ$.
Esto será de utilidad pues nos permitirá
acotar considerablemente el dominio de frecuencias
de $\Sigma_{x}$.



\begin{prop}
\label{prop: periodicidad espectro}
\textbf{(Periodicidad del espectro)}
Sean $n \geq 2$, $x \in \IR^{n}$.
Sea $\Sigma_{x}$ el espectro de $x$ como se definió en 
\eqref{def: espectro monofrecuenciales inicial}.
El espectro $\Sigma_{x}$ es $n-$periódico, es decir, 
para cualquier frecuencia
$0 \leq \omega \leq n$
y toda $K \in \IZ$, se tiene que 
\[
\sigma_{n}(x, \omega) = \sigma_{n}(x, \omega + Kn).
\]
\end{prop}
\noindent
\textbf{Demostración.}
Sólo observe que 
\begin{align*}
\tilde{c}_{n, \omega + Kn} = & \left( cos \left( 2 \pi
\left( \omega + Kn \right) \frac{m}{n} \right) \right)_{m=0}^{n-1} \\
= & \left( cos \left( 
2 \pi \omega \frac{m}{n} + 2 \pi K m
\right) \right)_{m=0}^{n-1} \\
= & \left( cos \left( 
2 \pi \omega \frac{m}{n}
\right) \right)_{m=0}^{n-1} = \tilde{c}_{n, \omega}
\end{align*}
y, similarmente, que 
\[
\tilde{s}_{n, \omega + Kn} = \tilde{s}_{n, \omega},
\]
luego, por definición de los espacios monofrecuenciales
(c.f. ecuación \ref{eq: espacio Pnw}),
\begin{align*}
P_{n, \omega + Kn} =
& span(\tilde{c}_{n, \omega + Kn}, \tilde{s}_{n, \omega + Kn}) \\
= & span(\tilde{c}_{n, \omega }, \tilde{s}_{n, \omega }) = P_{n, \omega};
\end{align*}
de esto se concluye, usando la definición
\ref{def: final de sigmas},
que 
\[
\sigma_{n}(x, \omega) = 
cos (\measuredangle(x, P_{n, \omega}))
= cos (\measuredangle(x, P_{n, \omega + Kn})) = 
\sigma_{n}(x, \omega + Kn).
\]
\QEDB
\vspace{0.2cm}

\begin{figure}[H]
	\sidecaption{
	Según la periodicidad establecida en la proposición 
	\ref{prop: periodicidad espectro}, basta calcular los
	coeficientes espectrales
	$\sigma_{n}(x, \omega)$ para frecuencias
	$0 \leq \omega \leq n$.
	\label{fig: periodicidad espectro}
	}
	\centering
	\includegraphics[scale = 0.9]{periodicidad_espectro} 
\end{figure}	


\begin{prop}
\textbf{(Simetría del espectro)}
Sean
$n \geq 2$,
$x \in \IR^{n}$. Para toda $0 \leq1 \omega \leq \frac{n}{2}$,
\[
\sigma_{n}(x, \omega) = 
\sigma_{n}(x, n-\omega ). 
\]
\end{prop}
\noindent
\textbf{Demostración.}
En efecto, 
\begin{align*}
\tilde{c}_{n, \omega + n} = & \left( cos \left( 2 \pi
\left( n- \omega \right) \frac{m}{n} \right) \right)_{m=0}^{n-1} \\
= & \left( cos \left( 
2 \pi n \frac{m}{n} - 2 \pi \omega
\frac{m}{n}
\right) \right)_{m=0}^{n-1} \\
= & \left( cos \left( 
2 \pi m - 2 \pi \omega \frac{m}{n} 
\right) \right)_{m=0}^{n-1} \\
= & \left( cos \left( 2 \pi \omega \frac{m}{n} \right) \right)_{m=0}^{n-1}
= \tilde{c}_{n, \omega}
\end{align*}
y, similarmente,
\[
\tilde{s}_{n, \omega + Kn} = -\tilde{s}_{n, \omega};
\]
de esto, como en la demostración de la proposición
\ref{prop: periodicidad espectro}, se concluye la igualdad
entre los espacios $P_{n, \omega}$ y $P_{n, n-\omega}$, y de esto
la igualdad deseada.
\QEDB
\vspace{0.2cm}

\begin{figure}[H]
	\sidecaption{
	Podemos así afinar la afirmación hecha en la figura 
	\ref{fig: periodicidad espectro} y concluir que basta
	calcular los coeficientes
	$\sigma_{n}(x, \omega)$ para $0 \leq \omega \leq \frac{n}{2}$,
	pues los demás pueden deducirse a partir de reflexiones y traslaciones.
	\label{fig: simetria espectro}
	}
	\centering
	\includegraphics[scale = 1.4]{simetria_espectro} 
\end{figure}	
 

\begin{nota}
\label{nota: muestreo dom frecuencia}
Según estas propiedades de periodicidad y simetría,
podemos limitarnos a evaluar el espectro
$\Sigma_{x}$ de una señal sólo en frecuencias
contenidas en el intervalo $[0, n/2]$, pues los valores
del espectro para otros valores pueden deducirse por periodicidad
y simetría. \\

Ahora bien, para poder escribir programas
para calcular un tal espectro $\Sigma_{x}$
se debe de usar
un conjunto discreto de puntos.
Para los espectros que calcularemos de ahora en 
adelante, adoptamos la convención de 
usar usar como dominio 
del espectro
$\Sigma_{x}$ de una señal $x \in \IR^{n}$
al conjunto
\TODO{revisa!}
\begin{equation}
\label{eq: malla frecuencias}
\left\{ \frac{a}{100} : \hspace{0.2cm}
0 \leq a \leq \frac{100n}{2} \right\},
\end{equation}

es decir, se toman $100$ muestras por
cada unidad del intervalo 
$\left[ 0, \frac{n}{2}\right]$
\end{nota}


Para terminar, hagamos algunos comentarios
sobre la continuidad del espectro $\Sigma_{x}$
de una señal $x \in \IR^{n}$.
Por la periodicidad y simetría del espectro, basta
analizar la continuidad de $\Sigma_{x}$ sólo en el
intervalo cerrado $[0, n/2]$. \\

Será de utilidad introducir la siguiente notación.
\begin{defi}
\label{def: momentos de x}
Sean $n \geq 2$, $x= (x_{m})_{m=0}^{n-1} \in \IR^{n}$, $k \geq 0$.
Se define el número real $M_{k}(x)$ como sigue;
	\begin{equation}
	\label{eq: momento k esimo de x}
	M_{k}(x) := \suma{m=0}{n-1}{m^{k}x_{m}}.
	\end{equation}
\end{defi}

\TODO{cambiar la notación para los momentos.}
\begin{prop}
\label{prop: limite del espectro por cero}
Sean $n \geq 2$, $x \in \IR^{n}$.
Sea $\Sigma_{x}: [0, n/2] \rightarrow [0,1]$ el espectro de $x$ como se definió
en \ref{def: espectro monofrecuenciales inicial}.
Sea 
\begin{align*}
\alpha(x)= \begin{cases}
1 & \textit{si } M_{1}(x) \geq \frac{2\pi^{2}}{3n^{2}}M_{3}(x), \\
-1 & \textit{ en otro caso}.
\end{cases}
\end{align*}

El espectro $\Sigma_{x}$ es continuo en $]0, n/2[$. Además,
\begin{equation}
\label{eq: limite del espectro a cero}
\limite{\omega \rightarrow 0^{+}}{\Sigma_{x}(\omega)}
=
\left(
\frac{
\frac{M_{0}(x)^{2}}{n} + \frac{6M_{1}(x)^{2}}{n(2n-1)(n-1)}
- (-1)^{\alpha(x)}M_{0}(x) \frac{6 |M_{1}(x)|}{n(2n-1	)}
}{
||x||^{2} \left(
1- 1.5 \frac{n-1}{2n-1}
\right)
}
\right)^{1/2}
\end{equation}
\end{prop}
\noindent
\textbf{Demostración.}
Por definición del espectro, 
si $\omega \geq 0$, entonces
$\Sigma_{x}(\omega) = \sigma_{n} (x, \omega)$,
donde los coeficientes
$\sigma_{n} (x, \omega)$ son como se definieron en
la proposición \ref{prp: ammm}.


Observe que la fórmula
\eqref{eq: coef sigma caso 1}, que sirve
para calcular $\sigma_{n} (x, \omega)$
cuando $\omega \in ]0, n/2[$, es una combinación
de sumas y productos de senos y cosenos
evaluados en funciones de la frecuencia $\omega$, luego, 
es una función continua, por lo tanto $\Sigma_{x}$
es continua en el interior del intervalo 
$[0, n/2]$.

Determinemos ahora 
el límite
\begin{equation}
\label{eq0: 22May}
\limite{\omega \rightarrow 0^{+}}{
\Sigma_{x}(\omega)}
= \limite{\omega \rightarrow 0^{+}}
{
\left(		  
		  \frac{\langle x, c_{n, \omega } \rangle^{2} +  \langle x, s_{n, \omega } \rangle^{2}	
	       -2  \langle x, c_{n, \omega } \rangle \langle x, s_{n, \omega } \rangle \langle c_{n, \omega }, s_{n, \omega } \rangle}{ || x ||^{2} \cdot
	       (1- \langle c_{n, \omega }, s_{n, \omega } \rangle^{2})}	  
\right) ^{1/2}
}.
\end{equation}

Para la tarea, usaremos las series de Taylor
de las funciones seno y coseno alrededor del cero
con términos de hasta la potencia $5$ para aproximar
a los sinusoides que aparecen en la expresión 
de la derecha de \eqref{eq0: 22May}, es decir, usaremos
las siguientes aproximaciones, válidas en las cercanías
del cero;
\[
sen(\omega) \sim \omega - \frac{\omega^{3}}{!}
+ o(\omega^{5}),
\hspace{0.2cm} \omega \sim 0.
\]
\[
cos(\omega) \sim 1 - \frac{\omega^{2}}{2!}
+ \frac{\omega^{4}}{4!} + o(\omega^{5}),
\hspace{0.2cm} \omega \sim 0.
\]
Obtenemos entonces los siguientes equivalentes asintóticos;

\begin{align*}
\xi_{n, \omega} \sim &
\sqrt{2} 
\left(
4 \pi
\frac{                                                                                                                                          
\omega - \frac{2\pi^{2}}{3n^{2}}(2n^2-3n+2)\omega^{3} + o(\omega^{5})
}{
\frac{2\pi}{n} \omega -
\frac{4 \pi^{3}}{3 n^{3}} \omega^{3} + o(\omega^{5})
}
\right)^{-1/2} \\
\sim &
\sqrt{2} 
\left(
2n
\frac{                                                                                                                                          
\omega - \frac{2\pi^{2}}{3n^{2}}(2n^2-3n+2)\omega^{3} + o(\omega^{5})
}{
\omega -
\frac{2 \pi^{2}}{3 n^{2}} \omega^{3} + o(\omega^{5})
}
\right)^{-1/2} 
\rightarrow \frac{1}{\sqrt{n}},
\end{align*}

\begin{align*}
\eta_{n, \omega} \sim &
\sqrt{2} 
\left(
\frac{
8 \pi^{3} (2n-1)(n-1)\omega^{3} + o(\omega^{5})
}{
6 \pi n \omega -
\frac{4 \pi^{3}}{n} \omega^{3} + o(\omega^{5})
}
\right)^{-1/2} \\
\sim & 
\sqrt{2} 
\left(
4\pi^{2}
\frac{
(2n-1)(n-1)\omega^{3} + o(\omega^{5})
}{
3 n \omega -
\frac{2 \pi^{2}}{n} \omega^{3} + o(\omega^{5})
}
\right)^{-1/2} 
 \rightarrow \infty.
\end{align*}
A partir de estas expresiones se calcula que
\[
\langle x,
c_{n, \omega}
\rangle = 
\xi_{n, \omega} 
\left(
X_{0} - \frac{2 \pi^{2}}{n^{2}}X_{2} \omega^{2} 
+ \frac{2 \pi^{4}}{3n^{4}} X_{4} \omega^{4} + o(\omega^{5})
\right) \rightarrow \frac{X_{0}}{\sqrt{n}} ,
\]
\TODO{Da más detalles para los otros dos.}
\begin{align*}
\langle x,
s_{n, \omega}
\rangle = &
\eta_{n, \omega} 
\left(
\frac{2 \pi}{n} X_{1} \omega - \frac{4 \pi^{3}}{3n^{3}}X_{3} \omega^{3} 
 + o(\omega^{5})
\right)\\
= &
\begin{cases}
\sqrt{2} \left(
\frac{
\frac{24 \pi^{3}}{n}X_{1}^{2}\omega^{3} + o(\omega^{5}) }{
8 \pi^{3} (2n-1)(n-1)\omega^{3} + o(\omega^{5})
}
\right)^{1/2}
\rightarrow \left(
\frac{6 X_{1}^{2}}{(2n-1)(n-1)n}
\right)^{1/2} & \textit{si } \alpha_{n, \omega}(x) \geq 0 \\
-\sqrt{2} \left(
\frac{
\frac{24 \pi^{3}}{n}X_{1}^{2}\omega^{3} + o(\omega^{5}) }{
8 \pi^{3} (2n-1)(n-1)\omega^{3} + o(\omega^{5})
}
\right)^{1/2}
\rightarrow -\left(
\frac{6 X_{1}^{2}}{(2n-1)(n-1)n}
\right)^{1/2} & \textit{si } \alpha_{n, \omega}(x) < 0,\\
\end{cases}
\end{align*}
y
\begin{align*}
\langle
c_{n, \omega}, s_{n, \omega}
\rangle \sim &
\frac{2\pi}{n}
\xi_{n, \omega} \eta_{n, \omega}
(n-1)  
\left(
\frac{n}{2} \omega - \frac{2\pi^{2}}{3} (n-1) \omega^{3} + o(\omega^{5})
\right) \\
= & 
\frac{2 \sqrt{2}}{n \sqrt{n}} \pi (n-1)
\left(
\frac{
\frac{3}{2} \pi n^{3}\omega^{3} + o(\omega^{5})
}{
8 \pi^{3}(2n-1)(n-1)\omega^{3} + o(\omega^{5})
}
\right)^{1/2}
\rightarrow \frac{
\sqrt{6(n-1)}
}{2 \sqrt{2n-1}}.
\end{align*} 
De estos límites se deduce que 

\begin{equation}
\label{ec: limite x, cnw cuadr}
\limite{\omega \rightarrow 0^{+}}{\langle
x, c_{n, \omega}
\rangle^{2} }
= \frac{M_{0}(x)^{2}}{\sqrt{n}},
\end{equation}
\begin{equation}
\label{ec: limite x, snw cuad}
\limite{\omega \rightarrow 0^{+}}{\langle
x, s_{n, \omega}
\rangle^{2} }
= \frac{6M_{1}(x)^{2}}{(2n-1)(n-1)n},
\end{equation}

\begin{equation}
\label{ec: limite -2abc}
\limite{\omega \rightarrow 0^{+}}{
-2 \langle x, c_{n, \omega} \rangle
\langle x, s_{n, \omega} \rangle
\langle c_{n, \omega}, s_{n, \omega} \rangle
= 
-(-1)^{\alpha(x)} \frac{6M_{0}(x)|M_{1}(x)|}{n(2n-1)},
}
\end{equation}
\begin{equation}
\label{ec: limite cnw, snw cuad}
\limite{\omega \rightarrow 0^{+}}{\langle
c_{n, \omega}, s_{n, \omega}
\rangle^{2} }
= \frac{3(n-1)}{2(2n-1)}.
\end{equation}
Observe que $1-\frac{3(n-1)}{2(2n-1)}$
nunca es cero, o sea, que sustituyendo la 
expresión \eqref{ec: limite cnw, snw cuad}
en 
\begin{equation}
\label{eq1: 22May}
\left(		  
		  \frac{\langle x, c_{n, \omega } \rangle^{2} +  \langle x, s_{n, \omega } \rangle^{2}	
	       -2  \langle x, c_{n, \omega } \rangle \langle x, s_{n, \omega } \rangle \langle c_{n, \omega }, s_{n, \omega } \rangle}{ || x ||^{2} \cdot
	       (1- \langle c_{n, \omega }, s_{n, \omega } \rangle^{2})}	  
\right) ^{1/2}
\end{equation}
no se tiene un denominador igual a cero, por lo que podemos
sustituir los límites
\eqref{ec: limite x, cnw cuadr},
\eqref{ec: limite x, snw cuad},
\eqref{ec: limite -2abc} y
\eqref{ec: limite cnw, snw cuad}
en \eqref{eq1: 22May}
para concluir que el límite buscado
\eqref{eq0: 22May} existe y es igual
a la expresión propuesta 
\ref{eq: limite del espectro a cero}.

\QEDB
\vspace{0.2cm}


\TODO{Antiguo:}
No pudimos determinar la
continuidad de $\Sigma_{x}$ en los puntos
extremos $0$ y $n/2$, ni encontramos un criterio que deba
cumplir la señal que implique la continuidad en uno
de estos puntos extremos.
No escribimos en detalle los intentos para
determinar los límites
\[
\limite{\omega \rightarrow 0^{+}}{\sigma_{n}(x, \omega)}
\hspace{0.2cm} \textit{y} \hspace{0.2cm}
\limite{\omega \rightarrow \frac{n}{2}^{-}}{\sigma_{n}(x, \omega)},
\]
pero comentamos que el problema que encontramos fue
intentar determinar el límite
(cuando $\omega \rightarrow 0^{+}$ y cuando $\omega \rightarrow \frac{n}{2}^{-}$)
de la expresión
\[
\sqrt{2} \left(
n - \frac{sen(2 \pi \omega) cos (2 \pi \omega \frac{n-1}{n})}{sen(2 \pi \omega/n)}
\right)^{-1/2}
\cdot
\suma{m=0}{n-1}{x_{m} sen(2 \pi \omega m/n)};
\]
puesto que se tiene una indeterminación de tipo
$\infty \cdot 0$, se aplicó la Regla de L'Hôpital
(c.f. \cite{hopital}) 
para intentar determinar el límite, pero esto sólo
nos llevó a una expresión más complicada que presentaba una indeterminación
de tipo $\frac{0}{0}$. No esperabamos que la tarea
de determinar la continuidad por los extremos fuese sencilla,
pues después de graficar algunos espectros notamos que algunos
de ellos eran continuos por ambos extremos, mientras que otros
presentaban discountinuidades (que, por estár acotados los valores
del espectro por $0$ y $1$, necesariamente son de salto)
en uno o ambos extremos, y no notamos características de 
la señal $x$ a partir de la cual se calcula el espectro
que indicaran cuándo se tenía la discontinuidad por uno 
de los puntos extremos.





\section{Relación entre los espectros basados en la TDF y en espacios monofrecuenciales}

Después de todo lo expuesto en las secciones anteriores, tenemos
ya dos alternativas para realizar un análisis
espectral de una señal $x \in \IR^{n}$.

Sean $n \geq 2$, $M := \lceil \frac{n}{2} \rceil$, $x \in \IR^{n}$.
\begin{itemize}
	\item \textbf{(Espectro-0: basado en la TDF)} 
	Usando la transformada discreta de Fourier
	(c.f. sección \ref{sec: TDF}),
	el espectro de $x$ es la función
	\[
	\Tau_{x}: Dom_{TDF, n} \longrightarrow \IR^{+}_{0}
	\]	
	definida en \ref{def: espectro DFT}.
	
	La gráfica es entonces la de las frecuencias
	enteras $\omega$ dadas (dependiendo de la 
	paridad de $n$) por las
	tablas 6.1 y 6.2
	versus los coeficientes
	$\tau_{n}(x, \omega)$ definidos en
	\ref{def: taus}.
	
	Puesto que el realizar un análisis de 
	$x$ via la TDF significa encontrar una
	expresión de $x$ como una suma
	ponderada de muestreos de senos y cosenos,
	de frecuencias enteras las indicadas en las tablas 6.1 o 6.2,
	se tiene que  
	\begin{itemize}
		\item Para toda frecuencia $\omega$ considerada
		por la TDF,
		\[
		0 \leq \tau_{n}(x, \omega) \leq 1,
		\]
		y que
		\item entre más se acerque
		$\tau_{n, \omega}(x)$
		a $1$, mayor es la
		importancia de la frecuencia $\omega$ para
		sintetizar s $x$; recíprocamente, si 
		$\tau_{n}(x, \omega)$ es cercana a cero, entonces
		la frecuencia $\omega$ no es muy relevante para 
		sintetizar a la señal $x$.
	\end{itemize}
	\begin{defi}
	\label{def: FM0}
	Llamaremos \textbf{frecuencia principal-0}
	(y denotaremos por $FP0(x)$) 
	a una 
	frecuencia $\omega \in Dom_{TDF, n}$
	tal que, para cualquier otra $\omega' \in Dom_{TDF, n}$ 
	se tenga que 
	\[
	\tau_{n}(x, \omega') = \Tau_{x}(\omega^{'}) \leq
	\Tau_{x}(\omega) =  
	 \tau_{n}(x, \omega).
	\]
	\end{defi}
	Observe que tal frecuencia principal existe por ser 
	el máximo de un conjunto finito de números, pero que no 
	estamos asegurando que sea única. 
	
	\item \textbf{(Espectro-1: basado en espacios monofrecuenciales)} 
	Usando
	las ideas propuestas en 
	la sección
	\ref{sec: metodologia para realizar un analisis espectral que considere frecuencias arbitrarias}, 
	es decir, si se usan cosenos de ángulos a
	espacios monofrecuenciales $P_{n, \omega}$
	(c.f. \ref{eq: espacio Pnw}), el espectro
	de una señal $x$ se definió en
	\ref{def: espectro monofrecuenciales inicial}
	como la función 
	\[
	\Sigma_{x} : \left[0, \frac{n}{2} \right] \longrightarrow [0,1].
	\]
	
	La gráfica de este espectro es la de 
	las frecuencias $\omega \in [0, \frac{n}{2}]$ versus	
	los coeficientes
	$\sigma_{n}(x, \omega)$ definidos en 
	\ref{prp: ammm}. Observe que
	\begin{itemize}
		\item para cualquier frecuencia $\omega \geq 0$, se tiene que
		\[
		0 \leq \sigma_{n}(x, \omega) \leq 1
		\]
		\item 
	y, el que
	$\sigma_{n}(x, \omega)$ sea cercano a $1$ significa que un
	muestreo uniforme de un sinusoide de frecuencia $\omega$
	modela bien el comportamiento de $x$,
	mientras que un $\sigma_{n}(x, \omega)$ cercano
	a cero significa que 
	$x$ es casi perpendicular a toda señal de frecuencia $\omega$
	(c.f. nota \ref{nota: significado de los sigma en AE}).
	\end{itemize}
\end{itemize}

Nos gustaría, así como hicimos en la definición
\ref{def: FM0}, definir la frecuencia principal 
de una señal $x$ basada en el espectro
$\Sigma_{x}$ de esta. No es tan sencillo hacer esto, pues,
como comentamos en la sección
\ref{sec: simetria, periodicidad, continuidad}, no parece
que para toda señal $x \in \IR^{n}$ el espectro
$\Sigma_{x}$ sea continuo en los puntos extremos
$0$ y $n/2$, por lo que intentar definir una
frecuencia principal como
\[
\omega \in [0, n/2] \textit{ tal que }
a_{x} = \Sigma_{x}(\omega), \textit{ donde }
a_{x} = sup \{ \Sigma_{x}(w): \hspace{0.2cm} \omega
\in [0, n/2] \}
\]
no es adecuado, pues no está excluida la 
posibilidad de que exista un espectro $\Sigma_{x}$
para el que no exista una $\omega \in [0, n/2]$ 
tal que $\Sigma_{x}(\omega) = a_{x}$, o sea, tal que 
$a_{x}$ que no sea máximo
del conjunto 
$\{ \Sigma_{x}(w): \hspace{0.2cm} \omega
\in [0, n/2] \}$. \\

\begin{figure}[H]
	\sidecaption{
	Se muestra un dibujo hipotético de un espectro
	$\Sigma_{x}$ para el que $a_{x} = 1$ pero no exista
	ninguna frecuencia $\omega \geq 0$ (una candidata
	a frecuencia principal) tal que $\Sigma_{x}(\omega) = a_{x}$.
 	\label{fig: ej FP1}
	}
	\centering
	\includegraphics[scale = 1]{ejemplo_FP1.png} 
\end{figure}	
Sorteamos este problema si restringimos las frecuencias
$\omega$ a considerar a un subconjunto finito del
rango inicial de frecuencias
$[0, n/2]$.

\begin{defi}
	\label{def: FM1}
	Sean $n \geq 2$, $x \in \IR^{n}$. Sea
	$\cali{P}$ un subconjunto finito
	de $[0, n/2]$.
	Llamaremos la \textbf{frecuencia principal-$1$}
	(y denotaremos
	por $FP1(x)$) 
	de $x$ respecto al conjunto $\cali{P}$	
	a una frecuencia $\omega \in \cali{P}$ 
	tal que, para cualquier otra $\omega'$ del $\cali{P}$, se tenga que
	\[
	\sigma_{n, \omega'}(x) = \Sigma_{x}(\omega') 
	\leq \Sigma_{x}(\omega) = \sigma_{n, \omega}(x).
	\]
\end{defi}
En la práctica, el que se tenga que fijar un conjunto
finito $\cali{P} \subseteq [0, n/2]$ de frecuencias
para calcular la FP1 no es un problema, pues ya teníamos que hacer
esto para calcular, de forma computacional, el espectro
$\Sigma_{x}$. Para respetar la convención
puesta en la nota 
\ref{nota: muestreo dom frecuencia}, vamos
a tomar a $\cali{P}$ como en 
\eqref{eq: malla frecuencias}.
	
\begin{nota}
Observe que no estamos asegurando
que las frecuencias principales
$FP0(x)$ y $FP1(x)$ como se definieron en 
\ref{def: FM0} y \ref{def: FM1}
sean únicas.
Si hay dos o más frecuencias $\omega'$ que satisfagan
la definición de frecuencia principal-0 
(resp. frecuencia principal-1), tomamos
como valor de $FP0(x)$ 
(resp. $FP1(x)$)
a la mayor de tales $\omega'$.
\end{nota}


Demostremos ahora que, de hecho, el espectro-0
de hecho es la restricción del espectro-1
al conjunto de frecuencias 
$Dom_{TDF, n}$ considerado por la transformada discreta
de Fourier.
\begin{prop}
\label{prop: coinciden espectr}
Sean $n \geq 2$, $x \in \IR^{n}$.
Sea $Dom_{TDF, n}$ el dominio del espectro-0 de $x$
como se definió en \ref{def. Dom tdf}. Se tiene que
\begin{equation}
\label{eq4: 4May}
\forall \omega \in Dom_{TDF, n}:
\hspace{0.2cm} \tau_{n}(x, \omega) = \sigma_{n}(x, \omega).
\end{equation}
\end{prop}

\noindent
\textbf{Demostración.}
Recuerde que los coeficientes
$\sigma_{n}(x, \omega)$
se definieron como
\[
\sigma_{n}(x, \omega) = \frac{|| \Pi_{P_{n, \omega}}(x) ||}{|| x ||}.
\]
Teniendo una base ortonormal del espacio 
$P_{n, \omega}$ puede calcularse la proyección involucrada en la fórmula
para $\sigma_{n}(x, \omega)$.
Recuerde que, por definición del espacio $P_{n, \omega}$,
\begin{itemize}
	\item los vectores $c_{n, \omega}$ y $s_{n, \omega}$ conforman
	una base de $P_{n, \omega}$ cuando $1 \leq \omega \leq M-1$ 
	(pues, para estos valores de omega se tiene siempre
	que $\omega \not\in \frac{n}{2} \IZ$) y
	\item $c_{n, \omega}$ conforma una base 
	de $P_{n, \omega}$ cuando $\omega = 0$ y,
	en el caso en el que $n$ es par, también para cuando $\omega = M$
	(pues sólo para estos valores de omega se tiene 
	que $\omega \in \frac{n}{2} \IZ$).
\end{itemize}
Además, según la proposición
\ref{prop: base de fourier version real},
para todas estas $\omega$,
los vectores de la lista anterior son ortogonales entre
sí y tienen norma uno, luego, más que una base de 
$P_{n, \omega}$ constituyen una BON para este espacio.
Así, $\Pi_{P_{n, \omega}}(x)$ puede encontrarse 
simplemente calculando los productos punto 
de $x$ con los elementos de estas BONs (c.f. 
nota \ref{nota: sobre la identidad de parseval});
comparando este cálculo con la definición 
\ref{def: taus}
de los coeficientes $\tau_{n}(x, \omega)$,
concluimos que en efecto se
tiene la iguadad \eqref{eq4: 4May}.

\QEDB
\vspace{0.2cm}

Así, \textbf{el espectro basado en espacios monofrecuenciales
es una extensión de la definición del espectro 
basado en la transformada discreta de Fourier}.
Como ya recordamos al inicio, la
ventaja de este primer espectro es que para calcularlo es posible usar
un rango cualquiera de frecuencias no negativas, mientras que el segundo, 
a pesar de que da no sólo un proceso de análisis de una señal 
de $x$ a partir de sinusoides, sino también uno de síntesis
(c.f. nota \ref{nota: ya?}), se limita a considerar las frecuencias 
en $Dom_{TDF, n}$. \\

Recuerde que en la nota 
\ref{nota: muestreo dom frecuencia} explicamos que 
basta evaluar a $\Sigma_{x}$ en el intervalo $[0, n/2]$, 
pues a partir de estos valores puede deducirse el valor
del espectro en cualquier otra frecuencia positiva; observe que,
para toda $n$, el conjunto de frecuencias enteras 
consideradas por la TDF, $Dom_{TDF, n}$, está
contenido en $[0, n/2]$, luego, basta calcular 
a $\Sigma_{x}$ en $[0, n/2]$
para tener ambos análisis espectrales.


\begin{nota}
\label{nota: la mejor frecuencia}
Fijados $n \geq 2$
y $x \in \IR^{n}$, \textbf{entre más cercano a $1$ sea 
el coeficiente 
$\sigma_{n}(x, \omega)$, mejor es usar un sinusoide
de frecuencia $\omega$ para ajustar la gráfica de $x$}.
Esto porque, recuerde, entre más cercano a uno sea uno de
esos coeficientes, más cercana estará la señal $x$ 
al espacio monofrecuencial $P_{n, \omega}$, luego, más
cercana está $x$
a tener
la propiedad de ser una discretización de un sinusoide
de la respectiva frecuencia $\omega$.
\end{nota}

\begin{nota}
\label{nota: proyeciones monof TDF}
Sea $x \in \IR^{n}$; sea 
\eqref{ec: sintesis 0} o 
\eqref{ec: sintesis 1}
(dependiendo de la paridad de $n$)
la síntesis de $x$ respecto a la base de Fourier
real $\cali{F}_{n}$. De esta suma, podemos
separar los sumandos que corresponden a una
cierta frecuencia $\omega \in Dom_{TDF, n}$; recordando
que, como se notó
en la demostración de la proposición
\ref{prop: coinciden espectr}, 
los correspondientes vectores
de frecuencia $\omega$ (que son ya sea uno o dos, dependiendo del valor
de $\omega$) conforman una BON para el correspondiente
espacio monofrecuencial 
$P_{n, \omega}$, tenemos que la parte de la 
síntesis de $x$ que corresponde a 
cierta frecuencia $\omega$ es igual a
$\Pi_{P_{n, \omega}}(x)$. \\

Aplicando esto al ejemplo \ref{ej: DFT1},
si $x$ es la señal definida en 
\ref{eq2: 10ab}, se tiene que
\[
\Pi_{7, 0} = 4.12 c_{7,0}, 
\]
\[
\Pi_{7, 1} = -8.76 c_{7,1} - 7.35 s_{7,1}, 
\]
\[
\Pi_{7, 2} = 4.77 c_{7,2} - 10 s_{7,2}, 
\]
\[
\Pi_{7, 3} = 0.14 c_{7,3} + 9.91 s_{7,z3}.
\]
\end{nota}



\begin{ejemplo}
\label{ej: espectros comparacion}

Sea $f_{\omega}$
el sinusoide definido como
\begin{equation}
\label{eq: sinusoide eje}
f_{\omega}(t) := -1.5 cos (2 \pi \cdot \omega t + 2 \pi \cdot 0.2).
\end{equation}
Considere a una señal $x \in \IR^{16}$ que sea el resultado
de muestrear uniformemente al sinusoide
$f_{3.4}$
con ruido aleatorio 
(con distribución, pongamos, uniforme en $[-0.5, 0.5]$).

A continuación, mostramos las gráficas
de los espectros de $x$. Para
calcular el espectro $\Sigma_{x}$,
usamos el dominio
establecido en la nota 
\ref{nota: muestreo dom frecuencia}.

\begin{figure}[H]
\centering
	\sidecaption{ De ahora en adelante, siempre que
	grafiquemos espectros,usaremos los colores y notaciones
	de esta figura. \label{fig: ejemplo_comparacion}}
    \includegraphics[scale = 1]{./estudios_espectrales/ejemplo_comparacion_1}
\end{figure}


Observe cómo el espectro-$1$ parece completar la información
dada por el espectro-$0$, pues, a diferencia del primero,
el espectro-$0$
da coeficientes de frecuencia $\tau_{n}(x, \omega)$ sólo
para algunas frecuencias enteras $\omega$, mientras que en el espectro-$1$
es posible considerar cualesquiera frecuencias $\omega \geq 0$; como puede observar
en la gráfica, 
\[
FP0 (x) = 3 \hspace{0.2cm} \textit{y} \hspace{0.2cm}
FP1 (x) = 3.42;
\]
esta segunda frecuencia es mucho más cercana a
la frecuencia $\omega =3.4$ del sinusoide del que
fue obtenida la señal $x$; como agregamos ruido
aleatorio en las mediciones, no 
es de extrañarse que no se haya
obtenido exactamente $FP1(x) = 3.4$.

A pesar de que el espectro-$0$, el obtenido a partir de la
transformada discreta de Fourier, no dio una mala estimación (del rango
de frecuencias $Dom_{TDF,n}$ considerado por esta herramienta,
$\omega =3$ es el valor más cercano al valor real $\omega = 3.4$), vemos en este
ejemplo que usando el espectro-$1$ es posible obtener mejores
estimaciones de frecuencias que modelen a la señal original. \\

Mostramos ahora la gráfica de $x$ junto con
\begin{itemize}
	\item la parte de la síntesis de $x$ respecto a la $TDF$
	que corresponde a la frecuencia principal
	$FP0(x)$ (c.f.
	nota \ref{nota: ya?}), que de hecho,
	según la nota \ref{nota: proyeciones monof TDF}, es
	$\Pi_{P_{36,3}}(x)$
	\begin{figure}[H]
			\sidecaption{
			Puesto que $\{ c_{36, 3}, s_{36, \omega} \}$
			es una BON de $P_{36, 3}$, claro que la señal 
			que resulta de discretizar el sinusoide morado en la malla
			$I_{36}$ es, de hecho, la proyección de $x$ 
			al espacio monofrecuencial $P_{36, 3}$.
 			\label{fig: comparacion 2}
			}
			\centering
			\includegraphics[scale = 0.5]{./estudios_espectrales/ejemplo_comparacion_2} 
		\end{figure}		
	
	y
	\item la señal $\Pi_{P_{16, 3.42}}(x)$, o sea, la señal de
	dimensión $16$ y frecuencia $FP1(x)=3.42$ más cercana a $x$, junto con
	el sinusoide continuo del que fue muestreado.
	\begin{figure}[H]
			\sidecaption{
			Para obtener la versión continua del sinusoide 
			discreto $\Pi_{P_{36, 3.42}}(x)$ (i.e. la gráfica naranja),
			usamos las fórmulas establecidas en los teoremas
			\ref{teo: amelie1} y \ref{teo: amelie2}.
			\label{fig: comparacion 3}
			}
			\centering
			\includegraphics[scale = 0.5]{./estudios_espectrales/ejemplo_comparacion_3} 
		\end{figure}		
\end{itemize}


Los sinusoides de las figuras \ref{fig: comparacion 2} y
\ref{fig: comparacion 3}
son las señales de frecuencia pura
$3$ y $3.42$, respectivamente, cuya distancia euclidea
a la señal original $x$ es mínima. Observe que la segunda
señal, aquella cuya frecuencia
fue determinada
a partir del estudio espectral basado en espacios
monofrecuenciales,
parece ajustarse un poco mejor a la gráfica de $x$. \\

\begin{figure}[H]
			\sidecaption{
			Mostramos ahora los espectros de la señal $x$ que se obtiene
			tomando $36$ muestras uniformemente espaciadas del mismo sinusoide
			$f_{3.4}(t)$, 	
			esta vez sin agregar ruido aleatorio a las mediciones.
			Observe que el espectro-1 detectó a la frecuencia $\omega = 3.4$
			como la mejor, y que el sinusoide naranja ajusta perfectamente la gráfica
			de $x$. Como la frecuencia real no es entera, usar la frecuencia principal
			dada por la TDF sigue sin dar resultados perfectos, aunque no malos.
			\label{fig: sinusoide sin ruido}
			}
			\centering
			\includegraphics[scale = 0.4]{./estudios_espectrales/sinusoide_sin_ruido} 
		\end{figure}		



	\begin{figure}[H]
			\sidecaption{
			Si ahora muestreamos sin ruido
			del sinusoide $f_{5}$,
			como 
			era de esperarse, la frecuencia principal de ambos
			espectros es $5$
			(y el valor de los 
			espectros en tal frecuencia
			es igual a $1$, la cota
			superior). Además, 
			las gráficas de frecuencia $5$ que resultan
			ajustan a la perfección a la señal original $x$.
			\label{fig: comparacion 4}
			}
			\centering
			\includegraphics[scale = 0.4]{./estudios_espectrales/ejemplo_comparacion_4} 
		\end{figure}	


Para terminar este ejemplo, tomemos una suma de sinusoides
de varias frecuencias, digamos, de frecuencias
$1$, $4$ y $7$; sea
\[
g(t) = 3 sen(2 \pi t) + sen(2 \pi \cdot 4t) + 0.5 cos(2 \pi \cdot 7t).
\]
Sea $x$ la señal que resulta de muestrar, sin ruido, este sinusoide
$g$, siendo $25$ el tamaño de la muestra.
\begin{figure}[H]
	\sidecaption{
	En la imagen se muestran los espectros de tal $x$. Observe que los
	espectros parecen concentrarse alrededor de las 
	frecuencias $1$, $4$ y $7$.
	\label{fig: sin varias frec}
	}
	\centering
	\includegraphics[scale = 0.4]{./estudios_espectrales/sinusoide_varias_frecuencias} 
\end{figure}	

\final
\end{ejemplo}

\subsection{Adaptación del análisis espectral a señales reales con una frecuencia de muestreo dada}

Para hacer nuestros análisis espectrales, hemos
usado la dimensión $n$ del PDL $\cali{L}^{n, k} \in \IR^{n}$
en cuestión
para buscar, en base a máximos globales
del espectro 
$\Sigma_{x}: [0, \frac{n}{2}] \longrightarrow [0,1]$, la
mejor frecuencia $\omega$ para aproximar la gráfica
de $\cali{L}^{n,k}$ en base a un sinusoide discreto
de dimensión $n$. \\


Note que en esa discusión nunca hablamos de 
parámetros importantes para, de forma canónica, hacer
un análisis espectral, como lo son la
duración en tiempo de la señal o la frecuencia
de muestreo.

\begin{defi}
\label{def: tiempo y frec de muestreo}
La cantidad de muestras tomadas (de forma uniforme)
de una señal por unidad de tiempo 
será denotada por $F_{s}$ y llamada \textbf{frecuencia
de muestreo} de la señal. A la cantidad de unidades de
tiempo que dura la medición se le denotará por $T$. \\

A la cantidad total de muestras tomadas se le denotará
por $L$.
\end{defi}
Observe que, para poder dividir una unidad
de tiempo en $F_{s}$ subintervalos de igual longitud,
se deben divider al eje del tiempo con los puntos
\begin{equation}
t_{k} := t_{0} + h k, \hspace{0.2cm}
\textit{con } h := \frac{1}{F_{s}}.
\end{equation}
A tal constante $h$, definida como el recíproco de la frecuencia
de muestreo, se le llama el \textbf{paso temporal} de la señal. \\

De las definiciones se sigue de inmediato que
\begin{equation}
\label{eq: relacion L, T Fs}
L = T F_{s}.
\end{equation}
\begin{figure}[H]
	\sidecaption{
	Adoptamos la convención de empezar a medir 
	un bloque de $F_{s}$ mediciones desde que inicia la
	unidad de tiempo (que, en el caso de la figura, se ha
	fijado como segundos).
	\label{fig: Fs 1}
	}
	\centering
	\includegraphics[scale = 1]{Fs_1} 
\end{figure}	

Nosotros, por el momento, sólo nos
hemos enfocado en buscar
una frecuencia $\omega \in [0, \frac{n}{2}]$ que de lugar
a un sinusoide que aproxime bien la gráfica
de $\cali{L}^{n,k}$;
observe que, al hacer esto, hemos supuesto de forma
implícita que estamos estudiando una señal
de duración una unidad de tiempo
($T = 1$) de longitud $n$
(o sea, $L = F_{s} = n$). \\
Supongamos ahora que
tenemos una señal $x$ que consta de $L$ mediciones, siendo
$F_{s}$ la frecuencia de muestreo.
\begin{figure}[H]
	\sidecaption{
	Para la imagen, hemos fijado $F_{s}= 10$
	y $L = 40$.
	\label{fig: frecuencia 1}
	}
	\centering
	\includegraphics[scale = 0.45]{frecuencia_1} 
\end{figure}	

Sea ahora $2 \leq n \leq L$ y
supogamos que
se hizo el análisis
espectral 
de una sección $x_{|n}$ de tal señal
que conste de $n$ puntos
(usando el espectro
$\Sigma_{x}: [0, \frac{n}{2}]
\longrightarrow [0,1]$).
Digamos que, como conclusión de ese análisis, se
obtuvo que un sinusoide de frecuencia $\omega \in [0, \frac{n}{2}]$
ajusta bien \textit{esa sección particular 
$x_{|n}$
de la señal $x$}.
\begin{figure}[H]
	\sidecaption{
	Para la imagen, hemos fijado $n= 6$;
	se calculó que la mejor frecuencia para ajustar 
	los primeros $6$ puntos que componen la señal
	original $x$ es $w = 2$.
	\label{fig: frecuencia 2}
	}
	\centering
	\includegraphics[scale = 0.45]{frecuencia_2} 
\end{figure}	

Observe que, en general, si se quisiera usar
directamente una frecuencia de $w$ para ajustar
a la señal $x$, la aproximación lograda en los
$n$ puntos escogidos previamente puede no ser válida.
\begin{figure}[H]
	\sidecaption{
	Usando los datos de la figura \ref{fig: frecuencia 2}, podriamos
	intentar en un principio usar a un sinusoide de frecuencia $2$
	para intentar modelar a la señal, pero un sinusoide de tal frecuencia,
	como es el caso de esta figura, puede que ni siquiera sea adecuado
	para modelar el pedazo $x_{|n}$ original a partir del cual se obtuvo
	la frecuencia $\omega$.
	\label{fig: frecuencia 2}
	}
	\centering
	\includegraphics[scale = 0.45]{frecuencia_3} 
\end{figure}
Esto se debe a que	
tal frecuencia $\omega$ es buena para aproximar
a dichos $n$ puntos cuando se ha tomado como
unidad de tiempo a $n$, pero, 
por la forma en que fue muestreada la señal original $x$,
son $F_{s}$ (y no necesariamente $n$) la cantidad de puntos
que conforman una unidad. Así, puesto que con $\omega$
ciclos de un sinusoide se aproximaron $n$ puntos, 
la frecuencia que debe escogerse para aproximar a todos los $L$
puntos es
\begin{equation}
\label{eq: rel frecuencia real y ficticia}
\tilde{w} := \frac{F_{s}}{n} \omega.
\end{equation}

\begin{figure}[H]
	\sidecaption{
	Es con una simple regla de tres que se deduce
	la relación \eqref{eq: rel frecuencia real y ficticia}.
	\label{fig: frecuencia real}
	}
	\centering
	\includegraphics[scale = 1.4]{frecuencia_real} 
\end{figure}	
\begin{figure}[H]
	\sidecaption{
	Según los datos de las figuras
	\ref{fig: frecuencia 1}
	y \ref{fig: frecuencia 2}, con un sinusoide de frecuencia
	$\tilde{\omega} = 10/3$ se aproximan bien a los seis
	puntos en base a los cuales se encontró a la primera
	frecuencia $\omega$.
	\label{fig: frecuencia 4}
	}
	\centering
	\includegraphics[scale = 0.45]{frecuencia_4} 
\end{figure}
\section{PROVISIONAL; límites}

En \TODO{rojo} se resaltan las fórmulas que YA se han
verificado calculándolas dos veces.

\TODO{
\[
\frac{sen(2 \pi \omega) cos(2 \pi \omega \frac{n-1}{n})}{sen
(2 \pi \frac{\omega}{n})}
\sim
\frac{
2 \pi \omega - \frac{4 \pi^{3}}{3n^{2}}
(4n^2-6n+3) \omega^{3} + o(\omega^{5})
}{
\frac{2\pi}{n} \omega -
\frac{4 \pi^{3}}{3 n^{3}} \omega^{3} + o(\omega^{5})
},
\]
}
por lo que

\TODO{
\begin{align*}
\xi_{n, \omega} \sim &
\sqrt{2} 
\left(
4 \pi
\frac{                                                                                                                                          
\omega - \frac{2\pi^{2}}{3n^{2}}(2n^2-3n+2)\omega^{3} + o(\omega^{5})
}{
\frac{2\pi}{n} \omega -
\frac{4 \pi^{3}}{3 n^{3}} \omega^{3} + o(\omega^{5})
}
\right)^{-1/2} \\
\sim &
\sqrt{2} 
\left(
2n
\frac{                                                                                                                                          
\omega - \frac{2\pi^{2}}{3n^{2}}(2n^2-3n+2)\omega^{3} + o(\omega^{5})
}{
\omega -
\frac{2 \pi^{2}}{3 n^{2}} \omega^{3} + o(\omega^{5})
}
\right)^{-1/2} 
\rightarrow \frac{1}{\sqrt{n}},
\end{align*}
}

\TODO{
\begin{align*}
\eta_{n, \omega} \sim &
\sqrt{2} 
\left(
\frac{
8 \pi^{3} (2n-1)(n-1)\omega^{3} + o(\omega^{5})
}{
6 \pi n \omega -
\frac{4 \pi^{3}}{n} \omega^{3} + o(\omega^{5})
}
\right)^{-1/2} \\
\sim & 
\sqrt{2} 
\left(
4\pi^{2}
\frac{
(2n-1)(n-1)\omega^{3} + o(\omega^{5})
}{
3 n \omega -
\frac{2 \pi^{2}}{n} \omega^{3} + o(\omega^{5})
}
\right)^{-1/2} 
 \rightarrow \infty.
\end{align*}
}

\TODO{
\[
\langle
c_{n, \omega}, s_{n, \omega}
\rangle = 
\frac{2\pi}{n}
\xi_{n, \omega} \eta_{n, \omega}
(n-1)  
\left(
\frac{n}{2} \omega - \frac{2\pi^{2}}{3} (n-1) \omega^{3} + o(\omega^{5})
\right) \rightarrow ?.    
\]
}

\TODO{
\[
\langle x,
c_{n, \omega}
\rangle = 
\xi_{n, \omega} 
\left(
X_{0} - \frac{2 \pi^{2}}{n^{2}}X_{2} \omega^{2} 
+ \frac{2 \pi^{4}}{3n^{4}} X_{4} \omega^{4} + o(\omega^{5})
\right) \rightarrow \frac{X_{0}}{\sqrt{n}} ,
\]
}

\TODO{
\[
\langle x,
s_{n, \omega}
\rangle = 
\eta_{n, \omega} 
\left(
\frac{2 \pi}{n} X_{1} \omega - \frac{4 \pi^{3}}{3n^{3}}X_{3} \omega^{3} 
 + o(\omega^{5})
\right),
\]
}

Así, los elementos que parecen en la fórmula
para $\sigma_{n}(x, \omega)$ cuando $\omega \not\in \frac{n}{2} \IZ$ son
\begin{itemize}
\item 
\TODO{
\[
\langle
c_{n, \omega}, s_{n, \omega}
\rangle^{2} \sim
\frac{4\pi^{2}}{n^{2}}(n-1)^{2} \xi_{n, \omega}^{2} \eta_{n, \omega}^{2}
\left(
\frac{n^{2}}{4} \omega^{2} - \frac{2n}{3} \pi^{2} (n-1) \omega^{4} + o(\omega^{5})
\right)
\]
}

\item
\TODO{
\[
\langle x,
c_{n, \omega}
\rangle^{2} = 
\xi_{n, \omega}^{2}
\left(
X_{0}^{2} - \frac{4 \pi^{2}}{n^{2}}X_{0}X_{2} \omega^{2} 
+ 
\frac{4 \pi^{4}}{n^{4}}
\left(
\frac{1}{3} X_{0}X_{4} + X_{2}^{2}
\right) \omega^{4} 
+ o(\omega^{5})
\right)
\]
}

\item
\TODO{
\[
\langle x,
s_{n, \omega}
\rangle^{2} = 
\frac{4 \pi^{2}}{n^{2}}
\eta_{n, \omega}^{2}
\left(
X_{1}^{2}\omega^{2} - \frac{4 \pi^{2}}{3n^{2}}X_{1}X_{3} \omega^{4} 
+  o(\omega^{5})
\right)
\]
}

\item
\[
\langle
x, c_{n, \omega}
\rangle
\langle
x, s_{n, \omega}
\rangle
\langle
c_{n, \omega}, s_{n, \omega}
\rangle \sim
\frac{2 \pi^{2}}{n}
\xi_{n, \omega}^{2} \eta_{n, \omega}^{2}(n-1)
\left(
X_{0}X_{1}\omega^{2} 
- \frac{2 \pi^{2}}{n}
\left(
\frac{1}{3n} X_{0}X_{3} + \frac{1}{n} X_{2}X_{1}
+ \frac{2}{3}(n-1)X_{0}X_{1}
\right) \omega^{4}
\right).
\]
\end{itemize}

\TODO{Cambia los iguales por $\sim$.}
Según estos cálculos, si
$a_{n, \omega} = \langle x, c_{n, \omega} \rangle$, 
$b_{n, \omega} = \langle x, s_{n, \omega} \rangle$, 
$c_{n, \omega} = \langle c_{n, \omega}, s_{n, \omega} \rangle$, 
entonces

\[
a_{n, \omega}^{2} + b_{n, \omega}^{2} - 
2a_{n, \omega}b_{n, \omega}c_{n, \omega}
= N_{1} + N_{2} +N_{3},
\]
donde

\[
N_{1} =
\xi_{n, \omega}^{2} \left(
X_{0}^{2} - \frac{4\pi^{2}}{n^{2}} X_{0}X_{2}
\omega^{2} + \frac{4 \pi^{4}}{n^{4}}
\left(
\frac{1}{3} X_{0}X_{4} + X_{2}^{2}
\right) \omega^{4}
\right)
\rightarrow 
\frac{X_{0}}{n}
\left(
X_{0}  - \frac{4 \pi^{2}}{n^{2}}X_{2}
\right),
\]

\[
N_{2} = \frac{4 \pi^{2}}{n^{2}} \eta_{n, \omega}^{2}
\left(
X_{1}^{2} \omega^{2} - \frac{4 \pi^{2}}{3n^{2}}X_{1}X_{3} \omega^{4}
+ o(\omega^{5})
\right)
=
\frac{16 \pi^{3}}{n^{2}}
\frac{3n \omega - \frac{2 \pi^{2}}{n} \omega^{3} + o(\omega^{5})}{
8 \pi^{3}(2n-1)(n-1) \omega^{3} + o(\omega^{5})
}
\rightarrow \infty,
\]

\[
N_{3} =
\frac{-4 \pi^{2}}{n}
\xi_{n, \omega}^{2} \eta_{n, \omega}^{2}(n-1)
\left(
X_{0}X_{1}\omega^{2} 
- \frac{2 \pi^{2}}{n}
\left(
\frac{1}{3n} X_{0}X_{3} + \frac{1}{n} X_{2}X_{1}
+ \frac{2}{3}(n-1)X_{0}X_{1}
\right) \omega^{4}
\right)
\rightarrow ?,
\]
luego, no puedo determinar este último límite, por lo tanto,
tampoco hablar solbre el límite de
de $a_{n, \omega}^{2} + b_{n, \omega}^{2} - 
2a_{n, \omega}b_{n, \omega}c_{n, \omega}$, pues
tengo una indeterminación del tipo 
$0 \cdot \infty$. \\

El denominador es
\[
1-c^{2} \sim
1 - \frac{4\pi^{2}}{n^{2}}(n-1)^{2} \xi_{n, \omega}^{2} \eta_{n, \omega}^{2}
\left(
\frac{n^{2}}{4} \omega^{2} - \frac{2n}{3} \pi^{2} (n-1) \omega^{4} + o(\omega^{5})
\right) \rightarrow ?,
\]
aquí también tengo una indeterminación de tipo
$0 \cdot \infty$.


\chapter{Análisis espectrales: resultados numéricos}
\label{chap: resultados numericos analisis espectrales}

Después de todo lo expuesto en las secciones anteriores, tenemos
ya dos alternativas para graficar el espectro
de una señal $x \in \IR^{n}$.

\begin{itemize}
	\item Usando la transformada discreta de Fourier
	(c.f. sección \ref{sec: TDF}), el espectro de
	una señal $x$ es la gráfica de las frecuencias
	enteras dadas (dependiendo de la 
	paridad de $n$) por las
	tablas 6.1 y 6.2
	versus los coeficientes
	$\tau_{n}(x)$ definidos en
	\ref{def: taus}.
	\TODO{Puesto que hacer un análisis con la
	DFT significa expresar a $x$ como una suma
	ponderada de muestreos de senos y cosenos
	de algunas frecuenicas enteras,...}
	
	\item Si, para hacer un análisis espectral, se usan
	las ideas propuestas en 
	la sección
	\ref{sec: metodologia para realizar un analisis espectral que considere frecuencias arbitrarias}, entonces, dado un rango de frecuencias 
	$\omega$,
	el espectro de $x$ es la gráfica de 
	las frecuencias $\omega$ versus	
	los coeficientes
	$\sigma_{n}(x, \omega)$.
	

\TODO{a diferencia de la dft, los casos extremos ya no son
estar o no estar, sino ser perpendicular o ser paralelo a su
representante del espacio monofrecuencial.}
\end{itemize}


\begin{figure}[H]
	\sidecaption{
	Ejemplo de los espectros resultantes
	de los dos métodos de análisis.
	\label{fig: espectro 1 }
	}
	\centering
	\includegraphics[scale = 1]{ejemplo_analisisEspectrales} 
\end{figure}	