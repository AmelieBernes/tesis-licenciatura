\chapter{Estudio espectral}
\label{chap: estudio espectral}


\section{Motivación para realizar un estudio espectral de los PDL}
No es difícil convencerse de que las condiciones de ortogonalidad
impuestas en la definición de la base de discreta de Legendre
$\cali{L}^{n}$
forzan a las entradas de los polinomios discretos $\cali{L}^{n,k}$
a cambiar más frecuentemente de signo conforme aumenta
el grado $k$, luego, conforme $k$ tiende a $n-1$,
la cantidad de oscilaciones aumenta; revisemos, 
por ejemplo, el caso $n=4$.



\begin{minipage}{0.5\textwidth}
\begin{figure}[H]
\includegraphics[scale=0.25]{oscil1}
\end{figure}
\end{minipage} \hfill
\begin{minipage}{0.45\textwidth}
1.- Por definición,
$\mathcal{L}^{4,0}$ se obtiene al normalizar 
al vector constante uno de $\IR^{4}$, o sea, 

\[
\cali{L}^{4,0} = \left(
\frac{1}{2}, \frac{1}{2}, \frac{1}{2}, \frac{1}{2}
\right).
\]
\end{minipage} 


\begin{minipage}{0.5\textwidth}
2.- La señal $\cali{L}^{4,1} \in \IR^{4}$ es un polinomio discreto de
dimensión 4 y grado 1 que se obtiene exigiendo las
siguientes condiciones

\[
\langle \cali{L}^{4,1} , \cali{L}^{4,0} \rangle=0
\hspace{0.2cm} \text{y} \hspace{0.2cm}
\langle \cali{L}^{4,1} , \cali{L}^{4,1} \rangle=1;
\]
esta primera condición se refleja en que 
las alturas de los puntos de la gráfica de 
$\cali{L}^{4,1}$ deben sumar cero;
esto implica un cambio de signo (y sólo uno,
pues el polinomio es lineal).

\end{minipage} \hfill
\begin{minipage}{0.45\textwidth}
\begin{figure}[H]
\includegraphics[scale=0.3]{oscil2}
\end{figure}
\end{minipage}

\noindent
3.- El vector
$\cali{L}^{4,2} \in \IR^{4}$ satisface las siguientes
tres condiciones:

\[
\langle \cali{L}^{4,2} , \cali{L}^{4,0} \rangle=0,
\hspace{0.2cm}
\langle \cali{L}^{4,2} , \cali{L}^{4,1} \rangle=0,
\hspace{0.2cm} \text{y} \hspace{0.2cm}
\langle \cali{L}^{4,2} , \cali{L}^{4,2} \rangle=1;
\]
observe que si las entradas de 
$\cali{L}^{4,2}$ fuesen todas positivas o todas negativas,
entonces no se tendría la ortogonalidad
con la señal constante $\cali{L}^{4,0}$.


\begin{figure}[H]
\includegraphics[scale=0.6]{oscil3}
\end{figure}

4.- Por último, $\cali{L}^{4,3} \in \IR^{4}$ satisface
las siguientes cuatro condiciones: 

\[
\langle \cali{L}^{4,3} , \cali{L}^{4,0} \rangle=0,
\hspace{0.2cm}
\langle \cali{L}^{4,3} , \cali{L}^{4,1} \rangle=0,
\langle \cali{L}^{4,3} , \cali{L}^{4,2} \rangle=0,
\hspace{0.2cm} \text{y} \hspace{0.2cm}
\langle \cali{L}^{4,3} , \cali{L}^{4,3} \rangle=1.
\]

Según los teoremas 
\ref{cor: propiedades importantes de espacios Wi}
y \ref{prop: simetrias en dimensiones pares},
$\cali{L}^{4,3}$ es la discretización de un polinomio
cúbico y además es una señal antisimétrica (en el sentido de la definición
\ref{def: espacios de seniales simetricas y antisimetricas}); dos gráficas de señales que cumplen esto se ilustran abajo:

\begin{figure}[H]
	\includegraphics[scale=0.6]{oscil4}
	\sidecaption{Observe que la señal cúbica de la izquierda tiene
	un sólo cambio de signo, mientras que la de la derecha (que es 
	$\cali{L}^{4,3}$ tiene tres.)}
\end{figure}

La señal cúbica de la izquierda, a pesar de 
ser ortogonal a $\mathcal{L}^{4,0}$ y a $\cali{L}^{4,2}$ por 
simetría (c.f. lema
\ref{lema: ortogonalidad entre sim y antisim}), 
definitivamente no puede ser ortogonal
a $\cali{L}^{4,1}$, pues las entradas de estos dos vectores de 
$\IR^{4}$ tienen el mismo signo. Sin embargo, la señal cúbica 
de la derecha (que de hecho es $\cali{L}^{4,3}$) sí cumple el ser
ortogonal a $\cali{L}^{4,1}$ pero, para lograr esto, sus
entradas deben cambiar de signo tres veces. \\


Estando de acuerdo en que es pertinente realizar un análisis
espectral de los PDL, después de dibujar las gráficas de algunos
de estos para varias dimensiones $n$, proponemos una hipótesis
que relaciona una presencia importante de cierta frecuencia
de oscilación asociada a un PDL.


\begin{figure}[H]
	\sidecaption{
	Aproximando las gráficas
	de $\cali{L}^{60,0}$ y $\cali{L}^{60,1}$
	con sinusoides. 
	\label{fig: hip_0,1}
	}
	\centering
	\includegraphics[scale = 0.6]{hip_0,1} 
\end{figure}	

\begin{figure}[H]
	\sidecaption{
	Aproximando las gráficas
	de $\cali{L}^{60,2}$ y $\cali{L}^{60,3}$
	con sinusoides. 
	\label{fig: hip_2,3}
	}
	\centering
	\includegraphics[scale = 0.6]{hip_2,3} 
\end{figure}	

\begin{figure}[H]
	\sidecaption{
	Aproximando las gráficas
	de $\cali{L}^{60,4}$ y $\cali{L}^{60,5}$
	con sinusoides. 
	\label{fig: hip_4,5}
	}
	\centering
	\includegraphics[scale = 0.6]{hip_4,5} 
\end{figure}	

\begin{hip}
\label{ref: hipotesis}
Sean $n \geq 2$, $0 \leq k \leq n-1$ entero. 
Sea $\cali{L}^{n,k}$ el PDL de dimensión $n$ y grado $k$.
La frecuencia que mejor modela la oscilación de la gráfica
de $\cali{L}^{n,k}$ es $\frac{k}{2}$ 
\end{hip}

Se ha formulado de forma intencional la hipótesis 
\ref{ref: hipotesis} en términos vagos, para tener libertad
después de definir las herramientas con las que abordaremos el problema
de dar forma concreta a la hipótesis y mejorarla o refutarla con simulaciones
numéricas.

\section{La transformada discreta de Fourier y estudios espectrales de señales finitas}

\TODO{aquí una introducción}
\TODO{La teoría se basa en el libro de Algoritmos.}

La teoría de esta sección se apoya en la existencia de las raíces
$n-$ésimas de la unidad (garantizada por el teorema fundamental
del álgebra \ref{teo: fundamental del algebra}) 
y propiedades de estas que hacen posible
tener un método eficiente de calcular lo que después llamaremos
la ``transformada discreta'' de un elemento de $\IC^{n}$.

\subsection{Raíces $n-$ésimas de la unidad y propiedades de estas}

\TODO{Habla sobre la exponencial compleja; tal vez pon su definición,
pero di que no entras en detalles (estos pueden consultarse, por ejemplo, en Marsden), sólo citamos unas propiedades de esta función de $\IC$ a $\IC$
que necesitaremos}

\TODO{Tal vez puedes hablar de cómo esta forma involucra al ángulo y
a la norma para representar a un punto, y por qué esto es útil
para hablar de mulitplicación y conjugados! También tienes que definir
la norma en $\IC^{n}$, esto no es trivial y de hecho me estaba dando
problemas.}

\begin{prop}
\label{prop: propiedades exp compleja}
(\textbf{Propiedades de la exponencial compleja}) a
	\begin{itemize}
	\item $exp(z) = 1$ si y sólo si $z= 2K \pi i$ para algún $K \in \IZ$
	\item Para todo $\omega \in \IZ$ y todo $z \in \IC$, $(exp(z))^{\omega} = 			exp(\omega z)$ \TODO{ve si $\omega$ puede de hecho ser cualquier complejo? 			tiene sentido esto?}
	\item Para cualesquiera $z_{1}, z_{2} \in \IC$, 
	$\frac{exp(z_{1})}{exp(z_{2})} = exp(z_{1} - z_{2})$.
	\item Para todo entero $n$ y todo real $b$,
	$(exp(bi))^{m} = exp (mb i)$
	\item \TODO{pon lo de exponencial de la suma.}
	\end{itemize}
\end{prop}


\begin{defi}
\label{defi: raices n esimas de la unidad}
Sea $n \in \IN$. A las $n$ raíces del polinomio
$p_{n}(t)= t^{n}-1$ se les denominará las \textbf{raíces $n-$ésimas de la unidad.}
\end{defi}


Las raíces $n-$ésimas de la unidad son pues los números complejos
tales que, elevados a la potencia $n$, son iguales a 1; según el 
teorema fundamental
del álgebra \ref{teo: fundamental del algebra}, sí hay números complejos
que satisfacen la definición \ref{defi: raices n esimas de la unidad}, y además
son a lo más $n$. Es fácil establecer, como hacemos a continuación, 
fórmulas explícitas \TODO{continua.}

\begin{prop}
Sea $n \in \IN$, $n \geq 2$. Hay exactamente $n$ raíces $n-$ésimas de la
unidad, y estas son los números complejos
 	\begin{equation}
	\label{eq3: 8ab}
	z_{n, \omega} : = exp \left( \frac{2 \pi i }{n} \omega
	\right), \hspace*{0.2cm} \textit{con} 
	\hspace*{0.2cm} \omega \in \{0, 1, \ldots, n-1 \}.
	\end{equation}
	
\end{prop}
\noindent
\textbf{Demostración.}
Por las propiedades expresadas en la proposición
\ref{prop: propiedades exp compleja}, es fácil ver que 
$z_{n,1} :=  exp \left( \frac{2 \pi i }{n} \right)$ es raíz $n-$ésima
de la unidad, pues
\[
(z_{n,1})^{n} = exp(2 \pi i ) = 1.
\]
Además, para todo $\omega \in \{ 0, \cdots , n-1 \}$, el número
\[
z_{n, \omega} : = (z_{n,1})^{\omega} = exp \left( \frac{2 \pi i }{n} \omega \right)
\]
también es es raíz $n-$ésima de la unidad, ya que

\[
(z_{n, \omega})^{n} = ((z_{n,1})^{\omega} )^{n} = 
((z_{n,1})^{n} )^{\omega} = 1^{\omega}=1. 
\]
Note ahora que los $n$ números complejos $z_{n, \omega}$ son todos 
distintos entre sí, pues si $\omega_{1}$ y $\omega_{2}$ son enteros
entre $0$ y $n-1$ tales que $z_{n, \omega_{1}} = z_{n, \omega_{2}}$,
o sea, tales que 
$exp \left( \frac{2 \pi i }{n} \omega_{1} \right) = 
exp \left( \frac{2 \pi i }{n} \omega_{1} \right)$, entonces, según el tercer
punto de la proposición \ref{prop: propiedades exp compleja},
$1 = exp \left( \frac{2 \pi i }{n} (\omega_{1}-\omega_{2}) \right)$, luego, 
según el primer punto de esta misma proposición, $\frac{\omega_{1}-\omega_{2}}{n}$
es entero, o sea, $n$ divide a $\omega_{1}-\omega_{2}$; por el rango de 
$\omega_{1}$ y $\omega_{2}$, esto sólo ocurre si $\omega_{1}-\omega_{2}$ es
cero, o sea, si $\omega_{1}$ y $\omega_{2}$
son iguales.
\QEDB
\vspace{0.2cm}

\TODO{Aquí la figura de siempre:)}



\subsection{DFT}




\begin{prop}
Sea $n \in \IN$. El conjunto

\begin{equation}
\label{eq2: 8ab}
\cali{B}_{n} : = \{
e_{\omega} = \left(
\frac{1}{\sqrt{n}} exp \left(
2 \pi i \omega \frac{m}{n}
\right)
\right)_{0 \leq m \leq n-1}
: \hspace{0.2cm} 0 \leq \omega \leq n-1
 \}
\end{equation}
es una base ortonormal del $\IC-$espacio
vectorial $\IC^{n}$.
\end{prop}

\noindent
\textbf{Demostración.}
Calculemos el producto punto de dos elementos
$e_{\omega_{1}}$ y $e_{\omega_{2}}$ del conjunto \eqref{eq2: 8ab};
si $\omega := \omega_{1}-\omega_{2}$,
\begin{align*}
\langle e_{w_{1}}, e_{w_{2}} \rangle = &
\frac{1}{n}
\suma{m=0}{n-1}{exp \left( 2 \pi i \frac{m}{n} \omega_{1} \right)
\cdot \overline{ exp \left( 2 \pi i \frac{m}{n} \omega_{2} \right) }} \\
= & \frac{1}{n}
\suma{m=0}{n-1}{\left( 2 \pi i \frac{m}{n} (\omega_{1}-\omega_{2}) \right)} \\
= & \frac{1}{n}\suma{m=0}{n-1}{exp\left( 2 \pi i \frac{\omega}{n} m \right)} \\
= & \frac{1}{n}\suma{m=0}{n-1}{exp\left( 2 \pi i \frac{\omega}{n}  \right)^{m}} \\
= & \frac{1}{n}\suma{m=0}{n-1}{(z_{n, \omega})^{m}};
\end{align*}

\noindent
esta última es una suma geométrica. 
\begin{itemize}
	\item Si $\omega_{1} \neq \omega_{2}$, entonces $n$ no puede dividir 
	a $\omega = \omega_{1}-\omega_{2}$ (pues, por el rango en el que se encuentran
	$\omega_{1}$ y $\omega_{2}$, $w \in [-(n-1), n-1]$, y el único múltiplo
	de $n$ en este intervalo es cero), luego, $z_{n, \omega} \neq 1$.
	En este caso se tiene entonces que 
	\[
	\langle e_{w_{1}}, e_{w_{2}} \rangle = 
	\frac{1}{n}\suma{m=0}{n-1}{(z_{n, \omega})^{m}}
	= \frac{1}{n} \cdot \frac{(z_{n, \omega})^{n}-1}{z_{n, \omega}-1}=
	\frac{1}{n} \cdot \frac{1-1}{z_{n, \omega}-1}=0.
	\]
	
	\item SI $\omega_{1} = \omega_{2}$, entonces $\omega = 0$, y
	\[
	\langle e_{w_{1}}, e_{w_{2}} \rangle = 
	\frac{1}{n}\suma{m=0}{n-1}{(z_{n, 0})^{m}}
	= \frac{1}{n}\suma{m=0}{n-1}{1} = \frac{1}{n} \cdot n = 1.
	\]
\end{itemize}

Demostramos así que los elementos de $\cali{B}_{n}$
tienen norma uno (c.f. \TODO{ref ec. norma en $\IC^{n}$}) y que además
son ortogonales
dos a dos, luego, según \TODO{ref}, $\cali{B}_{n}$ es un subconjunto l.i. 
de $\IC^{n}$; como $\IC^{n}$ es un $\IC-$ espacio vectorial de 
dimensión $n$, concluimos lo deseado.
\QEDB
\vspace{0.2cm}

Por ser \eqref{eq2: 8ab} una BON de $\IC^{n}$, siempre es
posible expresar a un vector $x = (x_{m})_{0 \leq m \leq n-1} \in \IC^{n}$
como combinación lineal de los elementos de \eqref{eq2: 8ab}
y además los coeficientes están dados por los productos puntos
de $x$ y los elementos de \eqref{eq2: 8ab}, que son

\begin{align*}
\langle x, e_{\omega} \rangle = & 
\frac{1}{\sqrt{n}} \suma{m=0}{n-1}{x_{m} exp \left(
2 \pi i \omega \frac{m}{n}
\right)} \\
= & 
\frac{1}{\sqrt{n}} \suma{m=0}{n-1}{x_{m} 
\left(
exp \left( \frac{2 \pi i }{n} \omega
\right) \right)^{m}} \\
= & A_{x}(z_{n, \omega}),
\end{align*}


\noindent
donde $z_{n, \omega}$ es como en \eqref{eq3: 8ab} y 
$A_{x} = A_{x}(t) \in \IC[t]$ es el polinomio de 
coeficientes complejos definido 
a partir de $x$ como sigue:

	\begin{equation}
		\label{eq4: 8ab}
		A_{x}(t) = \suma{m=0}{n-1}{\frac{x_{m}}{\sqrt{n}} t }\in \IC[t];
	\end{equation}

\noindent
así, \textbf{calcular los coeficientes de $x \in \IC^{n}$ respecto
a la BON $\cali{B}_{n}$ es lo mismo que evaluar al polinomio 
$A_{x}$ de grado $n-1$ definido en \eqref{eq4: 8ab} en todas las raíces
$n-$ésimas de la unidad.} Un algoritmo para evaluar eficientemente
polinomios es pues necesario.\TODO{cita el FFT}

\begin{defi}
Al proceso de calcular los coeficientes de $x$
respecto a $\cali{B}_{n}$
se le conoce como el \textbf{cálculo de la 
transformada discreta de $x$}
\end{defi}

{\Huge{\textcolor{red}{Dominio: tiempo}}} 


{\Huge{\textcolor{red}{Dominio: frecuencia}}}

{\Huge{ $x = (x_{m})_{0 \leq m \leq n-1}$ }}

{\Huge{ $\langle x, e_{\omega} \rangle$, $0 \leq \omega \leq n-1 $ }}



Dada la motivación de antes, es claro cómo usar la transformada
discreta de 

\textbf{a esto se le llama un análisis espectral.}


\subsection{Versión real de la DFT}

En el caso en el que todas las entradas de un vector
$x = (x_{m})_{0 \leq m \leq n-1}$ sean reales, se puede definir
una base ortonormal de $\IR^{n}$ que se defina en base a muestreos uniformes
de sinusoides de frecuencias enteras.
Esta base será entonces, así como lo era \TODO{ref} para el 
caso complejo, un sistema de representación en el que 

\TODO{sintetizar una señal en términos de frecuencias.}

\begin{prop}
\label{prop: base de fourier version real}
Sean $n \in \IN$ mayor a uno, $M = \lceil \frac{n}{2} \rceil$.
Para cualquier $\omega >0$, sean los vectores 

	\begin{equation}
	\label{eq0: 10ab}
	c_{n, \omega} := \left( \sqrt{\frac{2}{n}} cos
	\left(2 \pi \omega \frac{m}{n}
	\right) \right)_{0 \leq m \leq n-1}
	\hspace{0.2cm} \textit{y} \hspace{0.2cm} 
	s_{n, \omega} := \left( \sqrt{\frac{2}{n}} sin
	\left(2 \pi \omega \frac{m}{n}
	\right) \right)_{0 \leq m \leq n-1}.
	\end{equation}

El subconjunto $\cali{F}_{n}$ de $\IR^{n}$ definido como

	\begin{itemize}
	\item $\cali{F}_{n} : = \{ c_{n,0}, c_{n,1}, s_{n,1},
	\ldots , c_{n,M-1}, s_{n,M-1}, c_{n,M} \}$ si $n$ es par
	(o sea, si $n=2M$), y como
	\item $\cali{F}_{n} : = \{ c_{n,0}, c_{n,1}, s_{n,1},
	\ldots , c_{n,M-1}, s_{n,M-1} \}$ si $n$ es impar
	(o sea, si $n=2M-1$)
	\end{itemize}
	
es una base ortonormal del $\IR-$espacio vectorial $\IR^{n}$.
\end{prop}

\TODO{Deberías poner una imagen de cómo muestreas sinusoides para
obtener los vectores de la base. Tal vez deberías introducir el 
término ``vector de frecuencia''.}

\noindent
\textbf{Demostración.}
Supongamos $n$ par. Si $0 \leq \omega_{1}, \omega_{2} \leq M$
son enteros, entonces
$\omega_{1} + \omega_{2}$ sólo es divisible por $n$ si ambos números
son iguales a $M$. Si suponemos a $\omega_{1}$ y $\omega_{2}$ distintos, 
entonces

\begin{align*}
\langle c_{, \omega_{1}} , c_{n, \omega_{2}} \rangle = &
\frac{1}{n} \suma{m=0}{n-1}{cos \left(2 \pi \omega_{1} \frac{m}{n} \right) \cdot 
cos \left(2 \pi \omega_{2} \frac{m}{n} \right)} \\
= &\frac{1}{2n} \left(
cos \left(2 \pi (\omega_{1} + \omega_{2}) \frac{m}{n} \right) +
cos \left(2 \pi (\omega_{1} - \omega_{2}) \frac{m}{n} \right)
\right) \\
= & \frac{1}{4n} (
\suma{m=0}{n-1}{
(exp(2 \pi m(\omega_{1}+\omega_{2})i/n) +
exp(-2 \pi m(\omega_{1}+\omega_{2})i) } \\
&  + exp(2 \pi m(\omega_{1}-\omega_{2})i/n) +
exp(-2 \pi m(\omega_{1}-\omega_{2})i)) )\\
\textit{(suma geométrica)} = & 
\frac{exp(2 \pi i (\omega_{1}+\omega_{2}))-1}{4n (exp(2 \pi i (\omega_{1}+\omega_{2})/n)-1)} +
\frac{exp(- 2 \pi i (\omega_{1}+\omega_{2}))-1}{4n (exp(-2 \pi i (\omega_{1}+\omega_{2})/n)-1)}
\\
& + 
\frac{exp(2 \pi i (\omega_{1}-\omega_{2}))-1}{4n (exp(2 \pi i (\omega_{1}-\omega_{2})/n)-1)} +
\frac{exp(- 2 \pi i (\omega_{1}-\omega_{2}))-1}{4n (exp(-2 \pi i (\omega_{1}-\omega_{2})/n)-1)};
\\
\end{align*}

\noindent
puesto que $\omega_{1}+\omega_{2}$ y $\omega_{1}-\omega_{2}$
son ambos enteros, según la proposición 
\ref{prop: propiedades exp compleja} las exponenciales de los numeradores
de esta última expresión son todas iguales a uno, luego, 
$\langle c_{n, \omega_{1}} , c_{n, \omega_{2}} \rangle  =0$. 


Con argumentos similares se prueba 
que todos los elementos de $\cali{F}_{n}$ tienen norma uno, así como
la ortogonalidad entre dos elementos
distintos del conjunto $\cali{F}_{n}$, por lo tanto, la independencia lineal de
este conjunto, luego, el que $\cali{F}_{n}$ sea base 
(ortonormal) de $\IR^{n}$.


\QEDB
\vspace{0.2cm}



\begin{defi}
Sea $n \in \IN$, $n \geq 2$. Llamaremos a la BON
$\cali{F}_{n}$ de $\IR^{n}$ definida en \ref{prop: base de fourier version real}
la \textbf{base de Fourier real de dimensión $n$}.
\end{defi}

Observe que $\cali{F}_{n}$, a diferencia de $\cali{B}_{n} \subseteq \IC^{n}$, 
considera frecuencias enteras no mayores a $M := \lceil \frac{n}{2} \rceil$
(cuando $n$ es par) o a $M-1$ (cuando $n$ es impar), mientras que
en $\cali{B}_{n}$ se consideran las frecuencias enteras entre $0$
y $n-1$ (inclusivo). Es decir, si 
\TODO{vamos a sintentizar a una señal respecto a menos frecuencias enteras.} 



\textbf{Ejemplo:} Consideremos a la señal 
\begin{equation}
\label{eq2: 10ab}
x=(-0.5,-8,-5.3,15,-0.3,6,4) \in \IR^{7}.
\end{equation}

Según la construcción de $\cali{F}_{7}$ (c.f. 
proposición \ref{prop: base de fourier version real}),
una expresión de $x$ respecto a $\cali{F}_{7}$ 
es una síntesis de $x$ a partir de señales 
de frecuencias $\omega = 0,1,2,3$. En la imágen de abajo
se muestran los coeficientes de $x$ respecto a $\cali{F}_{7}$.

\begin{figure}[H]
	\sidecaption{
	Se muestran la gráfica de $x$ junto con la gráfica de los
	coeficientes de $x$ respecto a la BON $\cali{F}_{7}$. Observe 
	que, por definición, sólo un vector de $\cali{F}_{7}$ tiene frecuencia
	cero (i.e. es constante), mientras que para las otras frecuencias
	tenemos dos vectores de la misma frecuencia, uno construido a partir de un 			coseno y otro a partir de un seno.
	\label{fig: ejFrecuencia 1}
	}
	\centering
	\includegraphics[scale=0.4]{ejFrecuencia_1} 
\end{figure}	

Se tiene la siguiente descomposición
\sidenote{Se redondearon los coeficientes.} de $x$;

\[
x = 4.12 c_{0} - 8.76c_{1} -7.35s_{1}+
4.77c_{2}-10s_{2}+0.14c_{3}+9.91s_{3}.
\]
A continuación mostramos las gráficas
de los sinusoides que fueron discretizados
para obtener los vectores de frecuencia
$0,1,2$ y $3$ en los que descompusimos a $x$.

\begin{figure}[H]
	\sidecaption{
	Aporte de frecuencia $0$.
	\label{fig: ejFrecuencia 2}
	}
	\centering
	\includegraphics[scale=0.4]{ejFrecuencia_2} 
\end{figure}	

\begin{figure}[H]
	\sidecaption{
	Aporte de frecuencia $1$.
	\label{fig: ejFrecuencia 3}
	}
	\centering
	\includegraphics[scale=0.4]{ejFrecuencia_3} 
\end{figure}	

\begin{figure}[H]
	\sidecaption{
	Aporte de frecuencia $2$.
	\label{fig: ejFrecuencia 4}
	}
	\centering
	\includegraphics[scale=0.4]{ejFrecuencia_4} 
\end{figure}	


\begin{figure}[H]
	\sidecaption{
	Aporte de frecuencia $3$.
	\label{fig: ejFrecuencia 5}
	}
	\centering
	\includegraphics[scale=0.4]{ejFrecuencia_5} 
\end{figure}	

Sumando todas las gráficas de la derecha, obviamente
obtenemos una función de cosenos y senos tal que,
al muestrearla uniformemente en $[0,1]$, obtenemos
al vector $x$ \eqref{eq2: 10ab}.

\begin{figure}[H]
	\sidecaption{
	En morado se muestra la gráfica de la función suma
	de las gráficas derechas en las figuras anteriores.
	\label{fig: ejFrecuencia 6}
	}
	\centering
	\includegraphics[scale=0.45]{ejFrecuencia_6} 
\end{figure}	


\TODO{Tal vez el algoritmos de la FFT no merezca una subsección. Mejor sólo 
exboza los detalles y cite al libro.}
\TODO{Para esto puedes apoyarte mucho en las notas del libro 
de Algorithms. Eso sí lo entendí bien.}





















\subsection{Fórmulas para el coseno del ángulo de un punto a un plano}
\label{ap: Caso particular en el que el subespacio en cuestión es un plano}

Para proponer una 
metodología 
alternativa al uso de la TDF
para realizar un análisis espectral,
necesitaremos medir ángulos de señales a planos
(i.e. subespacios de dimensión $2$).
En esta subsección nos dedicamos a 
obtener expresiones que usaremos después 
\TODO{aaa}. Vamos pues a 
desarrollar la teoría de la sección
para este caso particular.

La situación es la siguiente: $V$ es un $\IR-$espacio
de Hilbert, $u$ y $v$ son elementos de $V$,
unitarios y linealmente
independientes entre sí. El espacio que ellos generan
es pues un plano, digamos,


\[
P := span \{ u, v \}.
\]

\noindent
Dado $x \in V$,
el coseno del ángulo entre $x$ y $W$ es,
según la proposición
\ref{prop: algunos hechos sobre el angulo entre un vector y un subespacio},

\begin{equation}
\label{eq0: 19Marzo}
cos \left( \measuredangle (x, P) \right) = 
\frac{|| \Pi_{P}(x) ||}{||x||};
\end{equation}
para lograr expresar el lado derecho de la igualdad en términos
sólo de $u$, $v$ y $x$ (que son los elementos básicos de
nuestra discusión), conviene primero obtener, a partir 
de estos elementos, una base
ortonormal del espacio $P$.


\begin{obs}
Si $u, v \in V$ son unitarios y linealmente independientes, y $P$
es el plano que generan, entonces
$\{ u, z \}$, donde

\begin{equation}
\label{eq2: 19Marzo}
z:= \frac{v- \langle u, v \rangle u}{||v- \langle u, v \rangle u||}
\end{equation}
es una BON de $P$
\end{obs}
\noindent
\textbf{Demostración.}
Basta aplicar el teorema de Gram-Schmidt 
\ref{Teo:Gram-Schmidt}.
\QEDB
\vspace{0.2cm}

Teniendo una BON de $P$, según el 
corolario 
\ref{cor: proyeccion en terminos de BON}, se tiene la siguiente
expresión para la proyección de $x$ en $P$;

\begin{equation}
\label{eq1: 19Marzo}
\Pi_{P}(x)= \langle x, u \rangle u + \langle x, z \rangle z;
\end{equation}

\noindent
puesto que, según la definición \eqref{eq2: 19Marzo} de 
$z$ este vector es función de $u$ y $v$, fácilmente se
puede derivar, a partir de \eqref{eq1: 19Marzo},
una expresión de $\Pi_{P}(x)$ en función sólo
de $x$, $u$ y $v$. Se plasman las fórmulas 
concretas a continuación.
	\begin{prop}
	\label{prop: formulas 20Marzo}
	Sean $V$ un espacio de Hilbert, $x \in V$,
	$u,v \in V$ linealmente independientes
	y unitarios. Si $P$ es el plano
	que generan $u$ y $v$, entonces,

		\begin{equation}
		\label{eq0: 24ap}
		\Pi_{P}(x)= \frac{\langle x, u \rangle -\langle u, v \rangle \langle x, v \rangle }{1-\langle u, v \rangle^{2}} u + \frac{\langle x, v \rangle -\langle u, v \rangle \langle x, u \rangle }{1-\langle u, v \rangle^{2}} v
		\end{equation}
	y 
		\begin{equation}
		\label{eq3: 19Marzo}
		  || \Pi_{P}(x) ||^{2}=
		  \frac{\langle x, u \rangle^{2} +  \langle x, v \rangle^{2}	
	       -2  \langle x, u \rangle \langle x, v \rangle \langle u, v \rangle	}{1- \langle u, v 		\rangle^{2}}.
		\end{equation}
 
	\end{prop}

\noindent
\textbf{Demostración.}
La demostración consiste de simples manipulaciones aritméticas.
Según \eqref{eq1: 19Marzo},
\begin{align*}
\Pi_{P}(x) = & \langle x, u \rangle u + \langle x, z \rangle z \\
 = & \langle x, u \rangle u
 + \frac{\langle x, v \rangle - \langle u, v \rangle \langle x, u \rangle}{|| v -\langle u,v \rangle u ||^{2}}
(v - \langle u,v \rangle u);\\
\end{align*}

\noindent
puesto que $u$ y $v$ son unitarios, 
tenemos que
\begin{align}
\label{eq3: 23ap}
|| v -\langle u,v \rangle u ||^{2} = & 
\langle v,v \rangle^{2} -2
\langle u,v \rangle^{2} +\langle u,v \rangle^{2}\langle u,u \rangle \notag  \\
= & 1 -\langle u,v \rangle^{2}; 
\end{align}
sustituyendo \eqref{eq3: 23ap} en la última expresión para 
$\Pi_{P}(x)$ llegamos a \eqref{eq3: 19Marzo}. \\

Finalmente, 
\begin{align*}
|| \Pi_{P}(x) ||^{2} = & 
\langle x,u \rangle^{2} + \langle x,z \rangle^{2} \\
= & \langle x,u \rangle^{2} + 
\left(
\frac{\langle x,v \rangle - \langle u,v \rangle
\langle x,u \rangle}{||v -\langle u,v \rangle u ||}
\right)^{2};\\
\end{align*}

\noindent
sustituyendo \eqref{eq3: 23ap} en esta última expresión
llegamos a \eqref{eq3: 19Marzo}.

\QEDB
\vspace{0.2cm}

Usando las expresiones
\eqref{eq: coseno a subespacio}
y \eqref{eq3: 19Marzo} es fácil establecer
la siguiente proposición.

\begin{prop}
Sean $V$ un espacio de Hilbert, $x \in V$,
	$u,v \in V$ linealmente independientes
	y unitarios. Si $P$ es el plano
	que generan $u$ y $v$, entonces,
	
	
\begin{equation}
\label{eq: coseno a plano}
cos (\measuredangle (x, P)) = 
\sqrt{
\frac{\langle x, u \rangle^{2} +  \langle x, v \rangle^{2}	
	       -2  \langle x, u \rangle \langle x, v \rangle \langle u, v \rangle	}{
	       ||x||^{2} \cdot 
	       (1- \langle u, v 	\rangle^{2})  }}.
\end{equation}
\end{prop}


\section{Estudio espectral de los PDL}

Ya podemos usar la base de Fourier real $\cali{F}_{n}$
definida en la proposición \ref{prop: base de fourier version real}
para hacer un estudio espectral de los PDL. \TODO{Grafico
algunos resultados en otro capítulo?} \\


Puesto que, por la construcción de $\cali{F}_{n}$, 
hacer un análisis espectral de una señal $x \in \IR^{n}$
via su análisis respecto a la BON $\cali{F}_{n}$ nos lleva
a considerar sólo ciertas frecuencias enteras
(c.f. nota \ref{nota: frecuencias en las bases de fourier}),
queremos no sólo usar la TDF para realizar
nuestro estudio espectral, pues
no queremos restringirnos
al estudio de frecuencias enteras
(después de todo, según la hipótesis planteada en 
\ref{ref: hipotesis}, 
creemos que la frecuencia que mejor aproxima al PDL
$\cali{L}^{n,k}$ es $\frac{k}{2}$, y este último número no siempre
es un entero), sino que nos gustaría
\begin{enumerate}
	\item poder elegir una frecuencia $\omega \geq 0$ respecto
a la cual comparar a la señal y,
	\item una vez fijada una frecuencia, buscar el desfase $\phi \in [0,1]$
	que mejor ajuste la gráfica de $x$.
\end{enumerate}

\begin{figure}[H]
	\sidecaption{
	Aquí se grafica una misma señal $x \in \IR^{16}$ y se 
	compara con dos sinusoides de frecuencia $3.6$, una con 
	desfase (normalizado) 0.8 y otra con 0.32. Observe que
	la primera parece ajustar mucho mejor la gráfica de $x$.
	\label{fig: ejemplo desfase}
	}
	\centering
	\includegraphics[scale=0.45]{desfase_ejemplo} 
\end{figure}	


Vamos a seguir
una linea de razonamiento totalmente análoga a la empleada 
en el ejemplo \ref{subs: ejm 3}, pues aquí también abordamos el problema
definiendo cúmulos
\sidenote{Hablamos más precisamente
de subespacios de $\IR^{n}$, pero 
\TODO{notación de data science. Habla tamién de esto 
en el ap. de cosine sim.}}
de $\IR^{n}$ que consten de elementos que cumplan
determinada propiedad (en el caso del ejemplo \ref{subs: ejm 3}, la propiedad
era ser elemento de determinado espacio $W_{n,k}$ mientras que 
en esta sección la propiedad de nuestro interés es ``ser la discretización
de un sinusoide de frecuencia $\omega$'') y usando el coseno del ángulo que
una señal $x$ forma con dichos cúmulos para dar una medida de qué tanto
tiene $x$ la propiedad considerada.


usamos el coseno del ángulo que forma una señal $x \in \IR^{n}$
con el plano $W_{n,1}$ para dar una medida de qué tan afín es $x$;
queremos ahora hacer algo similar, y usar el coseno de $x$ con espacios
de frecuencias $\omega >0$ como
una medida de qué tanto reacciona $x$ a la frecuencia $\omega$.



\begin{notacion}
Para simplificar la notación, denotamos por $I_{n}$ al intervalo
$\{ \frac{m}{n}  : 0 \leq m \leq n-1 \}$.
\end{notacion}

Digamos qué es lo que 
entendemos por ``señal de frecuencia pura $\omega$''.

\begin{defi}
Sean $n \in \IN$,  $\omega>0$, $\phi \in [0,1[$.  
A toda señal $n-$dimensional  
de la forma

\begin{equation}
A \left(
cos \left(  2 \pi \omega t + 2 \pi \phi
\right)
\right)_{t \in I_{n}}
\end{equation}

\noindent
con $A \in \IR$, se le llamará
\textbf{señal $n-$dimensional de frecuencia
pura $\omega$}. En este contexto,
a $\phi$ se le llama el \textbf{desfase normalizado}
de la señal, y a $A$ la \textbf{amplitud}.
\end{defi}

\begin{nota}
Observe que toda señal de la forma
\begin{equation*}
A \left(
sin \left(  2 \pi \omega t + 2 \pi \phi
\right)
\right)_{t \in I_{n}},
\end{equation*}
con $A \in \IR$, también es una señal $n-$dimensional
de frecuencia pura $\omega$, pues, como 
\[
sen(x) = - cos (x+ \pi/2) \hspace{0.2cm}
\textit{para toda } x \in \IR,
\]
entonces
\begin{equation*}
A \left(
sin \left(  2 \pi \omega t + 2 \pi \phi
\right)
\right)_{t \in I_{n}} =
-A \left(
cos \left(  2 \pi \omega t + 2 \pi \phi^{'}
\right)
\right)_{t \in I_{n}},
\end{equation*}
donde $\phi^{'}= (\phi + 1/4) \% 1 \in [0,1[$.
\end{nota}


\begin{figure}[H]
	\sidecaption{
	Se grafica a la función 
	$f(t) = cos(2 \pi \cdot \frac{5}{2} t + 2 \pi \cdot 0.3)$;
	muestreando este sinusoide de forma uniforme en el 
	intervalo [0,1] obtenemos una señal de frecuencia pura
	$\omega = \frac{5}{2}$. En la figura, $n=5$.
	\label{fig: muestreo coseno}
	}
	\centering
	\includegraphics[scale= 0.55]{muestreo_coseno} 
\end{figure}	

\begin{prop}
\label{prop: para que frecuencias omega vector seno es cero}
	Sean $n \geq 2$, $\omega \geq 0$.
	\begin{itemize}
		\item El vector 
		\begin{equation}
		\label{eq: coseno omega}
		c_{n, \omega} = \left(cos(2 \pi \omega m/n) \right)_{m=0}^{n-1} \in \IR^{n}
		\end{equation}
		no es cero, y 
		\item el vector 
		\begin{equation}
		\label{eq: seno omega}
		s_{n, \omega} = \left(sen(2 \pi \omega m/n) \right)_{m=0}^{n-1} \in \IR^{n}
		\end{equation}
		es cero si y sólo si $\omega \in \frac{n}{2} \IZ$.
	\end{itemize}
\end{prop}
\noindent
\textbf{Demostración.}
El primer punto es fácil de probar, pues la primera entrada del
vector \eqref{eq: coseno omega} es 
$cos(0)=1$.

Supogamos ahora que $\omega>0$ es tal que \eqref{eq: seno omega}
es el vector cero, o sea, que
para toda $0 \leq m \leq n-1$, se tiene que 
$sen(2 \pi \omega m/n)=0$. En particular, ocurre
$sen(2 \pi \omega /n)=0$; esto implica la igualdad 
$2 \pi \omega /n = \pi K$ para algún entero $K$. Despejando
a $\omega$ de la ecuación tenemos que 
$\omega = \frac{n}{2}K \in \frac{n}{2} \IZ$. Recíprocamente,
todo $\omega$ de la forma 
$\frac{n}{2}K$, con $K \in \IZ$ hace que el vector 
\eqref{eq: seno omega} sea cero, pues, para toda $0 \leq m \leq n-1$,
$sen\left(2 \pi \frac{n}{2}K \frac{m}{n}\right)=
sen((Km)\pi)=0$.
\QEDB
\vspace{0.2cm}


\begin{obs}
\label{obs: f y g son l.i. y de norma uno}
Sean $n \geq 2$ entero, $\omega \geq 0$ con $\omega \not\in \frac{n}{2} \IZ$.
Los vectores \eqref{eq: coseno omega} y \eqref{eq: seno omega}
de $\IR^{n}$ son linealmente independientes.
\end{obs}
\noindent
\textbf{Demostración.}
Sólo note que 
la primera entrada de \eqref{eq: coseno omega} es $1$, mientras que  
la primera entrada de $g_{n, \omega}$ sea cero, pero no
todas sus entradas sean cero (c.f. proposición 
\ref{prop: para que frecuencias omega vector seno es cero}). 
\QEDB
\vspace{0.2cm}


Según la observación 
\ref{obs: f y g son l.i. y de norma uno}, si $\omega \not\in \frac{n}{2} \IZ$,
el espacio $P_{\omega}$ que generan los vectores 
\eqref{eq: coseno omega} y \eqref{eq: seno omega}

\begin{align}
\label{eq2: 20Marzo}
P_{\omega}:= & span(c_{n, \omega}, s_{n, \omega}) \notag  \\  
= &
\{ a \left( cos \left(2 \pi \omega t \right) \right)_{t \in I_{n}} +
b ( sen (2 \pi \omega t ))_{t \in I_{n}} : 
\hspace{0.2cm} a, b \in \IR \},
\hspace{0.1cm} \omega \not\in \frac{n}{2} \IZ
\end{align}

\noindent
es un plano (i.e. un subespacio de dimensión $2$) de $\IR^{n}$.
\begin{prop}
\label{prop: Pw consta de las señales de frecuencia omega}
Sean $n \in \IN$, $\omega \geq 0$ 
con 
$\omega \not\in \frac{n}{2} \IZ$.
El espacio $P_{\omega}$ definido en \eqref{eq2: 20Marzo} consta exactamente
de las señales $n$ dimensionales de frecuencia $\omega$.
\end{prop}

\noindent
\textbf{Demostración.}

Sea $\phi \in [0,1]$ un desfase cualquiera y $A \in \IR$
una amplitud cualquiera; por la regla
del coseno de la suma de dos ángulos, tenemos que
\[
A(cos (2 \pi \omega t + 2 \pi \phi))_{t \in I_{n}}
= Aa  (cos(2 \pi \omega t))_{t \in I_{n}} +
Ab  (sen(2 \pi \omega t))_{t \in I_{n}} \in P_{\omega} \in P_{\omega}
\]
donde
\[
a := cos (2 \pi \phi) \hspace{0.2cm} \text{y}
\hspace{0.2cm} b := sin (2 \pi  \phi).
\]


Recíprocamente, si $a, b \in \IR$ son escalares cualesquiera, 
el elemento genérico
$x=  a \left( cos \left(2 \pi \omega t \right) \right)_{t \in I_{n}} +
b ( sen (2 \pi \omega t ))_{t \in I_{n}} $ de $P_{w}$ puede
expresarse como sigue:

\begin{equation}
\label{eq1: 28Mar23}
x = \sqrt{a^{2}+b^{2}} \left(
A  \left( cos \left(2 \pi \omega t \right) \right)_{t \in I_{n}} +
B  \left( sin \left(2 \pi \omega t \right) \right)_{t \in I_{n}}
\right),
\end{equation}

\noindent
donde
\[
A := \frac{a}{\sqrt{a^{2}+b^{2}}} \hspace{0.2cm} \text{y} \hspace{0.2cm}
B := \frac{b}{\sqrt{a^{2}+b^{2}}}.
\]
Como $A^{2}+ B^{2}=1$, existe $\phi \in [0,1]$ tal que
\begin{equation}
\label{eq0: 28Mar23}
A = cos (2 \pi \phi) \hspace{0.2cm} \text{y}  \hspace{0.2cm}
B = sin (2 \pi \phi);
\end{equation}
sustituyendo \eqref{eq0: 28Mar23} en \eqref{eq1: 28Mar23}, llegamos
a que

\begin{align*}
x = &  \sqrt{a^{2}+b^{2}} (
cos(2 \pi \phi) \cdot (cos (2 \pi \omega t))_{t \in I_{n}} + 
sin(2 \pi \phi) \cdot (sin (2 \pi \omega t))_{t \in I_{n}} 
) \\
= & \sqrt{a^{2}+b^{2}} (
cos(2 \pi \phi) \cdot cos (2 \pi \omega t) +
sin(2 \pi \phi) \cdot sin (2 \pi \omega t) 
)_{t \in I_{n}}  \\
= &  \sqrt{a^{2}+b^{2}} (
cos (2 \pi \omega t - 2 \pi \phi)
)_{t \in I_{n}}.
\end{align*}

\QEDB
\vspace{0.2cm}


\noindent 
Según la proposición
\ref{prop: Pw consta de las señales de frecuencia omega},
$P_{\omega} \subseteq \IR^{n}$ es el plano que consiste de las señales
de dimensión $n$ y frecuencia (pura) $\omega$.

Si $\omega \in \frac{n}{2} \IZ$, entonces, según 
la proposición 
\ref{prop: para que frecuencias omega vector seno es cero}, el vector
$s_{n, \omega}$ es el vector cero y $c_{n, \omega}$, luego, el espacio
\begin{align}
\label{eq0: 23Ap}
P_{\omega}:= & span(c_{n, \omega}, s_{n, \omega}) \notag  \\  
= &
\{ a \left( cos \left(2 \pi \omega t \right) \right)_{t \in I_{n}} : 
\hspace{0.2cm} a \in \IR \},
\hspace{0.1cm} \omega \in \frac{n}{2} \IZ
\end{align}
es una recta (i.e. un subespacio de dimensión $1$)
de $\IR^{n}$.

\begin{defi}
Si $n \geq 2$ y $\omega>0$, entonces
al subespacio $P_{\omega}$ 
de $\IR^{n}$,
definido en \eqref{eq2: 20Marzo} si 
$\omega \not\in \frac{n}{2} \IZ$ y en 
\eqref{eq0: 23Ap} en caso contrario,
le llamaremos el \textbf{espacio monofrecuencial
$n$ dimensional} de frecuencia $\omega$.
\end{defi}


Es razonable pues
medir la cercanía de una señal $n-$dimensional $x \in \IR^{n}$
a tener frecuencia $\omega$
con el ángulo que $x$ forma con el subespacio $P_{\omega}$,
cuyo coseno, según la proposición
\ref{prop: algunos hechos sobre el angulo entre un vector y un subespacio}
es
\begin{equation}
\label{eq0: 20Mar}
cos \left( \measuredangle (x, W) \right) = \frac{|| \Pi_{W}(x) ||}{||x||}
\in [0,1].
\end{equation}


\begin{figure}[H]
	\sidecaption{
	Según la relación \eqref{eq0: 20Mar}, 
	si $\frac{||\Pi_{P_{\omega}}(x)||}{||x||}$ es cercano 
	a uno (resp. a cero), entonces $x$ es muy parecido a una señal de frecuencia $\omega$
	(resp. se aleja de ser una señal de frecuencia $\omega$).
	\label{fig: 20Mar23_1}
	}
	\centering
	\includegraphics[scale= 1]{20Mar23_1} 
\end{figure}	

Si $x$ es unitaria,
tenemos la relación simplificada 

\begin{equation}
\label{eq1: 20Mar}
cos \left( \measuredangle (x, W) \right) = || \Pi_{W}(x) || 
\hspace{0.5cm} (x \hspace{0.1cm} \text{unitario).}
\end{equation}


\textbf{Usaremos pues, para dar una medida de qué tanto
reacciona una señal $x \in \IR^{n}$ a una frecuencia
$\omega >0$
el número 
$\frac{||\Pi_{P_{\omega}}(x)||}{||x||} \in [0,1]$.} \\




Para usar los
resultados expuestos en el apéndice
\ref{ap: Caso particular en el que el subespacio en cuestión es un plano}
conviente establecer una fórmula para
el producto punto entre 
los vectores $f_{n, \omega}$ y $g_{n, \omega}$.
Hacemos esto a continuación.

\begin{prop}
\label{prop: producto punto entre f y g}
Fijados $n \geq 2$ y $\omega \geq 0$ con $n \nmid \omega$, los vectores
$f_{n, \omega}$ y $g_{n, \omega}$, como se definieron
en \eqref{eq5: 19Marzo}
y \eqref{eq6: 19Marzo}, respectivamente, 
son unitarios y linealmente independientes.
Además, 

\begin{equation}
\label{eq9: 19Marzo}
\langle f_{n, \omega} , g_{n, \omega} \rangle =
\frac{\xi_{n, w} \eta_{n, \omega}}{2} \cdot 
\frac{sen(2 \pi \omega)
sen(2 \pi \omega \left( 1- \frac{1}{n} \right))}{sen \left(2 \pi 
\frac{\omega}{n} \right)}
\end{equation}

\end{prop}
\noindent
\textbf{Demostración.}


Aquí usaremos las siguientes tres igualdades:

\begin{equation}
\label{eq10: 19Marzo}
\forall \alpha \in \IR: \hspace{0.2cm}
sen(2 \alpha) = 2 sen(\alpha) cos(\alpha)
\end{equation}



\begin{equation}
\label{eq11: 19Marzo}
\forall z\in \IR: \hspace{0.2cm}
sen(z)= \frac{e^{iz}-e^{-iz}}{2i}
\end{equation}



\begin{equation}
\label{eq12: 19Marzo}
\forall a \in \IR-\{ 1 \}: \hspace{0.2cm}
\suma{m=0}{n-1}{a^{r}}= \frac{1-a^{n}}{1-a}.
\end{equation}

\noindent
Tenemos que

\begin{align*}
\langle f_{n,\omega} , g_{n, \omega} \rangle = &
\xi_{n, \omega} \eta_{n, \omega} \left\langle 
\left( cos \left( 2 \pi \omega \frac{m }{n} \right) \right)_{0 \leq m \leq N-1} ,  
\left( cos \left( 2 \pi \omega \frac{m }{n}\right) \right)_{0 \leq m \leq N-1} \right\rangle \\
= & \xi_{n, \omega} \eta_{n, \omega} \suma{m=0}{n-1}{
cos \left(2 \pi \omega \frac{m}{n}\right) sen\left( 2 \pi \omega \frac{m}{n}\right)} \\
= & \frac{\xi_{n, \omega} \eta_{n, \omega}}{2}
\suma{m=0}{n-1}{
\left( sen\left( 4 \pi \omega \frac{m}{n}\right) \right)} \\
= & \frac{\xi_{n, \omega} \eta_{n, \omega}}{4i} \suma{m=0}{n-1}{
\left( e^{4 \pi \omega i m/n} - 
e^{-4 \pi \omega i m/n} \right) } \\
= & \frac{\xi_{n, \omega} \eta_{n, \omega}}{4i} 
\left(
\frac{1-e^{4 \pi \omega i }}{1-e^{4 \pi \omega i /N}} - 
\frac{1-e^{-4 \pi \omega i }}{1-e^{-4 \pi \omega i /N}} 
\right) \\
= & \frac{\xi_{n, \omega} \eta_{n, \omega}}{4i} 
\left(
\frac{e^{2 \pi \omega i }}{e^{2 \pi \omega i/n }}
\frac{e^{-2 \pi \omega i }-e^{2 \pi \omega i }}{e^{-2 \pi \omega i/n }-e^{2 \pi \omega i /N}} - 
\frac{e^{-2 \pi \omega i }}{e^{-2 \pi \omega i/n }}
\frac{e^{2 \pi \omega i }-e^{-2 \pi \omega i }}{e^{2 \pi \omega i/n }-e^{2 \pi \omega i /N}} 
\right) \\
= & 
\frac{\xi_{n, \omega} \eta_{n, \omega}}{4i} 
\left(
e^{2 \pi \omega i \left( 1-1/n \right)}
\frac{sen(2 \pi \omega)}{sen(2 \pi \omega /n)} - 
e^{-2 \pi \omega i \left( 1-1/n \right)}
\frac{sen(2 \pi \omega)}{sen(2 \pi \omega /n)}
\right) 
\\
= & 
\frac{\xi_{n, \omega} \eta_{n, \omega}}{4i} 
\frac{sen(2 \pi \omega)}{sen(2 \pi \omega /n)}
\left(
e^{2 \pi \omega i \left( 1-1/n \right)} - e^{-2 \pi \omega i \left( 1-1/n \right)}
\right) \\
= &
\frac{\xi_{n, \omega} \eta_{n, \omega}}{4i} 
\frac{sen(2 \pi \omega)}{sen(2 \pi \omega /n)}
\left(
2i \cdot  sen \left( 2 \pi \omega  \left( 1- \frac{1}{n} \right) \right)
\right)\\
= & 
\frac{\xi_{n, \omega} \eta_{n, \omega}}{2} 
\frac{sen(2 \pi \omega)}{sen(2 \pi \omega /n)}
sen \left( 2 \pi \omega  \left( 1- \frac{1}{n} \right) \right). \\
\end{align*}


\QEDB
\vspace{0.2cm}


Podemos ya usar las fórmulas establecidas
en la proposición \ref{prop: formulas 20Marzo}
en nuestra situación particular para llegar a que

\[
 \Pi_{P_{\omega}}(x) = 
\frac{
\langle x, f_{n, \omega} \rangle - \langle f_{n, \omega}, g_{n, \omega} \rangle 
\langle x, g_{n, \omega} \rangle
}
{1-|\langle f_{n, \omega}, g_{n, \omega} \rangle |^{2}  }
f_{n, \omega} +
\frac{
\langle x, g_{n, \omega} \rangle - \langle f_{n, \omega}, g_{n, \omega} \rangle 
\langle x, f_{n, \omega} \rangle
}
{1-|\langle f_{n, \omega}, g_{n, \omega} \rangle |^{2}  }
g_{n, \omega},
\]

\noindent
y

\begin{align*}
cos (\measuredangle (x, P_{\omega})) = & || \Pi_{P_{\omega}}(x) ||  \\
= & 
\left(		  
		  \frac{\langle x, f_{n, \omega } \rangle^{2} +  \langle x, g_{n, \omega } \rangle^{2}	
	       -2  \langle x, f_{n, \omega } \rangle \langle x, g_{n, \omega } \rangle \langle f_{n, \omega }, g_{n, \omega } \rangle}{1- \langle f_{n, \omega }, g_{n, \omega } \rangle^{2}}	  
\right) ^{1/2}.
\end{align*}


\begin{defi}
Sean $n \geq 2$ entero, $\omega >0$ con $n \nmid \omega$.
Definimos a la función $\sigma_{n , \omega}: \mathcal{B}_{n}(1) \longrightarrow [0,1]$
como sigue;
\begin{equation}
\label{eq: def sigmas}
\sigma_{n, \omega}(x) : = 
\left(	
\frac{\langle x, f_{n, \omega } \rangle^{2} +  \langle x, g_{n, \omega } \rangle^{2}	
	       -2  \langle x, f_{n, \omega } \rangle \langle x, g_{n, \omega } \rangle 
	       \langle f_{n, \omega }, g_{n, \omega } \rangle}{1- \langle f_{n, \omega }, 
	       g_{n, \omega } \rangle^{2}}	  
 \right) ^{1/2},
\end{equation}

\noindent
donde $f_{n, \omega}$ y $g_{n, \omega}$ son como en 
\eqref{eq5: 19Marzo} y \eqref{eq6: 19Marzo}.
\end{defi}


Según todo lo discutido hasta ahora, dado $x \in \IR^{n}$ unitario
y fijada una frecuencia $\omega >0$ con $n \nmid \omega$,
\begin{itemize}
\item si $\sigma_{n, \omega}(x)$ es ``cercano'' a cero, $\omega$ no
es una frecuencia con la que es razonable aproximar a $x$ (pues $x$ será
cercano a ser ortogonal a toda señal de dimensión $n$ y frecuencia 
$\omega$),  mientras que

\item si $\sigma_{n, \omega}(x)$ es ``cercano'' a uno, también es muy cercano
(hablando en términos de distancia euclídea) a su proyección al espacio
$P_{\omega}$, luego $x$ es muy parecido a tener frecuencia $\omega$.
\end{itemize}


\subsection{Buscando el mejor desfase con cierta frecuencia que se ajuste una señal de dimensión $n$}

Dada $x \in \IR^{n}$ unitaria y $\omega > 0$ una frecuencia fija
con $n \nmid \omega$, 
buscamos el desfase $\phi \in [0,1]$ que mejor se ajusta a $x$.


Puesto que
$\Pi_{P_{\omega}}(x)$ (donde $P_{\omega}$ es como se definió en 
\eqref{eq2: 20Marzo}) es la señal de frecuencia $\omega$ que está a menor
distancia euclidea de $x$, claro que el desfase $\phi$ buscado es de hecho
el número entre cero y uno tal que 
\[
\Pi_{P_{\omega}}(x) = A (cos(2 \pi \omega t -  2 \pi \phi ))_{t \in I_{n}}
\]
para alguna amplitud $A$.


Como los vectores $f_{n, \omega}$ y $g_{n, \omega}$ 
son linealmente independientes (c.f. observación
\ref{obs: f y g son l.i. y de norma uno}),
podemos usar la ecuación \eqref{eq2: 19Marzo}
para escribir a la proyección de $x$ en $P_{\omega}$ como sigue

\begin{equation}
\label{eq3: 20Marzo}
\Pi_{P_{\omega}}(x)= c (cos (2 \pi \omega t))_{t \in I_{n}} + d 
(sin (2 \pi \omega t))_{t \in I_{n}},
\end{equation}
donde

\begin{equation}
\label{eq4: 20Marzo}
c= \frac{
\langle x, f_{n, \omega} \rangle - \langle f_{n, \omega}, g_{n, \omega} \rangle
\langle x, g_{n, \omega} \rangle
}{1-\langle f_{n, \omega}, g_{n, \omega} \rangle^{2}} \xi_{n, \omega}
\end{equation}
y
\begin{equation}
\label{eq5: 20Marzo}
d= \frac{
\langle x, g_{n, \omega} \rangle - \langle f_{n, \omega}, g_{n, \omega} \rangle
\langle x, f_{n, \omega} \rangle
}{1-\langle f_{n, \omega}, g_{n, \omega} \rangle^{2}} \eta_{n, \omega}.
\end{equation}

\noindent 
Nos conviene más reescribir a \eqref{eq3: 20Marzo} como
\begin{equation}
\label{eq6: 20Marzo}
\Pi_{P_{\omega}}(x)= 
\sqrt{c^{2}+d^{2}}
\left(
C (cos (2 \pi \omega t))_{t \in I_{n}} +
D (sen (2 \pi \omega t))_{t \in I_{n}} 
\right),
\end{equation}

\noindent 
donde

\begin{equation}
\label{eq3: 28Marz23}
C:= \frac{c}{\sqrt{c^{2}+d^{2}}} \hspace{0.2cm} \text{y}
\hspace{0.2cm} D:= \frac{d}{\sqrt{c^{2}+d^{2}}},
\end{equation}

pues, como $C^{2} + D^{2}=1$, existe un único
$\phi \in [0,1]$ tal que
\begin{equation}
\label{eq7: 20Marzo}
C= cos(2 \pi \phi), \hspace{0.2cm} 
D= sin(2 \pi \phi).
\end{equation}

\noindent 
Sustituyendo \eqref{eq7: 20Marzo} en \eqref{eq6: 20Marzo},
llegamos a que

\begin{align*}
\Pi_{P_{\omega}}(x) = & 
\sqrt{c^{2}+d^{2}} \left(
cos(2 \pi \phi) \cdot (cos (2 \pi \omega t))_{t \in I_{n}} +
sin(2 \pi \phi) \cdot (sin (2 \pi \omega t))_{t \in I_{n}} 
\right) \\
= & 
\sqrt{c^{2}+d^{2}} 
((cos(2 \pi \phi) \cdot cos (2 \pi \omega t) +
sin(2 \pi \phi) \cdot sin (2 \pi \omega t) )_{t \in I_{n}} \\
= & 
\sqrt{c^{2}+d^{2}} 
(cos(2 \pi \omega t - 2 \pi \phi))_{t \in I_{n}}.
\end{align*}

\noindent
Además, de \eqref{eq7: 20Marzo} y \eqref{eq7: 20Marzo} 
se deduce que
\begin{equation}
\phi =
\begin{cases}
\frac{tan^{-1}(d/c) }{2 \pi}  \hspace{0.4cm}    \text{   si }   d, c > 0,  \\
\frac{tan^{-1}(d/c) + \pi }{2 \pi} \hspace{0.2cm}  \text{si }  d, c < 0,  \\
\frac{tan^{-1}(d/c) + \pi }{2 \pi} \hspace{0.2cm}  \text{si }  d>0,  c < 0,  \\
\frac{tan^{-1}(d/c) + 2\pi }{2 \pi} \hspace{0.2cm}  \text{si }  d>0,  c < 0. 
\end{cases}
\end{equation}

Hemos probado el siguiente
\begin{teo}
Dados $n \geq 2$, $\omega > 0$ con $n \nmid \omega$ y $x \in \IR^{n}$
unitario, 
\begin{equation}
\label{ec: desfase explicito}
\Pi_{P_{\omega}} (x) = \sqrt{c^{2}+d^{2}} \cdot (
cos (2 \pi \omega t - 2 \pi \phi)
)_{t \in I_{n}} \in \IR^{n},
\end{equation}

\noindent
donde $c$ y $d$ son como en \eqref{eq4: 20Marzo} y 
\eqref{eq5: 20Marzo}, respectivamente, y $\phi$ está 
dado por \eqref{ec: desfase explicito}.
\end{teo}


\begin{defi}
Sean $n \geq 2$ es un entero y $\omega \geq 0$ un número no negativo.

Si $n$ divide a $\omega$, definimos a los siguientes vectores de $\IR^{n}$
\begin{equation}
\label{eq0: 27Marzo23}
f_{n, \omega} = \left( 
\frac{1}{\sqrt{n}}, \ldots , \frac{1}{\sqrt{n}}
\right) \in  \IR^{n}
\end{equation}
y
\begin{equation}
\label{eq1: 27Marzo23}
g_{n, \omega} = \left( 
0,  \ldots , 0
\right) \in \IR^{n}.
\end{equation}

En caso contrario, definimos a los vectores

	\begin{equation}
	\label{eq5: 19Marzo}
	f_{n, \omega}= \left( \xi_{n, \omega} cos \left(2 \pi \omega \frac{m }{n} \right) \right)_{0 \leq m \leq N-1}
	\in \IR^{n}
	\end{equation}
y 

	\begin{equation}
	\label{eq6: 19Marzo}
	g_{n, \omega}= \left( \eta_{n, \omega} sen \left(2 \pi \omega \frac{m }{n}\right) \right)_{0 \leq m \leq N-1}
	\in \IR^{n},
	\end{equation}
donde

\begin{equation}
\label{eq7: 19Marzo}
	\xi_{n, \omega}= 
	\sqrt{2} \cdot \left( n + \frac{sen(2 \pi \omega)
	cos(2 \pi \omega \left( 1- \frac{1}{n} \right))}{sen \left(2 \pi 
	\frac{\omega}{n} \right)} \right)^{-\frac{1}{2}} 
\end{equation}
y

	\begin{equation}
	\label{eq8: 19Marzo}
	\eta_{n, \omega}= \sqrt{2} \cdot \left( n - \frac{sen(2 \pi \omega)
	cos(2 \pi \omega \left( 1- \frac{1}{n} \right))}{sen \left(2 \pi 
	\frac{\omega}{n} \right)} \right)^{-\frac{1}{2}}
	\end{equation}

\noindent	
\end{defi}