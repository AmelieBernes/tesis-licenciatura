\textbf{23 Noviembre- Pendientes}
\begin{itemize}

\item \TODO{Cambia las figuras. Fija un color
	(rosa) para TODAS
	las gráficas de señales $x$.}.

\item \textbf{revisar literatura}
\item \textbf{implementar en python los polinomios discretos de Lendre}: Listo!
Ahora debería de intentar usar algunas librerías que encontré
sobre polinomios discretos de Legendre.

\item hacer un mapa sobre cómo se relacionan unos
resultados con otros.
\item \textbf{Dimensión, grado, posición.}
\item Cambia todos los entornos de demostración

\item revisa que siempre digas que la dimensión
del polinomio discreto es $n$.

\item ¿Deberían adjuntar en la tesis una copia de Survey?Porque
cito muchas veces fórmulas de ahí.  

\item \textbf{SÍ es importante poner la dependencia a $n$ de los espacios
$W_{k}$. Es muy confuso si no lo haces así.}

\item cambia el formato de las captions para figuras.

\item revisa que usas $t_{i}$ y no $x_{i}$ para puntos
de la malla.

\item nemu

\item Deberías ver este video:  \url{https://www.youtube.com/watch?v=q5Nsc4LyhdE&ab_channel=SeminarioGAMMA}

\item poner macadores para el final de un ejemplo.

\item Cambiar comas por puntos en el entorno align. Creo que
esto se corrige si pones la cita dentro de un entorno
de texto.

\item marcador al final de ejemplos. \final

\item Cambia los símbolos <,> por langle y rangle.

\item Preguntar al Dr. Gabriel por información.

\item ve cuándo es la primera vez que citas en TFA y da la
formulación dada en rotman.

\item Di que algunos autores dicen que el grado del polinomio
cero no está definido, pero para nosotros es cero.

\item checa que las discretizaciones de los de legendre no son
los que tengo yo.

\end{itemize}

\newpage