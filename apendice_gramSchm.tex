\section{El teorema de Gram-Schmidt}

\begin{teo} \label{Teo:Gram-Schmidt}
\TODO{Mejor empieza a enumerar desde el cero!}
(\textbf{de Gram-Schmidt}): Sean $V$ un espacio vectorial
con producto punto, $S=\{ v_{1}, \ldots ,v_{n} \}$ un
subconjunto linealmente independiente de $V$. 
Defínanse los vectores

\[
\begin{split}
w_{1}:= & v_{1}, \\
w_{k} := & v_{k} - \suma{j=1}{k-1}{
\frac{\langle v_{k} , w_{j} \rangle}{
\langle w_{j} , w_{j} \rangle}  w_{j}},
\hspace{0.2cm} k=2, \ldots ,n;
\end{split}
\]
\noindent
$S'=\{ w_{1}, \ldots , w_{n} \}$ es un subconjunto
ortogonal que genera el mismo espacio que $S$. \TODO{[Friedberg]}
\end{teo}


La esencia del teorema de Gram-Schmidt es el reemplazar una base
$S=\{ v_{1}, \ldots ,v_{n} \}$ de un subespacio 
$U := span(S)$ cualquiera de $V$ 
por una base ortogonal $S'$ para este. \\
Muchas veces nos interesa normalizar a la base $S'$
para así obtener una base ortonormal del espacio; llamaremos
a este el ``proceso de ortonormalización de Gram-Schmidt'',
y lo abreviaremos como ``G-S''. \TODO{AA} \\

La ventaja de la formulación del Teorema
de Gram-Schmidt dada en \ref{Teo:Gram-Schmidt}
es que esta
da explícitamente la forma de calcular a los
vectores $w_{k}$, pero,
para nuestros fines, una formulación que
involucre proyecciones ortogonales sobre espacios
será preferible, pues de este modo la geometría
que hay detrás del proceso puede
vislumbrase mejor. 


\begin{prop} \label{Prop:Gram-Schmidt2}
\textbf{(versión del teorema de Gram-Scmidt en términos de
proyecciones)}
(\TODO{cambia la notación. Usa $v$ y $\xi$}.)
Sean $V$ un espacio vectorial con producto punto,
\[
S=\{ v_{1}, \ldots , v_{n}\}
\] un subconjunto
linealmente independiente de $V$ y
\[
S'=\{ w_{1}, \ldots , w_{n}\}
\] el subconjunto
ortogonal que resulta de aplicar el proceso de
Gram-Schmidt \ref{Teo:Gram-Schmidt} a $S$. \\
Si 
\[
W_{k}=span(v_{1}, \ldots , v_{k}), 
\]
con $k \in \{ 2, \ldots n\}$, entonces
\begin{center}
\framebox{ $w_{k+1}=v_{k+1}- \Pi_{W_{k}}(v_{k+1})$.}
\end{center}
\end{prop}
\noindent
\textbf{Demostración.}
Según el teorema de Gram-Schmidt (\ref{Teo:Gram-Schmidt}),
\[
W_{k}=span(w_{1}, \ldots , w_{k}).
\]
Si mostramos que
\begin{itemize}
	\item el vector $v_{k+1}-w_{k+1}$ es elemento
	del espacio $W_{k}$, y que
	\item $w_{k+1}=v_{k+1}-(v_{k+1}-w_{k+1})$
	es elemento de $W_{k}^{\perp}$,
\end{itemize}
por la unicidad establecida en
el teorema de la proyección ortogonal \ref{Teo:proyOrt}
podremos concluir la igualdad deseada. \\
Lo primero es claro, pues, según las fórmulas
dadas en el teorema \ref{Teo:Gram-Schmidt},
\[
v_{k+1}-w_{k+1}= 
\suma{j=1}{k}{
\left( \frac{v_{k} \cdot w_{j}}{w_{j} \cdot w_{j}} \right) w_{j}} \in W_{k}.
\]
Lo segundo se sigue de observar que
$w_{k+1}$ es ortogonal a los vectores $w_{1}, \ldots , w_{k}$;
como estos conforman una base para $W_{k}$, concluimos
que $w_{k+1} \in W_{k}^{\perp}$.

%\begin{figure}[ht]
%   \centering
%   \incfig{ConFab}
% \end{figure}

\QEDB
\vspace{0.2cm}



\newpage