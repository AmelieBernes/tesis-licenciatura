\section{Definiciones básicas}


\TODO{Yo creo que aquí debes de poner la def. de espacio
vectorial con producto punto y de espacio de Hilbert. Checa
la def. de Erwin y de Kolmogorov. Di además que
consideras que el producto punto es positive definite.} \\



Una de las
grandes ventajas de considerar
bases ortogonales sobre bases cualesquiera para un espacio es que
contamos con una fórmula sencilla
para los coeficientes de la representación lineal de los
elementos del subespacio que generan. Además, información
sobre tales coeficientes \TODO{asdfljka}.
En general. si $S$ es una base cualquiera para
el espacio $V$, los coeficientes no dan información
geométrica sobre el vector \TODO{aquí pon tu ejemplo
con las potencias.} \\
 
\TODO{recuerda poner aqu'los ejemplos canónicos
de espacio HIlbertiano que después cites...}

\TODO{$F \in \{ \IR , \IC \}$}. 
 
\TODO{Di que, dada la def. de producto punto, es natural definir la otrogonalidad y la
norma. Da la definición de base ortonormal (siendo la palabra 'base' usada
en el sentido usual)} 
 
 
\TODO{Por aquí da la nción de subespacio ortogonal, y explica
cómo este suma directamente con el original para formar
a todo el espacio ambiente.} 
 
\begin{defi}
Si $S=\{ w_{1}, \ldots , w_{n} \}$ es una base
ortogonal de $V$ y $v \in V$ cualquiera, a los
números
\[
\frac{\langle v , w_{k} \rangle}{\langle  w_{k}, w_{k} \rangle},
\hspace{1cm} 1 \leq k \leq n
\]
les llamaremos los \textbf{coeficientes de Fourier}
de $v$ respecto a la BO S.
\end{defi}

Observe que si la BO de la definición anterior de hecho
es una BON, entonces los coeficientes de Fourier de un 
vector $v$ respecto a esta no son más que los productos
puntos de $v$ con sus elementos.

\begin{prop}
Los coeficientes de Fourier de un vector $v$
respecto a una BO S son los coeficientes
de la combinación lineal de elementos de $S$ igual a $v$.
\end{prop}
\noindent
\textbf{Demostración.}
En efecto, si $v=\suma{i=1}{n}{c_{i}w_{i}}$,
la bilinealidad del producto punto y la hipótesis 
de ortogonalidad nos llevan a concluir que, 
para todo $k$,
\begin{align*}
\langle  v , w_{k} \rangle &= \langle \suma{i=1}{n}{c_{i}w_{i}}  , w_{k} \rangle \\
& = c_{1} \langle w_{1}  , w_{1} \rangle.
\end{align*}
\QEDB
\vspace{0.2cm}




\begin{ej} \label{ej: espacios con producto punto Rn y ell}
(de espacios con producto punto, que serán usados después en nuestro trabajo)

\textbf{Dimensión finita: $\IR^{n}$}. \TODO{aquí
la def del producto punto.}

\textbf{Dimensión infinita: $\ell^{2}(\IZ)$}. 
\[
\ell^{2}(\IZ) = \{ x=(x_{k})_{k \in \IZ} \in \IR^{\IZ} :
\suma{k \in \IZ}{}{|x_{k}|^{2} < \infty }\}.
 \]
 
Tenemos que hacer algunos comentarios sobre la forma
en que hemos expresado al clásico espacio $\ell^{2}(\IZ)$:

\begin{itemize}
	\item[(I)] $\IZ$ es un conjunto numerable,
	es decir, es posible establecer una biyección
	\[
	\begin{split}
	f: \IN & \longrightarrow \IZ \\
	k & \mapsto z_{k}.
	\end{split}
	\]
	Gracias a una tal $f$ podemos hablar de las sumas
	parciales $\suma{k=1}{n}{|x_{z_{k}}|^{2}}$ de una
	función $x \in \IR^{\IZ}$ y considerar el límite cuando 
	$n \rightarrow \infty $, permitiéndonos hablar de
	la serie $\suma{k \in \IZ}{}{|x_{k}|^{2}}$ que,
	por ser de términos no negativos, o bien converge a un 
	número real $L$ o bien diverge a $\infty$ (i.e. las sumas
	parciales pueden hacerse arbitrariamente grandes).
	En caso de converger o diverger, lo hace absolutamente, 
	luego, no importa cuál sera la biyección 
	$f: \IN \longrightarrow \IZ$ usada para formar
	las sumas parciales, estas siempre convergen a $L$
	o divergen, respectivamente. \TODO{cita el teorema
	correspondiente de Spivak}
	
	\item[(II)] Por lo general, por ``sucesión'' se 
	entiende una función de $\IN$ en $\IR$; sin embargo, lo único 
	que nos importa a nosotros sobre las sucesiones 
	$x: \IN \longrightarrow \IR$ es que
	\begin{enumerate}
		\item tienen un dominio discreto y numerable, y que
		\item el conjunto de sucesiones cuadrado-sumables
		es un $\IR$ -espacio vectorial. 
	\end{enumerate}
	Si en cambio consideramos funciones $x$ de
	$\IZ$ en $\IR$, seguiremos hablando de funciones
	con dominio discreto y numerable, y el conjunto
	de aquellas para las que la serie 
	$\suma{k \in \IZ}{}{|x_{k}|^{2}} $ sea convergente
	es también un espacio vectorial. La ventaja de usar
	a $\IZ$ como dominio es que esto hace de las funciones
	$x$ unas buenas candidatas para representar señales,
	pues no tendremos que preocuparnos por el inicio de
	una señal (dificultad que sí se presenta si por dominio
	se toma al conjunto acotado inferiormente $\IN$).
\end{itemize}



Según la desigualdad de Cauchy-Schwarz, para cualesquiera
$x, y \in \IR^{\IZ}$
\[
\left\lvert
\suma{k \in \IZ}{}{x_{k}y_{k}}
\right\rvert \leq
\left( \suma{k \in \IZ}{}{|x_{k}|^{2}} \right)^{1/2}
\left( \suma{k \in \IZ}{}{|y_{k}|^{2}} \right)^{1/2};
\]

es esta desigualdad la que justifica la buena definición
de la función 
	\[
	\begin{split}
	< , >: \ell^{2}(\IZ) \times \ell^{2}(\IZ) & \longrightarrow \IR \\
	(x,y) & \mapsto \suma{k \in \IZ}{}{x_{k}y_{k}};
	\end{split}
	\]

\TODO{Tengo un problema aquí. La serie involucrada para el producto
punto no es de términos positivos, luego, no aplica el argumento de 
antes para justificar que  ``podemos usar cualquier
orden (biyección)''.}

\noindent no es difícil comprobar que $< , >$ es un producto
punto en el espacio $\ell^{2}(\IZ)$. La norma que
induce está dada por la relación
\[
||x|| = \left[ \suma{k \in \IZ}{}{|x_{k}|^{2}} \right]^{\frac{1}{2}},
\hspace{0.8cm} x \in \ell^{2}(\IZ).
\]

\noindent Se demuestra en [\TODO{Kolmogorov}] que, salvo isometrías,
el espacio con producto interior $(\ell^{2}(\IZ), <, >)$
es el único completo, separable e infinito dimensional. \\


\begin{defi}
Si $\delta_{jk}$ es la delta de Kronecker asociada
a los enteros $j$ y $k$, o sea, el número real definido como
\begin{align*}
\delta_{jk} = \begin{cases}
1 & \text{ si } j=k \\
0 & o.c., 
\end{cases}
\end{align*}
entonces por $\delta_{j}$ denotamos al elemento 
de $\ell^{2}(\IZ)$ definido como

\[
\delta_{j}=(\delta_{jk})_{k \in \IZ}, 
\hspace{0.5cm} j \in \IZ.
\]
\end{defi}

\end{ej}

