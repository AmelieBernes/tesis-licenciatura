\section{Definiciones básicas de espacios de Hilbert}


\TODO{Yo creo que aquí debes de poner la def. de espacio
vectorial con producto punto y de espacio de Hilbert. Checa
la def. de Erwin y de Kolmogorov. Di además que
consideras que el producto punto es positive definite.} \\



Hay que aclarar que $V$ tiene estructura
de espacio métrico, siendo la métrica inducida por
la norma $|| \cdot || $(que, a su vez, fue inducida
por el producto punto $\langle  \cdot , \cdot$ )
dada por la relación 

\[
d(v,w)= ||v-w||, \hspace{1cm} v, w \in V.
\]
Dotamos al producto cartesiano $V \times V $
estructura de espacio métrico al \TODO{...} \\


Una de las
grandes ventajas de considerar
bases ortogonales sobre bases cualesquiera para un espacio es que
contamos con una fórmula sencilla
para los coeficientes de la representación lineal de los
elementos del subespacio que generan. Además, información
sobre tales coeficientes \TODO{asdfljka}.
En general. si $S$ es una base cualquiera para
el espacio $V$, los coeficientes no dan información
geométrica sobre el vector \TODO{aquí pon tu ejemplo
con las potencias.} \\
 
\TODO{recuerda poner aqu'los ejemplos canónicos
de espacio HIlbertiano que después cites...}

\TODO{$F \in \{ \IR , \IC \}$}. 
 
\TODO{Di que, dada la def. de producto punto, es natural definir la otrogonalidad y la
norma. Da la definición de base ortonormal (siendo la palabra 'base' usada
en el sentido usual)} 
 
 
\TODO{Por aquí da la nción de subespacio ortogonal, y explica
cómo este suma directamente con el original para formar
a todo el espacio ambiente.} 
 
\begin{defi}
Si $S=\{ w_{1}, \ldots , w_{n} \}$ es una base
ortogonal de $V$ y $v \in V$ cualquiera, a los
números
\[
\frac{\langle v , w_{k} \rangle}{\langle  w_{k}, w_{k} \rangle},
\hspace{1cm} 1 \leq k \leq n
\]
les llamaremos los \textbf{coeficientes de Fourier}
de $v$ respecto a la BO S.
\end{defi}

Observe que si la BO de la definición anterior de hecho
es una BON, entonces los coeficientes de Fourier de un 
vector $v$ respecto a esta no son más que los productos
puntos de $v$ con sus elementos.

\begin{prop}
Los coeficientes de Fourier de un vector $v$
respecto a una BO S son los coeficientes
de la combinación lineal de elementos de $S$ igual a $v$.
\end{prop}
\noindent
\textbf{Demostración.}
En efecto, si $v=\suma{i=1}{n}{c_{i}w_{i}}$,
la bilinealidad del producto punto y la hipótesis 
de ortogonalidad nos llevan a concluir que, 
para todo $k$,
\begin{align*}
\langle  v , w_{k} \rangle &= \langle \suma{i=1}{n}{c_{i}w_{i}}  , w_{k} \rangle \\
& = c_{1} \langle w_{1}  , w_{1} \rangle.
\end{align*}
\QEDB
\vspace{0.2cm}





\textbf{Dimensión finita: $\IR^{n}$}. \TODO{aquí
la def del producto punto.}

