%Agradecimientos--------------------
\thispagestyle{empty} %para no tener número en esta página
\begin{center}
{\Huge{\textbf{Agradecimientos}}}
\end{center}
\textcolor{ameMorado}{:)}



\newpage

\begin{comment}
{\raggedleft{
\textit{
—Ves, Momo —le decía, por ejemplo—, las cosas son así: a veces tienes ante ti una calle larguísima. Te parece tan terriblemente larga que nunca crees que podrás acabarla. }\\

\textit{ 
Miró un rato en silencio a su alrededor; entonces siguió:} \\
\textit{ 
—Y entonces te empiezas a dar prisa, cada vez más prisa. Cada vez que levantas la vista, ves que la calle no se hace más corta. Y te esfuerzas más todavía, empiezas a tener miedo, al final estás sin aliento. y la calle sigue estando por delante. Así no se debe hacer.}\\
\textit{ 
Pensó durante un rato. Entonces siguió hablando:} \\
\textit{ 
—Nunca se ha de pensar en toda la calle de una vez, ¿entiendes? Sólo hay que pensar en el paso siguiente, en la inspiración siguiente, en la siguiente barrida. Nunca nada más que en el siguiente.} \\
\textit{ 
Volvió a callar y reflexionar, antes de añadir:} \\
\textit{ 
—Entonces es divertido; eso es importante, porque entonces se hace bien la tarea. Y así ha de ser.} \\
\textit{ 
Después de una nueva y larga interrupción, siguió:}\\
\textit{ 
—De repente se da uno cuenta de que, paso a paso, se ha barrido toda la calle. Uno no se da cuenta cómo ha sido, y no se está sin aliento.}\\
\textit{ 
Asintió en silencio y dijo, poniendo punto final:}\\
\textit{ 
—Eso es importante.}
}}
\end{comment}