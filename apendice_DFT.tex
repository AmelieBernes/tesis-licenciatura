\section{La transformada discreta de Fourier y estudios espectrales de señales finitas}

\TODO{aquí una introducción}

\subsection{DFT}

\begin{notacion}
Sea $n \in \IN$ mayor a dos.
Si $0 \leq \omega \leq n-1$, denotaremos por $z_{n, \omega}$
a la $\omega-$ésima raíz $n-$ésima de la unidad, es decir, 

	\begin{equation}
	\label{eq3: 8ab}
	z_{n, \omega} : = exp \left( \frac{2 \pi i }{n} \omega
	\right).
	\end{equation}

\end{notacion}


\begin{prop}
Sea $n \in \IN$. El conjunto

\begin{equation}
\label{eq2: 8ab}
\cali{B}_{n} : = \{
e_{\omega} = \left(
\frac{1}{\sqrt{n}} exp \left(
2 \pi i \omega \frac{m}{n}
\right)
\right)_{0 \leq m \leq n-1}
: \hspace{0.2cm} 0 \leq \omega \leq n-1
 \}
\end{equation}
es una base ortonormal del $\IC-$espacio
vectorial $\IC^{n}$.
\end{prop}

\noindent
\textbf{Demostración.}
\TODO{pendiente.}

\QEDB
\vspace{0.2cm}

Por ser \eqref{eq2: 8ab} una BON de $\IC^{n}$, siempre es
posible expresar a un vector $x = (x_{m})_{0 \leq m \leq n-1} \in \IC^{n}$
como combinación lineal de los elementos de \eqref{eq2: 8ab}
y además los coeficientes están dados por los productos puntos
de $x$ y los elementos de \eqref{eq2: 8ab}, que son

\begin{align*}
\langle x, w_{\omega} \rangle = & 
\frac{1}{\sqrt{n}} \suma{m=0}{n-1}{x_{m} exp \left(
2 \pi i \omega \frac{m}{n}
\right)} \\
= & 
\frac{1}{\sqrt{n}} \suma{m=0}{n-1}{x_{m} 
\left(
exp \left( \frac{2 \pi i }{n} \omega
\right) \right)^{m}} \\
= & A_{x}(z_{n, \omega}),
\end{align*}


\noindent
donde $z_{n, \omega}$ es como en \eqref{eq3: 8ab} y 
$A_{x} = A_{x}(t) \in \IC[t]$ es el polinomio de 
coeficientes complejos definido 
a partir de $x$ como sigue:

	\begin{equation}
		\label{eq4: 8ab}
		A_{x}(t) = \suma{m=0}{n-1}{\frac{x_{m}}{\sqrt{n}} t }\in \IC[t];
	\end{equation}

\noindent
así, \textbf{calcular los coeficientes de $x \in \IC^{n}$ respecto
a la BON $\cali{B}_{n}$ es lo mismo que evaluar al polinomio 
$A_{x}$ de grado $n-1$ definido en \eqref{eq4: 8ab} en todas las raíces
$n-$ésimas de la unidad.} Un algoritmo para evaluar eficientemente
polinomios es pues necesario.

\begin{defi}
Al proceso de calcular los coeficientes de $x$
respecto a $\cali{B}_{n}$
se le conoce como el \textbf{cálculo de la 
transformada discreta de $x$}
\end{defi}

\TODO{cita el FFT}

Dada la motivación de antes, es claro cómo usar la transformada
discreta de 

\textbf{a esto se le llama un análisis espectral.}

\subsection{FFT}

\TODO{Para esto puedes apoyarte mucho en las notas del libro 
de Algorithms. Eso sí lo entendí bien.}




















