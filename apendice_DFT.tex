\section{La transformada discreta de Fourier y estudios espectrales de señales finitas}

\TODO{aquí una introducción}
\TODO{La teoría se basa en el libro de Algoritmos.}

La teoría de esta sección se apoya en la existencia de las raíces
$n-$ésimas de la unidad (garantizada por el teorema fundamental
del álgebra \ref{teo: fundamental del algebra}) 
y propiedades de estas que hacen posible
tener un método eficiente de calcular lo que después llamaremos
la ``transformada discreta'' de un elemento de $\IC^{n}$.

\subsection{Raíces $n-$ésimas de la unidad y propiedades de estas}

\TODO{Habla sobre la exponencial compleja; tal vez pon su definición,
pero di que no entras en detalles (estos pueden consultarse, por ejemplo, en Marsden), sólo citamos unas propiedades de esta función de $\IC$ a $\IC$
que necesitaremos}

\TODO{Tal vez puedes hablar de cómo esta forma involucra al ángulo y
a la norma para representar a un punto, y por qué esto es útil
para hablar de mulitplicación y conjugados! También tienes que definir
la norma en $\IC^{n}$, esto no es trivial y de hecho me estaba dando
problemas.}

\begin{prop}
\label{prop: propiedades exp compleja}
(\textbf{Propiedades de la exponencial compleja}) a
	\begin{itemize}
	\item $exp(z) = 1$ si y sólo si $z= 2K \pi i$ para algún $K \in \IZ$
	\item Para todo $\omega \in \IZ$ y todo $z \in \IC$, $(exp(z))^{\omega} = 			exp(\omega z)$ \TODO{ve si $\omega$ puede de hecho ser cualquier complejo? 			tiene sentido esto?}
	\item Para cualesquiera $z_{1}, z_{2} \in \IC$, 
	$\frac{exp(z_{1})}{exp(z_{2})} = exp(z_{1} - z_{2})$.
	\item Para todo entero $n$ y todo real $b$,
	$(exp(bi))^{m} = exp (mb i)$
	\item \TODO{pon lo de exponencial de la suma.}
	\end{itemize}
\end{prop}


\begin{defi}
\label{defi: raices n esimas de la unidad}
Sea $n \in \IN$. A las $n$ raíces del polinomio
$p_{n}(t)= t^{n}-1$ se les denominará las \textbf{raíces $n-$ésimas de la unidad.}
\end{defi}


Las raíces $n-$ésimas de la unidad son pues los números complejos
tales que, elevados a la potencia $n$, son iguales a 1; según el 
teorema fundamental
del álgebra \ref{teo: fundamental del algebra}, sí hay números complejos
que satisfacen la definición \ref{defi: raices n esimas de la unidad}, y además
son a lo más $n$. Es fácil establecer, como hacemos a continuación, 
fórmulas explícitas \TODO{continua.}

\begin{prop}
Sea $n \in \IN$, $n \geq 2$. Hay exactamente $n$ raíces $n-$ésimas de la
unidad, y estas son los números complejos
 	\begin{equation}
	\label{eq3: 8ab}
	z_{n, \omega} : = exp \left( \frac{2 \pi i }{n} \omega
	\right), \hspace*{0.2cm} \textit{con} 
	\hspace*{0.2cm} \omega \in \{0, 1, \ldots, n-1 \}.
	\end{equation}
	
\end{prop}
\noindent
\textbf{Demostración.}
Por las propiedades expresadas en la proposición
\ref{prop: propiedades exp compleja}, es fácil ver que 
$z_{n,1} :=  exp \left( \frac{2 \pi i }{n} \right)$ es raíz $n-$ésima
de la unidad, pues
\[
(z_{n,1})^{n} = exp(2 \pi i ) = 1.
\]
Además, para todo $\omega \in \{ 0, \cdots , n-1 \}$, el número
\[
z_{n, \omega} : = (z_{n,1})^{\omega} = exp \left( \frac{2 \pi i }{n} \omega \right)
\]
también es es raíz $n-$ésima de la unidad, ya que

\[
(z_{n, \omega})^{n} = ((z_{n,1})^{\omega} )^{n} = 
((z_{n,1})^{n} )^{\omega} = 1^{\omega}=1. 
\]
Note ahora que los $n$ números complejos $z_{n, \omega}$ son todos 
distintos entre sí, pues si $\omega_{1}$ y $\omega_{2}$ son enteros
entre $0$ y $n-1$ tales que $z_{n, \omega_{1}} = z_{n, \omega_{2}}$,
o sea, tales que 
$exp \left( \frac{2 \pi i }{n} \omega_{1} \right) = 
exp \left( \frac{2 \pi i }{n} \omega_{1} \right)$, entonces, según el tercer
punto de la proposición \ref{prop: propiedades exp compleja},
$1 = exp \left( \frac{2 \pi i }{n} (\omega_{1}-\omega_{2}) \right)$, luego, 
según el primer punto de esta misma proposición, $\frac{\omega_{1}-\omega_{2}}{n}$
es entero, o sea, $n$ divide a $\omega_{1}-\omega_{2}$; por el rango de 
$\omega_{1}$ y $\omega_{2}$, esto sólo ocurre si $\omega_{1}-\omega_{2}$ es
cero, o sea, si $\omega_{1}$ y $\omega_{2}$
son iguales.
\QEDB
\vspace{0.2cm}

\TODO{Aquí la figura de siempre:)}



\subsection{DFT}




\begin{prop}
Sea $n \in \IN$. El conjunto

\begin{equation}
\label{eq2: 8ab}
\cali{B}_{n} : = \{
e_{\omega} = \left(
\frac{1}{\sqrt{n}} exp \left(
2 \pi i \omega \frac{m}{n}
\right)
\right)_{0 \leq m \leq n-1}
: \hspace{0.2cm} 0 \leq \omega \leq n-1
 \}
\end{equation}
es una base ortonormal del $\IC-$espacio
vectorial $\IC^{n}$.
\end{prop}

\noindent
\textbf{Demostración.}
Calculemos el producto punto de dos elementos
$e_{\omega_{1}}$ y $e_{\omega_{2}}$ del conjunto \eqref{eq2: 8ab};
si $\omega := \omega_{1}-\omega_{2}$,
\begin{align*}
\langle e_{w_{1}}, e_{w_{2}} \rangle = &
\frac{1}{n}
\suma{m=0}{n-1}{exp \left( 2 \pi i \frac{m}{n} \omega_{1} \right)
\cdot \overline{ exp \left( 2 \pi i \frac{m}{n} \omega_{2} \right) }} \\
= & \frac{1}{n}
\suma{m=0}{n-1}{\left( 2 \pi i \frac{m}{n} (\omega_{1}-\omega_{2}) \right)} \\
= & \frac{1}{n}\suma{m=0}{n-1}{exp\left( 2 \pi i \frac{\omega}{n} m \right)} \\
= & \frac{1}{n}\suma{m=0}{n-1}{exp\left( 2 \pi i \frac{\omega}{n}  \right)^{m}} \\
= & \frac{1}{n}\suma{m=0}{n-1}{(z_{n, \omega})^{m}};
\end{align*}

\noindent
esta última es una suma geométrica. 
\begin{itemize}
	\item Si $\omega_{1} \neq \omega_{2}$, entonces $n$ no puede dividir 
	a $\omega = \omega_{1}-\omega_{2}$ (pues, por el rango en el que se encuentran
	$\omega_{1}$ y $\omega_{2}$, $w \in [-(n-1), n-1]$, y el único múltiplo
	de $n$ en este intervalo es cero), luego, $z_{n, \omega} \neq 1$.
	En este caso se tiene entonces que 
	\[
	\langle e_{w_{1}}, e_{w_{2}} \rangle = 
	\frac{1}{n}\suma{m=0}{n-1}{(z_{n, \omega})^{m}}
	= \frac{1}{n} \cdot \frac{(z_{n, \omega})^{n}-1}{z_{n, \omega}-1}=
	\frac{1}{n} \cdot \frac{1-1}{z_{n, \omega}-1}=0.
	\]
	
	\item SI $\omega_{1} = \omega_{2}$, entonces $\omega = 0$, y
	\[
	\langle e_{w_{1}}, e_{w_{2}} \rangle = 
	\frac{1}{n}\suma{m=0}{n-1}{(z_{n, 0})^{m}}
	= \frac{1}{n}\suma{m=0}{n-1}{1} = \frac{1}{n} \cdot n = 1.
	\]
\end{itemize}

Demostramos así que los elementos de $\cali{B}_{n}$
tienen norma uno (c.f. \TODO{ref ec. norma en $\IC^{n}$}) y que además
son ortogonales
dos a dos, luego, según \TODO{ref}, $\cali{B}_{n}$ es un subconjunto l.i. 
de $\IC^{n}$; como $\IC^{n}$ es un $\IC-$ espacio vectorial de 
dimensión $n$, concluimos lo deseado.
\QEDB
\vspace{0.2cm}

Por ser \eqref{eq2: 8ab} una BON de $\IC^{n}$, siempre es
posible expresar a un vector $x = (x_{m})_{0 \leq m \leq n-1} \in \IC^{n}$
como combinación lineal de los elementos de \eqref{eq2: 8ab}
y además los coeficientes están dados por los productos puntos
de $x$ y los elementos de \eqref{eq2: 8ab}, que son

\begin{align*}
\langle x, w_{\omega} \rangle = & 
\frac{1}{\sqrt{n}} \suma{m=0}{n-1}{x_{m} exp \left(
2 \pi i \omega \frac{m}{n}
\right)} \\
= & 
\frac{1}{\sqrt{n}} \suma{m=0}{n-1}{x_{m} 
\left(
exp \left( \frac{2 \pi i }{n} \omega
\right) \right)^{m}} \\
= & A_{x}(z_{n, \omega}),
\end{align*}


\noindent
donde $z_{n, \omega}$ es como en \eqref{eq3: 8ab} y 
$A_{x} = A_{x}(t) \in \IC[t]$ es el polinomio de 
coeficientes complejos definido 
a partir de $x$ como sigue:

	\begin{equation}
		\label{eq4: 8ab}
		A_{x}(t) = \suma{m=0}{n-1}{\frac{x_{m}}{\sqrt{n}} t }\in \IC[t];
	\end{equation}

\noindent
así, \textbf{calcular los coeficientes de $x \in \IC^{n}$ respecto
a la BON $\cali{B}_{n}$ es lo mismo que evaluar al polinomio 
$A_{x}$ de grado $n-1$ definido en \eqref{eq4: 8ab} en todas las raíces
$n-$ésimas de la unidad.} Un algoritmo para evaluar eficientemente
polinomios es pues necesario.

\begin{defi}
Al proceso de calcular los coeficientes de $x$
respecto a $\cali{B}_{n}$
se le conoce como el \textbf{cálculo de la 
transformada discreta de $x$}
\end{defi}

\TODO{cita el FFT}

Dada la motivación de antes, es claro cómo usar la transformada
discreta de 

\textbf{a esto se le llama un análisis espectral.}

\subsection{FFT}

\TODO{Para esto puedes apoyarte mucho en las notas del libro 
de Algorithms. Eso sí lo entendí bien.}




















