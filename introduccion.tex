\chapter{Introducción}
\label{chapter: introduccion}
Durante todo este trabajo trataremos con señales
finitas
(digamos, de
longitud $n$, con $n$
un entero positivo), y las pensaremos como elementos de $\IR^{n}$.
En la práctica, por lo general
una señal es un conjunto de mediciones tomadas
a intervalos regulares de tiempo; dicha
información perfectamente puede expresarse como 
una $n-$tupla. En lo que sigue, a menos que se diga
lo contrario, usaremos indistintamente los nombres
``vector'' y ``señal''.

Es costumbre empezar
a enumerar las entradas de un vector
de $x \in \IR^{n}$ a partir de uno, pero
para nuestros fines,
como se verá más adelante, es más 
conveniente empezar desde el cero.
Adoptamos pues esta convención.

\marginnote{Preferimos usar el nombre
``afín'' en lugar de ``lineal'', pues en el contexto de funciones
de $\IR$ a $\IR$ el primero hace referencia a funciones
de la forma $f(t) = at+b$, mientras que el segundo se refiere al caso
específico de funciones de la forma $f(t)=at$.} 

\begin{defi}
\label{def: grafica senial}
Por la \textbf{gráfica de una señal} $x=(x_{j})_{j=0}^{n-1}$
de dimensión $n$ entenderemos
 el siguiente subconjunto de $\IR^{2}$:
\[
G_{x} := 
\{ (j, x_{j} ) : \hspace{0.1cm} 0 \leq j \leq n-1
\hspace{0.2cm} \text{entero}\} .
\]
Si $G_{x}$ está contenida en la gráfica de una recta, diremos que la
señal es \textbf{afín}
(en el caso particular en el que
la ecuación de la recta en cuestión sea de la forma $y= h$,
con $h$ una constante,
diremos que la señal es
\textbf{constante}). Si  $G_{x}$ está contenida en 
una parábola de eje vertical, o sea, en una
gráfica de ecuación cartesiana
$y=ax^{2}+ bx +c$, donde $a \neq 0$, diremos 
que la señal es \textbf{cuadrática}.
\end{defi}


\begin{figure}[H]
	\sidecaption{En la figura se pintan las gráficas
	de dos señales de dimensión 30. La gráfica de cruces 
	(resp. la de puntos)
	sugiere la forma de una recta (resp. una parábola).
	Tomamos a las formas de una parábola como cánones
	de curvas (y no, por ejemplo, trozos de circunferencias)
	sólo porque son buenos ejemplos de curvas con una sola 
	convexidad.}
	\includegraphics[scale=0.6]{ejemplo_intro} 
 \end{figure}

\section{Bases ortonormales de espacios vectoriales finito dimensionales}

Sea $F \in \{ \IR, \IC \}$.  
En general, dado un $F-$espacio vectorial $V$ de dimensión finita,
uno siempre cuenta con 
bases para este espacio,
es decir, con subconjuntos de $V$ que son tanto linealmente
independientes (término abreviado como ``l.i.'') como generadores
de todo el espacio. La importancia de 
estas es obvia, ya que una base permite representar
de forma única a cualquier elemento $v \in V$ por medio
de una colección finita de escalares.
Es por eso que, en ocasiones,
usamos el nombre ``sistema de representación''
para referirnos a
\sidenote{El orden de los elementos en una base es importante;
nosotros no 
denotamos esto en la notación para bases, pero suponemos que el orden
en el que se enlistan los elementos es el orden en el que deben usarse.}
una base $\{ v_{k} : \hspace{0.2cm} k \in I \}$
del espacio V.

A lo largo de este trabajo
nos limitaremos a usar
\begin{itemize}
\item los $\IR-$ espacios vectoriales $\IR^{n}$, con $n \geq 1$, y
\item los $\IC-$ espacios vectoriales $\IC^{n}$, con $n \geq 1$, 
\end{itemize}
todos estos de dimensión finita, por lo tanto,
todos ellos espacios de Hilbert 
(c.f. apéndice 
\ref{sec: def basicas Hilbert}).

El tener además en $V$ definido un \textbf{producto punto}
(c.f. apéndice \ref{sec: def basicas Hilbert})
dota de estructura extra al espacio. En particular, 
en este contexto podemos también hablar del ángulo entre vectores
del espacio (c.f.
\ref{angulo entre elementos de un espacio con producto punto})
- en particular, se tiene una noción 
de ortogonalidad de vectores
(c.f. definición \ref{def: ortogonalidad})-
y de la norma de estos
(c.f \ref{prop: La norma inducida por un producto punto}).

\begin{prop}
(\textbf{La ortogonalidad implica independencia lineal})
Si $V$ es un $F-$espacio vectorial finito dimensional y 
$W :=\{ v_{k} : 0\leq k \leq j \}$ es un subconjunto ortogonal de
$V$, 
con ningún $v_{k}$ igual a cero, 
entonces $W$ es linealmente independiente.
\end{prop}
\noindent
\textbf{Demostración.}
Sea
$\{a_{k} \}_{k=0}^{j} \subseteq F$
una colección de escalares para los que se
tenga la siguiente igualdad en $V$.
\[
\suma{k=0}{j}{a_{k}v_{k}} = 0.
\]
Sea un índice $0 \leq k' \leq j$.
Calculando el producto punto de ambos lados
de la ecuación anterior con el vector $v_{k}$
resulta, por la hipótesis de ortogonalidad, la
siguiente ecuación en $F$. 
\[
0 = 
\langle
0, v_{k'}
\rangle =
\left\langle
\suma{k=0}{j}{a_{k}v_{k}}
, v_{k'}
\right\rangle
= 
\suma{k=0}{j}{
a_{k}
\langle
v_{k}
, v_{k'}
\rangle
}
= 
a_{k'} ||v_{k'} ||^{2};
\]
\noindent
puesto que $v_{k'}$ no es cero,
se tiene que 
$||v_{k'} ||^{2} \neq 0$. De esto y la igualdad 
$0 = a_{k'} ||v_{k'} ||^{2}$ se deduce la igualdad
a cero del escalar $a_{k'}$.
\QEDB
\vspace{0.2cm}

En particular, todo subconjunto ortogonal de cardinalidad
$dim(V)$ es una base del espacio $V$; si además todos los elementos
de una tal base tienen norma uno, llamamos a esta una
\textbf{base ortonormal} del espacio $V$. 
\sidenote{Abreviamos ``base ortonormal'' como ``BON''.}


\begin{nota}
\label{nota: sobre la identidad de parseval}
\textbf{(Sobre la identidad de Parseval)}
Digamos que $dim(V)=n$; si 
$\cali{E} := \{e_{k}: \hspace{0.2cm} 0 \leq k \leq n-1 \}$ es una
BON del espacio de Hilbert $V$, es fácil comprobar que, para todo
$x \in V$, se tiene que 

\begin{equation}
\label{eq0: 11ap}
x = \suma{k=0}{n-1}{\langle x , e_{k} \rangle e_{k}}
\end{equation}
y además

\begin{equation}
\label{eq1: parseval}
|| x ||^{2} = \suma{k=0}{n-1}{\langle x , e_{k} \rangle^{2}}
\hspace{0.4cm} \textit{(Identidad de Parseval).}
\end{equation}


\noindent
Así, si se usa a una BON $\cali{E}$ del espacio como sistema
de representación (o sea, si se representa a todo $x$ del espacio 
por medio de sus $n$ coeficientes respecto a $\cali{E}$), se tiene que
\begin{itemize}
	\item tales coeficientes son de hecho los productos punto entre
	$x$ y los elementos de la base $\cali{E}$ (c.f. 
	\eqref{eq0: 11ap}), y
	\item se tiene una relación lineal muy sencilla entre el cuadrado
	de la norma de $x$ y el cuadrado de los coeficientes de la representación
	(c.f. \eqref{eq1: parseval}).
\end{itemize}

Estos dos hechos permiten \textbf{usar la magnitud de un coeficiente en 
\eqref{eq0: 11ap} para valorar la contribución del respectivo 
elemento de la base $\cali{B}$ para sintetizar a $x$},
i.e. para reconstruir a $x$ a partir de sus coeficientes
$\langle x, e_{k} \rangle$ respecto a la base $\cali{E}$.
En efecto, si, por ejemplo, el coeficiente $\langle x, e_{k_{1}} \rangle$
es ``muy cercano a cero'', digamos 
$|\langle x, e_{k_{1}} \rangle| = \epsilon$,  
entonces el vector 
$x_{k_{1}}:= \suma{\substack{ {k=0}, \\  {k \neq k_{1}} }}{n-1}{
\langle x , e_{k} \rangle e_{k}}$
es, en la misma medida, cercano al vector inicial $x$, pues
\[
|| x - x_{k_{1}} || = || \langle x , e_{k_{1}} \rangle e_{k_{1}} ||
= |\langle x , e_{k_{1}} \rangle| = \epsilon.
\]


\noindent
En concordancia con la terminología usual empleada
en el área de procesamiento de señales, 
llamaremos al proceso de calcular los coeficientes
de una señal $x$ respecto a una base ortonormal 
(i.e. el calcular los productos punto de $x$ con
los elementos de la base) \textbf{análisis}, y al proceso
de recuperar a la señal via la identidad 
\ref{eq0: 11ap} \textbf{síntesis}.
\end{nota}

Esta última nota da fuertes razones sobre la utilidad
de las bases ortonormales como sistema de representación
en espacios de Hilbert 
de dimensión finita cuando lo que se quiere es
usar los coeficientes de un vector respecto a la base
para obtener información geométrica del vector
(concretamente, sobre la norma).


\section{Lista de deseos}
Puesto que el espacio subyacente de 
la teoría de este trabajo es el $\IR-$espacio vectorial 
$\IR^{n}$, tenemos la ventaja de contar con
bases (subconjuntos de $\IR^{n}$ linealmente
independientes y generadores de todo el espacio),
que pensamos en este contexto como sistemas
de representación; si 
\begin{equation}
\label{eq0: 13Feb}
\cali{B}=\{v_{k}
 \hspace{0.1cm} |
\hspace{0.1cm} 0 \leq k \leq n-1 \} 
\subseteq \IR^{n}
\end{equation}
es base del espacio, entonces, dado $x \in \IR^{n}$, 
\textbf{existen únicos}
\sidenote{Así, elegida una base $\cali{B}$ de $\IR^{n}$, podemos elegir no 
trabajar directamente con una señal $x \in \IR^{n}$, sino con sus $n$ coeficientes
(que son números reales) respecto a $\cali{B}$.}
escalares
$a_{k} \in \IR$, con $0 \leq k \leq n-1$
tales que

\begin{equation}
\label{eq1: 13Feb}
x = \suma{k=0}{n-1}{a_{k}v_{k}};
\end{equation}
observe entonces que $x$ 
queda completamente determinado por los
números $a_{k}$. Si lo que nos interesa es
la forma de la gráfica de $x$, es natural 
buscar un sistema de representación
para el que sea posible establecer criterios
relativamente sencillos
que relacionen 1. los coeficientes de un $x \in V$
respecto a $\cali{B}$ con
2. la forma de la gráfica de $x$ (interesándonos
particularmente por los patrones descritos en 
\ref{def: grafica senial}). \\
 

Esto motiva la
creación de la siguiente lista de deseos.


\noindent 
\begin{listaObj}
\label{lista de objetivos}
Fijada una dimensión $n$, 
buscamos una base 
\begin{equation}
\label{eq1: 28Nov}
\cali{L}^{n}=\{ \cali{L}^{n,k} \hspace{0.1cm} |
\hspace{0.1cm} 0 \leq k \leq n-1  \} \subseteq \IR^{n} 
\end{equation}
del espacio vectorial
$\IR^{n}$ ``adecuada'' para la representación de señales
$x \in \IR^{n}$,
en el sentido que satisfaga
las siguientes condiciones: 

\begin{enumerate}
\item 
que
sea ortonormal, para poder hacer uso de todas
las bondades que conlleva 
trabajar con tales bases
(c.f. nota \ref{nota: sobre la identidad de parseval}), 
bondades entre las que se encuentran
las que a continuación citamos.
\begin{itemize}
\item 
\textbf{\color{violet}{(Los coeficientes de $x$ respecto
a $\cali{B}$ son muy fáciles de calcular, ...)}}
El poder no sólo dar explícitamente los coeficientes
de un vector respecto a la base, 
	\marginnote{Si $x \in \IR^{n}$ y $\cali{L}^{n}$ como en 
	\eqref{eq1: 28Nov} es BON de $\IR^{n}$, entonces identificamos a
	$x$ con los $n$ números  
	$a_{k}:= \langle x, \cali{L}^{n,k} \rangle$,
	$0 \leq k \leq n-1$.}
sino poder calcular a estos
de forma relativamente sencilla (a saber, vía productos punto,
c.f. nota \ref{nota: sobre la identidad de parseval}); si
\eqref{eq1: 28Nov} es ortonormal, entonces

\begin{equation}
\label{eq2: 13Feb}
x= \suma{k=0}{n-1}{a_{k}\cali{L}^{n,k}},
\hspace{0.2cm} \text{ donde } \hspace{0.2cm}
a_{k}= \langle x , \cali{L}^{n,k} \rangle.
\end{equation}
\item
\textbf{\textcolor{violet}{(... dan información sobre el tamaño (norma) de $x$...)}}
El tener disponible una igualdad de tipo Parseval, igualdad que
relaciona de manera sencilla (de hecho, lineal)
el cuadrado de
la magnitud de una señal $x \in \IR^{n}$ (magnitud dada
gracias al producto punto definido en $\IR^{n}$)
con el cuadrado de las magnitudes de sus coeficientes
respecto a la base;

\[
||x||^{2}= \suma{k=0}{n-1}{|a_{k}|^{2}}.
\]
\marginnote{
		Si 
	$x' := \suma{\substack{ {k=0}, \\  {k \neq k'} }}{n-1}{
	a_{k} \cali{L}^{n,k}}$ y 		
	$|a_{k'}|$ es un número ``cercano a cero'', entonces
	$||x '|| \approx ||x||$. Pensando en la norma de $x$ como
	la cantidad de información que tenía, parafraseamos esto
	como ``quitar coeficientes pequeños no hace que se
	pierda mucha información''	
	.}
Esto es útil al momento
de intentar determinar
(de forma intuitiva) 
la importancia 
de cierto vector de la base para describir a $x$.
También es bueno contar con esta igualdad
al momento de hacer procesos de síntesis,
es decir, de modificación de la señal
via cambios en sus coeficientes respecto
a un sistema de representación; si un coeficiente
$a_{k}$ es pequeño en magnitud, retirando el sumando
$a_{k} \cali{L}^{n,k}$ de \eqref{eq2: 13Feb}, 
estamos seguros de obtener
un vector $x'$ similar al vector original $x$
en magnitud.
\end{itemize}
\item \textbf{
\textcolor{violet}{(...y sobre la forma de la
gráfica de $x$.)}} El que,
dada la expresión \eqref{eq2: 13Feb},
sea posible establecer
una relación sencilla entre \textbf{los coeficientes }
$a_{k}$
(en base a su tamaño y signo) y
\textbf{la forma de la gráfica de la señal}.
En particular, nos gustaría encontrar condiciones
sencillas,
necesarias y suficientes 
en términos de los coeficientes $a_{k}$ para determinar
si la gráfica de $x$ es afín o cuadrática.
Queremos pues que los coeficientes $a_{k}$
\textit{reflejen atributos geométricos de $x$};
de esta forma, 
\textbf{con herramientas del álgebra lineal lograremos cuantificar
la propiedad geométrica intuitiva de ``parecerse a''
una recta o una parábola.} También nos gustaría poder ampliar
el rango de respuestas, y no sólo poder tratar con eventos
binarios del tipo ``$x$ es o no es afín'', sino
también tener alguna forma de medir qué tanto se aproxima
$x$ a la propiedad de ser afín.
\end{enumerate} 

\end{listaObj}


En general, una base de un 
$F-$espacio vectorial $V$ de dimensión finita $n$
es utilizada para, en vez de tratar 
directamente con los elementos abstractos
del espacio, usar $n$ elementos de $F$
para representar unívocamente a cada objeto. En nuestro caso,
trabajamos ya con $n-$tuplas de números reales,
lo que en realidad equivale a representar elementos
del espacio con la base canónica de $\IR^{n}$
(c.f. \cite{friedberg}, p.43), pero,
como explicamos en la lista de deseos
\ref{lista de objetivos},
nosotros estamos buscando una base $\cali{L}^{n}$ 
de $\IR^{n}$ tal que
sea posible, a partir de representaciones
de una señal $n-$dimensional $x$
respecto a esta, dar
criterios para determinar si $x$ es afín, lineal o cuadrática.
Como se menciona, también nos gustaría poder dar medidas
de qué tanto tiende 
la gráfica de
$x$ a parecerse a alguna de estas formas.



\section{Polinomios de Legendre de variable continua}
Puesto que son las
formas básicas de recta y parábola las que nos interesa
capturar,
consideramos a las gráficas de las potencias básicas
\begin{equation}
\label{eq0: 8En}
f_{i}(t)= t^{i}, \hspace{0.4cm}
-1 \leq t \leq 1, \hspace{0.2cm}
i \in \overline{\IN};
\end{equation}
restringiendo su dominio al intervalo $[-1,1]$,
las funciones resultantes son 
elementos del 
espacio de Hilbert $L^{2}[-1,1]$. \sidenote{
Puede consultar la definición del espacio de Hilbert
$L^{2}[a,b]$, con $a<b$ cualesquiera números reales,
en \cite{Kreyszig}, p. 132.
}


\begin{notacion}
\label{notacion: 17ap}
Sea $k \geq 0$ entero.
Denotaremos por 
\begin{itemize}
	\item $\IR_{-1,1}[t]$ al subespacio de $L^{2}[-1,1]$ que consta de los
	polinomios restringidos al intervalo $[-1,1]$,
	
	\item $\IR_{k,-1,1}[t]$ al subespacio de $\IR_{-1,1}[t]$
	que consta de los polinomios restringidos a $[-1,1]$ y grado a lo
	más $k$.
\end{itemize}
\end{notacion}


Dos observaciones se siguen de inmediato:
\begin{itemize}
	\item si $k_{1} \leq k_{2}$ se tiene trivialmente la contención
$\IR_{k_{1}, -1,1} \subseteq \IR_{k_{2}, -1,1}$, o sea, la familia
$\{\IR_{k, -1,1}\}_{k \in \overline{\IN}}$ de subespacios
de $L^{2}[-1,1]$ está anidada; además, 

	\item la unión de la familia $\{\IR_{k, -1,1}\}_{k \in \overline{\IN}}$
	es todo $\IR_{-1,1}[t]$.
\end{itemize}


Se comprueba trivialmente que, si los polinomios
$f_{i}$ son como en \eqref{eq0: 8En}, entonces,
para toda $k \geq 0$, $\{f_{i}(t): \hspace{0.2cm} 0 \leq i \leq k \}$
es una base del espacio $\IR_{k,-1,1}[t]$, luego, con el 
método de Gram-Schmidt
(c.f. \ref{ap: gram schmidt}) esta puede ortonormalizarse; haciendo esto
para toda $k$, se obtiene la colección
de polinomios mutuamente ortogonales


\begin{equation}
\label{eq1: 8En}
\{ P_{k}(t)\in \IR_{k, -1,1}\}_{k \in \overline{\IN}}, \hspace{0.4cm}
-1 \leq t \leq 1, \hspace{0.2cm}
 k \in \overline{\IN},
\end{equation}
estando cada polinomio $P_{k}$ determinado unívocamente
por las $k$ condiciones de ortogonalidad
\[
\forall \hspace{0.1cm} 0 \leq m \leq k-1: \hspace{0.1cm}
\int_{-1}^{1}{P_{k}(t)P_{m}(t)} dt = 0
\]
y por la condición de normalización 
\[
\int_{-1}^{1}{P_{k}(t)^{2}} dt = 1.
\]
A los polinomios \eqref{eq1: 8En}
se les conoce como \textbf{polinomios de Legendre}
\sidenote{Muchas veces
esta condición de normalización cambia; algunos 
prefieren tomar como condiciones de normalización a 
\[
\int_{-1}^{1}{P_{k}(t)P_{m}(t)} dt = \frac{2 \delta_{km}}{2n+1}.
\]
Por eso las fórmulas que usualmente uno encuentra para
los polinomios de Legendre (por ejemplo \cite{leg})
son distintas a las presentadas aquí.} (c.f. \cite{friedberg} p.346
y \cite{DSML} p.390).

Se calcula fácilmente que los primeros cuatro 
polinomios de Legendre son

\[
P_{0}(t) = 1/\sqrt{2},
\]

\[
P_{1}(t) = \sqrt{\frac{3}{2}}t,
\]

\[
P_{2}(t) = \sqrt{\frac{5}{8}}\left( 3t^{2}-1 \right),
\]
y
\[
P_{3}(t) = \frac{5}{2} \sqrt{\frac{7}{2}}\left( t^{3}- \frac{3}{5}t\right).
\]

En esta discusión tenemos ya involucrada una base ortogonal
(en el espacio de funciones $L^{2}[-1,1]$);
si planeamos, de alguna forma, usar a estos polinomios
para el estudio morfológico de señales, por la naturaleza
discreta de estos últimos objetos, será
inevitable realizar, en algún punto 
del argumento, un proceso de discretización
(por ejemplo, 
via una discretización
puntual, c.f definición \ref{def: operador de discretizacion puntual}), es decir,
pasar del ``contexto continuo'' en el que se han
definido los polinomios de Legendre a uno discreto. \\

Para partir de
objetos en el espacio ambiente
$\IR^{n}$ de nuestro interés,
también tiene sentido primero realizar
un proceso de discretización (por ejemplo, 
puntual) y luego realizar
un proceso de ortogonalización. La figura
de abajo
ilustra, para el caso $n=4$, estos dos caminos,
o sea, el que consiste en primero ortonormalizar
polinomios en el espacio $L^{2}[-1,1]$ y luego discretizarlos,
y en el que consiste en primero discretizar polinomios
para obtener vectores de $\IR^{n}$ que pueden ortonormalizarse
en este espacio. Observe que sólo usando este segundo camino 
obtenemos con seguridad un subconjunto ortonormal de $\IR^{n}$,
pero no se puede asegurar lo mismo usando el primer camino
(c.f. \cite{stockel}).
Puesto que lo
que interesa es tener condiciones de ortogonalidad
en el espacio $\IR^{n}$, es obvio que 
el camino a seguir debe ser pues uno parecido al segundo. \\


Escogida pues la opción de primero 
discretizar polinomios para posteriormente
ortonormalizar, queda todavía abierta
la decisión de qué método de discretización usar. \\

Elegimos empezar
con discretizaciones puntuales de polinomios.
En base a estos objetos lograremos definir
una base ortonormal de $\IR^{n}$ que cumple satisfactoriamente
los requisitos explicados en la lista de deseos
\ref{lista de objetivos}, y a la que llamaremos
la \textbf{base de Legendre discreta de dimensión $n$}. 
A sus elementos los llamaremos los \textbf{polinomios de Legendre
discretos de dimensión $n$}.\\

Un segundo punto de vista, basado
en discretizaciones por promedios integrales,
se presentará después en
\ref{Construcción de Ln en base a discretizaciones con sumas integrales}; 
como demostraremos a lo largo del capítulo \ref{cap 2}, 
ambas alternativas
de discretización, igual de naturales la una que la otra, nos
conducen a la misma base ortonormal de $\IR^{n}$,
que, como se ilustra en el capítulo 
\ref{chap: Analisis de señales en base a coeficientes respecto a las BLDs},
resulta ser un sistema de representación
realmente útil para el estudio morfológico de señales finitas.


\begin{figure}[H]
\centering\captionsetup{format = hang}
	\begin{measuredfigure}
		\label{fig: ortogonalizacion, discretizacion}
		\includegraphics[scale=1.3]{discr_ortog} 
		\caption{Como se aprecia, las operaciones de
		ortogonalización y discretización no conmutan.}
 	\end{measuredfigure}
 \end{figure}


\section{Sobre los polinomios discretos de Legendre en la literatura}
\label{sec: Sobre los polinomios discretos de Legendre en la literatura}

Después de revisar la literatura, encontramos que
las funciones que llamamos ``polinomios discretos de Legendre
de dimensión $n$'', y que denotamos
por $\cali{L}^{n,k}$ con $0 \leq k \leq n-1$, ya se han estudiado.

En \cite{nikiforov} se desarrolla la teoría 
de los polinomios ortogonales de variable discreta
en un contexto
más general, y se refieren a los que resultan ser
los polinomios discretos $\cali{L}^{n,k}$ que definimos
en este trabajo
como``polinomios de Chebyshev de variable discreta'', 
siendo estos un caso particular
de los polinomios de Hahn 
de variable discreta.
Se menciona que estos polinomios
fueron introducidos
por P.L. Chebyschev en 1858,
y redescubiertos por
Gram en 1915
(véase \cite{tcheb}, \cite{Neuman}).

Para una discusión histórica muy completa
sobre el desarrollo de los polinomios
ortogonales discretos,
véase \cite{roy}. Como ahí se explica, esta
noción fue desarrollada por Chebyshev a partir
de estudios de probabilidad y de ajuste de datos
por mínimos cuadrados. Explicamos con más detalle las ideas
contenidas en el artículo \cite{george}
sobre esta última aplicación en
la subsección \eqref{subsec: ajuste de datos}. \\

Otras fuentes más modernas no usan el nombre de 
``polinomios de Chebyshev de variable discreta'', sino que
adoptan el nombre de ``polinomios ortogonales discretos de Legendre''
(``Discrete Legendre
Orthogonal Polynomials'' en inglés, por lo tanto abreviados
como ``DLOPs''), pues, como se nota en 
\cite{Neuman}, los coeficientes de estos polinomios son,
exceptuando cambios de signo, 
los coeficientes de los polinomios de Legendre de variable
continua definidos en $[0,1]$.
La referencia \cite{Neuman} es un compendio de fórmulas
-que serán utilizadas más adelante en nuestro trabajo-
de estos polinomios ampliamente citado. \\


Los polinomios ortogonales de variable discreta
ya han sido estudiados y utilizados en la práctica,
no sólo para optimizar el proceso de ajuste de
datos por mínimos cuadrados 
(c.f. \cite{george}), sino más recientemente como 
una herramienta en el área de procesamiento de señales.
Por ejemplo, en 
\cite{colomer}
se utilizan los polinomios
discretos de Legendre
(bajo el nombre de
``Discrete Legendre Transform'', abreviado DLT) 
para la compresión de datos de electrocardiogramas
y, más recientemente, se han
relacionados los polinomios discretos de Legendre 
con un tipo de redes neuronales, a
saber, las ``Legendre Delay Networks''
(c.f. \cite{stockel}).
En \cite{furlong} se continúa esta línea y 
se introduce el algoritmo de aprendizaje ``Learned Legendre Predictor'' (LLP),
basado en redes neuronales, para la predicción de datos en tiempo real. \\

Por ser objetos tan usados en las aplicaciones, no es de
extrañarse que se hayan estudiado ya
algoritmos para el cálculo eficiente
de los polinomios discretos de Legendre
(c.f. \cite{Neuman}, \cite{abur}, \cite{abur2}, \cite{dris}, \cite{mukun}) \\


Sin embargo,
no encontramos en la literatura
un uso de los polinomios discretos
de Legendre para el estudio morfológico de señales finitas
como el que proponemos (después de construir cuidadosamente
estos objetos en el capítulo \ref{cap 2})
en el capítulo 
\ref{chap: Analisis de señales en base a coeficientes respecto a las BLDs}.
Dedicamos la segunda parte de esta tesis, con una metodología 
definida por nosotros, a realizar un análisis espectral de las gráficas
de estos polinomios discretos.
