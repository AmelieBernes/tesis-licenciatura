%%%% Mis paquetes %%%%
\usepackage[utf8]{inputenc}
\usepackage{comment} %para comentar párrafos largos
						

				
\usepackage[most]{tcolorbox} %para encerrar texto en cajas de colores
\usepackage[spanish]{babel}
\decimalpoint
\usepackage{mathtools}% http://ctan.org/pkg/mathtool
\usepackage{amsthm, amsmath, bm} 

\usepackage{subcaption, threeparttable}
\usepackage{graphicx} 
\graphicspath{{./imagenes}}
\DeclareCaptionFormat{custom}
{
    \textbf{#1#2}\textit{\small #3}
}
\captionsetup{format=custom}
\usepackage[font=small,labelfont=bf]{caption} %para que los nombres 'Figura x' estén en negritas.

%Para rotar imágenes:
\usepackage{wrapfig}
\usepackage{lscape}
\usepackage{rotating}
\usepackage{epstopdf}

\usepackage{tabularx} %sí lo uso?

\allowdisplaybreaks %Para que environments align muy largos
%sean puestos en dos páginas, sin necesidad de hacer breaks para que queden
%en una sola.

%Gilles C. paquetes para imagenes con Inkscape
%\usepackage{import}
%\usepackage{xifthen}
%\usepackage{pdfpages}
%\usepackage{transparent}

%\newcommand{\incfig}[1]{%
%    \def\svgwidth{\columnwidth}
%    \import{./imagenes}{#1.pdf_tex}}


\usepackage{marginnote} %Para escribir en los márgenes
%IMPORTANTE: sidenotes no son aceptadas dentro de ambientes de thmbox, pero marginnotes sí!

\usepackage{xcolor, color, soul}
\DeclareRobustCommand{\hlAME}[2]{{\sethlcolor{#1}\hl{#2}}}

%mis colores
\definecolor{ameMorado}{HTML}{816780}
\definecolor{ameDorado}{HTML}{e4bc8c}
\definecolor{ameRosa}{HTML}{FF69B4}
\definecolor{ameAzul}{HTML}{41BBE0}

\definecolor{ameVerde}{HTML}{9acd32}
\definecolor{ameCyanO}{HTML}{008b8b}
\definecolor{ameGris}{HTML}{778899}

\usepackage{tikz}
\usepackage[toc,page]{appendix}  
\usepackage{enumerate}
\usepackage{lscape} %Para tablas horizontales.
\usepackage{verbatim}

\usepackage{url}

%Bilbiografía
\usepackage[backend=bibtex,style=alphabetic]{biblatex}
\addbibresource{Referencias.bib}


%\usepackage{natbib} %no me funciona.
%\usepackage{chngcntr}
%\counterwithin{equation}{section}


%Para escribir pseudocódigo
\usepackage{algorithm}
\usepackage{algorithmic}
\renewcommand{\algorithmicrequire}{\textbf{Input:}}
\renewcommand{\algorithmicensure}{\textbf{Output:}}
\newcommand{\code}[1]{\texttt{#1}}
%\swapnumbers %Para tener un solo contador para TODOS los 'equation' environment.


\usepackage{amssymb}
\usepackage{mdframed} %Símbolos en cajas
\usepackage{amsmath} %it is best to load this package if doing any serious mathematical layout with LaTeX
\usepackage{cases}
\usepackage{bm} %Bold math text
\usepackage{marvosym}
\usepackage{fancyref}
\usepackage{graphicx}
\usepackage{float}
\usepackage{graphicx}
\usepackage{titlesec}
\usepackage{imakeidx} 
\usepackage{scalerel}

%Paquetes usados en estilo.tex
\usepackage{geometry}
\usepackage{sidenotes} 
\usepackage[font=footnotesize,format=plain,labelfont={bf,sf},textfont={it},width=10pt]{caption}
% Captions at the side of the page
%\usepackage[wide]{sidecap}
\usepackage{morefloats}
\usepackage{marginfix}

\usepackage{hyperref}
\usepackage{subfiles} % Best loaded last in the preamble

%%%% Mis macros %%%%

%% Secciones:
%Pongo a todas el contador de 'teo'

%%%%%%%%%%%%%%%%%%%%%%%%%%%%%%%%%%%%%%%%%%%%%%%%%%%%%%%%%%%%%%%%%%%%%%%%%%%%
%Fuentes: %https://tex.stackexchange.com/questions/513107/what-is-this-fancy-theorem-environment 
%https://tex.stackexchange.com/questions/521597/how-to-spare-selected-theorems-from-thmbox-styles
%NOTA: Dentro de un ambiente de teorema que use el paquete thmbox, NO puedes usar sidenotes, footnotes ni imágenes.

%\usepackage{amsthm} %Creo que ya lo definí antes.


\let\newemptytheorem\newtheorem
\usepackage{thmbox}

\newemptytheorem{ejemplo}{Ejemplo}


			\newtheorem[M]{teo}{Teorema}[section]
			%Pongo a todas el contador de 'teo'
			\newtheorem[M]{listaObj}[teo]{Lista de deseos}
			\newtheorem[M]{preg}[teo]{Pregunta}
			\newtheorem[M]{lema}[teo]{Lema}
			\newtheorem[M]{hip}[teo]{Hipótesis}
			\newtheorem[M]{prop}[teo]{Proposición}
			\newtheorem[M]{obs}[teo]{Observación}
			\newtheorem[M]{cor}[teo]{Corolario}
			\newtheorem[M]{defi}[teo]{Definición}
			\newtheorem[M]{notacion}[teo]{Notación}
			\newtheorem[M]{nota}[teo]{Nota}
			\newtheorem{pregunta}{Pregunta}



%Ambientes de teorema antiguos.
%			\newtheorem{teo}{Teorema}[section]
%			%Pongo a todas el contador de 'teo'
%			\newtheorem{listaObj}[teo]{Lista de objetivos}
%			\newtheorem{lema}[teo]{Lema}
%			\newtheorem{prop}[teo]{Proposición}
%			\newtheorem{obs}[teo]{Observación}
%			\newtheorem{cor}[teo]{Corolario}
%			\newtheorem{defi}[teo]{Definición}
%			\newtheorem{notacion}[teo]{Notación}
%			\newtheorem{nota}[teo]{Nota}

%%%%%%%%%%%%%%%%%%%%%%%%%%%%%%%%%%%%%%%%%%%%%%%%%%%%%%%%%%%%%%%%%%%%%%%%%%%%





			%% Símbolos matemáticos
			
			\newcommand{\demostracion}{\textbf{Demostración.} \hspace*{0.1cm}}
			\newcommand*{\QEDA}{\null\nobreak\hfill\ensuremath{\blacksquare}}%
			\newcommand*{\QEDB}{\null\nobreak\hfill\ensuremath{\square}}%
			
			\newcommand{\TODO}[1]{\textcolor{purple}{#1}}
			\newcommand{\bien}[1]{\textcolor{blue}{Revisado.}}
			
			\newcommand*{\final}{\null\nobreak\hfill\ensuremath{\diamond}}
			\newcommand{\IR}{\mathbb{R}}
			\newcommand{\IC}{\mathbb{C}}
			\newcommand{\IN}{\mathbb{N}}
			\newcommand{\IP}{\mathbb{P}}
			\newcommand{\IZ}{\mathbb{Z}}

			\newcommand{\La}{\Lambda}		
			\newcommand{\Tau}{\mathrm{T}}		
			
			\newcommand{\suma}[3]{\sum\limits_{#1}^{#2}#3} %Sumas y series
			\newcommand{\union}[3]{\bigcup\limits_{#1}^{#2}{#3}} %uniones
			\newcommand{\producto}[3]{\prod_{#1}^{#2}{#3}} %productos
			\newcommand{\limite}[2]{\lim\limits_{#1}{#2}} %límites
			\newcommand{\limsu}[2]{\lim\limits_{#1 \rightarrow \infty }#2_{#1}}
			%para límites de sucesiones
			\newcommand{\Om}{\Omega}
			\newcommand{\cali}[1]{\mathcal{#1}} %Letras caligráficas
			\newcommand{\cont}[2]{$\mathcal{C} [#1, #2]$}
			\newcommand{\integ}[3]{\int_{#1}^{#2}{#3}}
			\newcommand{\ldos}{\mathit{l}^{2}}


			\DeclareMathOperator*{\ameboxplus}{{\boxplus}}



%macros Moisés
\newcommand{\aplica}[5]{\text{${\begin{array}{crcl} #1: & #2 & \longrightarrow & #3 \\ \, & #4 & \longmapsto & #5 \end{array}}$}} %Notación para la definición de una aplicación.
\newcommand{\demo}[1]{\textbf{Demostración:} #1 \nopagebreak[4] $\square$ \\}
			
			
%Source: https://tex.stackexchange.com/questions/68010/how-to-create-multilevel-colored-boxes-using-tcolorbox-any-other-package
\makeatletter
\newcommand{\DrawLine}{%
  \begin{tikzpicture}
  \path[use as bounding box] (0,0) -- (\linewidth,0);
  \draw[color=black,dashed,dash phase=2pt]
        (0-\kvtcb@leftlower-\kvtcb@boxsep,0)--
        (\linewidth+\kvtcb@rightlower+\kvtcb@boxsep,0);
  \end{tikzpicture}%
  }
\makeatother




%%%% Datos generales del archivo %%%%
\title{Tesis}
\author{Amélie Bernès}
