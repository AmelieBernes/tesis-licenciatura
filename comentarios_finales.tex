\subsection{Fórmulas explícitas de la base $\cali{L}^{n}$ para dimensiones $2 \leq n \leq 6$} \label{subsect:Formulas explicitas}
\label{formulas explicitas para Ln con n de 2 hasta 6}

A continuación tabulamos los
elementos de la base de Legendre
discreta $\cali{L}^{n}$ para dimensiones
$2 \leq n \leq 6$.
Más adelante daremos una fórmula general para calcular
las BDL, por lo que en realidad no será necesario tabular estos valores.
De todas formas, calculamos con la definición
\TODO{ref} a las bases de Legendre discretas hasta dimensión $6$, pues
se usarán los valores para hacer unos ejemplos y,
más adelante, hacer hipótesis sobre las simetrías en las entradas
de los PDL (c.f. \TODO{ref}.)


%Tabla para n=2,3,4
\begin{center}
\begin{tabular}{ c c c c c c }
k $\backslash$ n & 2 & 3 & 4   \\ 
\hline

0 & $\left(
\frac{1}{\sqrt{2}}, \frac{1}{\sqrt{2}}\right)$ & 
$\left(\frac{1}{\sqrt{3}}, \frac{1}{\sqrt{3}}, \frac{1}{\sqrt{3}} \right)$ & 
$\left(\frac{1}{2}, \frac{1}{2}, \frac{1}{2}, \frac{1}{2} \right)$ \\ 
1 & $\left(-\frac{1}{\sqrt{2}}, \frac{1}{\sqrt{2}}\right)$ & 
$\left(-\frac{1}{\sqrt{2}}, 0, \frac{1}{\sqrt{2}} \right) $ & 
$\left(-\frac{3}{2\sqrt{5}}, -\frac{1}{2\sqrt{5}}, \frac{1}{2\sqrt{5}}, \frac{3}{2\sqrt{5}} \right)$  \\ 
2 & $---$ & $\left(\frac{1}{\sqrt{6}}, -\sqrt{\frac{2}{3}}, \frac{1}{\sqrt{6}} \right) $ & 
$\left(\frac{1}{2}, -\frac{1}{2}, -\frac{1}{2}, \frac{1}{2} \right)$ \\ 
3 & $---$ & $---$ & 
$\left(-\frac{1}{2\sqrt{5}}, \frac{3}{2\sqrt{5}}, -\frac{3}{2\sqrt{5}}, \frac{1}{2\sqrt{5}} \right)$  \\ 
\end{tabular}
\end{center} 
 
%Tabla para n=5,6
\begin{center}
\begin{tabular}{ c c c c c c }
k $\backslash$ n & 5 & 6  \\ 
\hline
0 & 
$\left(\frac{1}{\sqrt{5}}, \frac{1}{\sqrt{5}}, \frac{1}{\sqrt{5}},
\frac{1}{\sqrt{5}}, \frac{1}{\sqrt{5}} \right)$ 
& $\left(\frac{1}{\sqrt{6}}, \frac{1}{\sqrt{6}}, \frac{1}{\sqrt{6}},
\frac{1}{\sqrt{6}}, \frac{1}{\sqrt{6}}, \frac{1}{\sqrt{6}} \right)$ \\ 
1 &  
$\left(-\sqrt{\frac{2}{5}}, -\frac{1}{\sqrt{10}}, 0,
\frac{1}{\sqrt{10}}, \sqrt{\frac{2}{5}} \right)$  & 
$\left(-\sqrt{\frac{5}{14}}, -\frac{3}{\sqrt{70}}, -\frac{1}{\sqrt{70}},
\frac{1}{\sqrt{70}}, \frac{3}{\sqrt{70}}, \sqrt{\frac{5}{14}} \right)$ \\ 
2 & 
$\left(\sqrt{\frac{2}{7}}, -\frac{1}{\sqrt{14}}, -\sqrt{\frac{2}{7}},
-\frac{1}{\sqrt{14}}, \sqrt{\frac{2}{7}} \right)$ 
& $\left(\frac{5}{2\sqrt{21}}, -\frac{1}{2\sqrt{21}}, -\frac{2}{\sqrt{21}},
-\frac{2}{\sqrt{21}}, -\frac{1}{2\sqrt{21}}, \frac{5}{2\sqrt{21}} \right)$ \\ 
3 & 
$\left(-\frac{1}{\sqrt{10}}, \sqrt{\frac{2}{5}}, 0,
-\sqrt{\frac{2}{5}}, \frac{1}{\sqrt{10}} \right)$ &
$\left(-\frac{\sqrt{5}}{6}, \frac{7}{6\sqrt{5}}, \frac{2}{3\sqrt{5}},
-\frac{2}{3\sqrt{5}}, -\frac{7}{6\sqrt{5}}, \frac{\sqrt{5}}{6} \right)$ \\ 
4 & $\left(\frac{1}{\sqrt{70}}, -\frac{2\sqrt{2}}{\sqrt{35}}, 
\frac{3\sqrt{2}}{\sqrt{35}},
-\frac{2\sqrt{2}}{\sqrt{35}}, \frac{1}{\sqrt{70}} \right) $ & 
$\left(\frac{1}{2\sqrt{7}}, -\frac{3}{2\sqrt{7}}, \frac{1}{\sqrt{7}},
\frac{1}{\sqrt{7}}, -\frac{3}{2\sqrt{7}}, \frac{1}{2\sqrt{7}} \right)$ \\ 
5 & $---$ & 
$\left(-\frac{1}{6\sqrt{7}}, \frac{5}{6\sqrt{7}}, -\frac{5}{3\sqrt{7}},
\frac{5}{3\sqrt{7}}, -\frac{5}{6\sqrt{7}}, \frac{1}{6\sqrt{7}} \right)$ 
\end{tabular}
\end{center}