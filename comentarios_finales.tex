\section{Comentarios finales sobre la construcción} 
\label{sec: commentarios finales}
Nuestro primer intento de 
construcción de lo que ahora llamamos 
``base de Legendre discreta'' de dimensión $n$,
y que denotamos por
$\cali{L}^{n}$, se basó en 
un caso particular del punto de vista
dado en la subsección 
\ref{Construcción de Ln en base a discretizaciones con sumas integrales}
(discretizando con el operador $\Delta_{n,a,b}$
a las potencias $f_{k}$
con $0 \leq k \leq n-1$
en el intervalo $[-1,1]$); es la clara analogía 
entre nuestra construcción discreta y la continua
de los polinomios discretos de Legendre (c.f.
definición \ref{def: base de Legendre discreta}) una de las
razones por la que acuñamos el nombre para
la BON $\cali{L}^{n}$. \\ 

La determinación de expresiones analíticas para los
elementos de $\cali{L}^{n}$ se hará después de la sección
de revisión de literatura, pues para la tarea haremos uso de las 
fórmulas ofrecidas en uno de los artículos consultados. \\

Para terminar, mencionamos que fue en dicho artículo
(el ~\cite{Neuman})
en donde conocimos el término de ``polinomio dicreto''
(como fue dado en la definición \ref{def: polinomio discreto});
aunque parece una pequeñez, el considerar que
un vector de $\IR^{n}$ fue construido a partir
de la discretización de un polinomio fue de gran importancia
para la generalización de la construcción de la base de Legendre
(que se basó, principalmente,
en identificar la importancia de
los espacios $W_{n,i}$ y usarlos para definir
a los polinomios de Legendre),
pues asociada a la noción de \textbf{polinomio} está la de \textbf{grado}; 
en nuestro
contexto, a esta última noción está ligada la de \textbf{pertenencia a
algún espacio $W_{n,k}$}, y a esta \textbf{criterios sobre la forma de la
gráfica de una señal}
de dimensión $ns$, que es, en principio, lo que nos interesa estudiar.