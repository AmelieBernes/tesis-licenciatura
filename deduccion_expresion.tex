\section{Deducción de la expresión analítica de la base $\cali{L}^{n}$}

Fijada una dimensión $n$, planeamos usar 
las fórmulas dadas 
\sidenote{
Para que no exista discrepancias en la notación,
en esta discusión hicimos
los cambios de variable $m=k$, $K=m$
y $N=n-1$, siendo la variable de la izquierda la empleada
originalmente en
\cite{Neuman} y la de la derecha la que usamos nosotros.
}
en ~\cite{Neuman}
para establecer expresiones analíticas
de los vectores de nuestra base $\cali{L}^{n}$; queremos,
pues, a partir de las expresiones dadas de los
elementos de la base
\begin{equation}
\label{eq: base DLOP}
\{
y_{k}:= (P_{k}(m, n-1))_{m=0}^{m=n-1}
: \hspace{0.2cm} 0 \leq k \leq n-1
\}
\end{equation}

\noindent
obtener las de la base de Legendre discreta
\begin{equation}
\label{eq: base Legendre}
\cali{L}^{n}=
\{
\cali{L}^{n,k}: \hspace{0.2cm} 0 \leq k \leq n-1
\}.
\end{equation}

\begin{lema}
\label{prop: legendre y DLOPS son paralelos}
Sea $n \in \IN$. 
Considere a 
las bases de $\IR^{n}$
\begin{equation}
\label{eq0: base DLOP}
\{
y_{k}:= (P_{k}(m, n-1))_{m=0}^{m=n-1} \in \IR^{n}
: \hspace{0.2cm} 0 \leq k \leq n-1
\},
\end{equation}
donde las funciones $P_{k}(\cdot , n-1)= P_{k}(m , n-1)$ de variable 
$0 \leq m \leq n-1$
son las únicas que satisfacen las condiciones
\eqref{eq:dlop1} y \eqref{eq:dlop2},
y la base de Legendre discreta de dimensión $n$
\begin{equation}
\label{eq0: base de Legendre discreta}
\cali{L}^{n}=\{ \cali{L}^{n,k}= (\cali{L}_{m}^{n,k})_{m=0}^{n-1} \in \IR^{n} : 
\hspace{0.1cm} 0 \leq k \leq n-1\}.
\end{equation}
Para todo $0 \leq k \leq n-1$, los vectores 
$y_{k}$ y $\cali{L}^{n,k}$ son paralelos.
\end{lema}
\noindent
\textbf{Demostración.}
Procedemos por inducción sobre $k$.
\begin{itemize}
	\item  (Base de inducción) ys habíamos calculado que
	$\cali{L}^{n,0}=\frac{1}{\sqrt{n}}(1, \ldots , 1)$; 
	según ~\cite{Neuman}
	\TODO{[Survey], eq},
	 $P_{0}(\cdot , n-1)$ es función discreta constante uno, 
	luego, $y_{0}=(1, \ldots , 1)$. 
	Es	
	claro entonces que $\cali{L}^{n,0}$ 
	y $y_{0}$	
	son paralelos.
	
	\item (Paso inductivo) supongamos cierta la proposición para
	todo entero entre 0 y $k-1$ (inclusivo).	\\
	Según la fórmula
	~\cite{Neuman}[7], $y_{k}$ es
	un polinomio discreto con dominio 
	uniforme, donde la máxima potencia
	del polinomio discretizado para obtenerse 
	$y_{k}$ es $k$, luego,
	\[
	y_{k} \in W_{n,k}.
	\]	
	Según la condición \eqref{eq:dlop'1}, $y_{k}$ es ortogonal a
	los $k$ vectores
	\[
	y_{l}, \hspace{0.2cm} 0 \leq l \leq k-1;
	\] 
	como estos, según
	nuestra hipótesis de inducción, son paralelos (respectivamente)
	a los
	\[
	\cali{L}^{n, l}, \hspace{0.2cm} 0 \leq l \leq k-1,
	\] 
	tenemos que $y_{k}$ es ortogonal a todos estos vectores,
	vectores que conforman una base del espacio $W_{n,k-1}$
	(c.f. corolario \ref{cor: BON de legendre para espacios Wk}); así,
	\[
	y_{k} \in W_{n,k} 
	\ominus W_{n,k-1}=\text{span}(\cali{L}^{n,k});
	\]
	de esto se concluye, como queríamos, que los vectores 
	$y_{k}$ y $\cali{L}^{n,k}$
	son paralelos. \QEDB 
\end{itemize}
\vspace{0.2cm}



\noindent Del paralelismo
establecido en este último lema
\ref{prop: legendre y DLOPS son paralelos}
podemos establecer una igualdad 
entre los vectores $y_{k}$ y los $\cali{L}^{n,k}$.

\begin{prop}
\label{prop: igualdad entre los de legendre y los del survey}
Sean $n \in \IN$ y sean las bases 
\eqref{eq0: base DLOP} y 
\eqref{eq0: base de Legendre discreta} de $\IR^{n}$.
Para toda $0 \leq k \leq n-1$, 
\begin{equation}
\label{eq9: 2Dic}
\cali{L}^{n,k}= (-1)^{k} \cdot \frac{y_{k}}{||y_{k}||}.
\end{equation}
\end{prop}
\noindent
\textbf{Demostración.}
Sea $0 \leq k \leq n-1$. Del lema
\ref{prop: legendre y DLOPS son paralelos}
y el que el vector
$\cali{L}^{n,k}$ sea unitario
se deduce la existencia de un escalar $a_{k}$
tal que 
\begin{equation}
\label{eq10: 2Dic}
a_{k}\cali{L}^{n,k}= y_{k}, \hspace{0.5cm}
\text{con} \hspace{0.2cm} a_{k} \in \{ \pm || y_{k} || \}.
\end{equation}
Determinemos el signo de $a_{k}$. \\

\noindent Puesto que $\cali{L}^{n,k}$ es ortogonal a todos los polinomios
discretos de grado menor a $k$
(c.f. corolario
\ref{cor: Ln,k ortogonal a todo pol discreto de grado menor a k}), aplicando la bilinealidad
del producto punto en $\IR^{n}$ y usando la fórmula 
\TODO{[Survey(7)]},
tenemos que

\begin{align*}
a_{k}= & a_{k} \cdot 1 = 
a_{k} 
\cdot  \langle  \cali{L}^{n,k} , \cali{L}^{n,k} \rangle \\
= & \langle a_{k} \cali{L}^{n,k} , \cali{L}^{n,k} \rangle \\
= & \langle y_{k} , \cali{L}^{n,k} \rangle \\
= & \langle (P_{k}(m, n-1))_{m=0}^{n-1} 
\hspace{0.1cm}, \hspace{0.1cm} \cali{L}^{n,k} \rangle \\
= &
\left\langle
\frac{(-1)^{k}\binom{2k}{k}}{(n-1)^{(k)}} \hspace{0.1cm} \cdot
v_{k} , \cali{L}^{n,k}
\right\rangle, \\
= & \frac{(-1)^{k}\binom{2k}{k}}{(n-1)^{(k)}} \hspace{0.1cm}
\langle v_{k} , \cali{L}^{n,k} \rangle ,
\end{align*}
donde $v_{k}$ es,
como siempre, el vector 
dado por \eqref{vectores vk}.
En resumen, hemos llegado a que
\begin{equation}
\label{eq1: 9Nov}
a_{k}= (-1)^{k}\cdot 
\frac{\binom{2k}{k}}{(n-1)^{(k)}}\cdot 
\langle v_{k} , \cali{L}^{n,k}
\rangle.
\end{equation}
Como $\frac{\binom{2k}{k}}{(n-1)^{(k)}}$
es un número positivo, sólo los factores $(-1)^{k}$ y 
$\langle v_{k} , \cali{L}^{n,k} \rangle$
determinan el signo de $a_{k}$.
\noindent 
Recuerde ahora que,
según nuestra construcción inicial,
los polinomios discretos de Legendre
de dimensión $n$ (i.e.
los elementos del conjunto 
\eqref{eq0: base de Legendre discreta}) 
se obtienen 
al ortonormalizar con el proceso
de Gram-Schmidt a la base 
de $\IR^{n}$ formada por
los vectores $v_{k}$; 
podemos entonces aplicar el lema \ref{Lema1} para 
concluir que el signo del producto punto 
$\langle v_{m} , \cali{L}^{n,m} \rangle$
es positivo; así,
según \eqref{eq1: 9Nov},
el signo de $a_{k}$ 
es $(-1)^{k}$;
puesto que $a_{k}$ sólo podía ser
$||y_{k}||$ o $-||y_{k}||$ 
(c.f. \eqref{eq10: 2Dic})
y por ser la norma
de un vector es no negativa, concluimos que
$a=(-1)^{k}||y_{k}||$; sustituyendo esto en
\eqref{eq10: 2Dic},
concluimos lo deseado.
\\
\QEDB
\vspace{0.2cm}


\noindent 

Recuerde que nuestro objetivo es determinar expresiones
de los vectores $\cali{L}^{n,k}$; 
después de lo demostrado en 
la proposición
\ref{prop: igualdad entre los de legendre y los del survey}
estamos en posición de hacer esto, pues
en base a la relación de ortogonalidad 
~\cite{Neuman}, p. 746
podemos calcular el cuadrado
de la norma de cada vector $y_{k}$ (o sea,
el valor absoluto de $a_{k}$)
y  en base a 
~\cite{Neuman}[7]
podemos
obtener una expresión para $y_{k}$.

Planteamos, por fin, una expresión para
todo vector de la forma
$\cali{L}^{n,k}$ en el siguiente teorema.
Para su formulación necesitamos antes
definir el concepto de ``fading factorial''
usado en el artículo ~\cite{Neuman}.
\begin{defi}
\label{def: fading factorial}
Sean $K, m \in \overline{\IN}$. Se define
\footnote{Esta es la definición de
fading factorial dada en ~\cite{Neuman}, p.745 }
el \textbf{fading factorial} $K^{(m)}$ como sigue:
\[
K^{(m)}= K \cdot (K-1) \cdot \cdots \cdot (K-(m-1)),
\]
es decir,\begin{align*}
K^{(m)}= \begin{cases}
\frac{K!}{(K-m)!} & \hspace{0.2cm} \text{si } K \geq m, \\
0 & \hspace{0.2cm} \text{en otro caso.} 
\end{cases}
\end{align*}
\end{defi}

\begin{tcolorbox}
\begin{teo}
\label{teo: expresión analítica de BON de Legendre}
Sea $n\in \IN$. Para toda $0 \leq k \leq n-1$,
\begin{equation}
\label{eq0: 6En}
\cali{L}^{n,k}_{m}= (-1)^{k} \sqrt{\frac{(2k+1)(n-1)^{(k)}}{(n+k)^{(k+1)}}}
\suma{j=0}{k}{(-1)^{j}\binom{k}{j}\binom{k+j}{j}
\frac{m^{(j)}}{(n-1)^{(j)}}},
\end{equation}
donde $\cali{L}^{n,k}=(\cali{L}_{m}^{n,k})_{m=0}^{n-1}$
es el vector de la definición
\ref{def: base de Legendre discreta}.
\end{teo}  
\end{tcolorbox}
\noindent
\textbf{Demostración.}
Basta hacer los siguientes cambios de variable
(siendo la variable de la izquierda la que se
usa originalmente en ~\cite{Neuman} y la de la 
derecha la adoptada por nosotros);
\[
m=k, \hspace{0.2cm} K=m,
N=n-1, \hspace{0.2cm} l=m.
\]
Según la fórmula de ortogonalidad,
\begin{equation}
\label{eq12: 2Dic}
||y||_{k}^{2}= \suma{m=0}{n-1}{P_{k}(m, n-1)^{2}}
= \frac{1}{2k+1} \cdot \frac{(n+k)^{(k+1)}}{(n-1)^{(k)}}.
\end{equation}
Además, 
\begin{equation}
\label{eq13: 2Dic}
y_{k}= (P_{k}(m, n-1))_{m=0}^{m=n-1}
= \left(
\suma{j=0}{k}{(-1)^{j}\binom{k}{j}\binom{k+j}{j}
\frac{m^{(j)}}{(n-1)^{(j)}}}
\right)_{m=0}^{n-1}.
\end{equation}
Sustituyendo \eqref{eq12: 2Dic}
y \eqref{eq13: 2Dic} en la expresión
\eqref{eq10: 2Dic} obtenida
en la proposición
\ref{prop: igualdad entre los de legendre y los del survey},
concluimos que

\[
(\cali{L}_{m}^{n,k})_{m=0}^{n-1}
= \cali{L}^{n,k}
= (-1)^{k} \cdot 
\sqrt{\frac{(2k+1)(n-1)^{(k)}}{(n+k)^{(k+1)}}}
\left(
\suma{j=0}{k}{(-1)^{j}\binom{k}{j}\binom{k+j}{j}
\frac{m^{(j)}}{(n-1)^{(j)}}}
\right)_{m=0}^{n-1}.
\]


\QEDB 
\vspace{0.2cm}