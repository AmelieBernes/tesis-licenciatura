%Presentación examen profesional
%

\documentclass[]{beamer}
\usepackage{babel}
\usepackage{mathtools}
\usepackage{amsthm}
\usepackage{xcolor}
\usepackage[most]{tcolorbox} %para encerrar texto en cajas de colores

\usepackage{adjustbox} %Para poder ajustar el tamaño de las tablas

%%%% Mis macros %%%%

\let\newemptytheorem\newtheorem
\usepackage{thmbox}
			%% Secciones:
	
			\newtheorem{teo}{\bf Teorema}
			\newtheorem{lema}{\textbf{Lema}}
			\newtheorem{prop}{\bf Proposición}
			\newtheorem{obs}{\bf Observación}
			\newtheorem{dem}{\bf Demostración}
			\newtheorem{cor}{\bf Corolario}
			\newtheorem{defi}{\bf Definición}
			\newtheorem{ej}{\bf Ejemplo}
			\newtheorem{notacion}{\bf Notación}
			\newtheorem{hip}{\bf Hipótesis}

			
			\theoremstyle{definition}
			\newtheorem{nota}{\bf Nota}

			%% Símbolos matemáticos
			\newcommand*{\QEDA}{\null\nobreak\hfill\ensuremath{\blacksquare}}%
			\newcommand*{\QEDB}{\null\nobreak\hfill\ensuremath{\square}}%
			\newcommand{\TODO}[1]{\textcolor{purple}{#1}}
			\newcommand*{\final}{\null\nobreak\hfill\ensuremath{\diamond}}
			\newcommand{\IR}{\mathbb{R}}
			\newcommand{\IC}{\mathbb{C}}
			\newcommand{\IN}{\mathbb{N}}
			\newcommand{\IZ}{\mathbb{Z}}
			\newcommand{\suma}[3]{\sum\limits_{#1}^{#2}#3} %Sumas y series
			\newcommand{\union}[3]{\bigcup\limits_{#1}^{#2}{#3}} %uniones
			\newcommand{\producto}[3]{\prod_{#1}^{#2}{#3}} %productos
			\newcommand{\limite}[2]{\lim\limits_{#1}{#2}} %límites
			\newcommand{\limsu}[2]{\lim\limits_{#1 \rightarrow \infty }#2_{#1}}
			%para límites de sucesiones
			\newcommand{\Om}{\Omega}
			\newcommand{\cali}[1]{\mathcal{#1}} %Letras caligráficas
			\newcommand{\cont}[2]{$\mathcal{C} [#1, #2]$}
			\newcommand{\integ}[3]{\int_{#1}^{#2}{#3}}
			\newcommand{\ldos}{\mathit{l}^{2}}


			\DeclareMathOperator*{\ameboxplus}{{\boxplus}}





\title{La transformada de Haar-Legendre}
\author{Amélie Bernès}

%\usecolortheme{owl}
\usecolortheme[snowy]{owl}
\usetheme{Hannover}
%\useinnertheme{circles}


\title[]{Estudio y análisis espectral de los polinomios discretos de Legendre} 

\author[]{Amélie Bernès \and Moisés Soto Bajo \and Javier Herrera Vega} 

\institute[BUAP]{Benemérita Universidad Autónoma de Puebla \\ \smallskip \textit{ammel.bernes@gmail.com}}

\date[\today]{\today} % Presentation date or conference/meeting name, the optional parameter can contain a shortened version to appear on the bottom of every slide, while the required parameter value is output to the title slide

\begin{document}


% ------------------------------------------------------------ %
% ------------------------------------------------------------ %

\begin{frame}
\titlepage
\end{frame}

% ------------------------------------------------------------ %
% ------------------------------------------------------------ %


% ------------------------------------------------------------ %
% ------------------------------------------------------------ %
\begin{frame}{Outline}
    \tableofcontents
\end{frame}

% ------------------------------------------------------------ %
% ------------------------------------------------------------ %






\input{0.motivacion}
\input{1.espaciosWnk}
\input{2.PDL}
\input{3.AE}
\input{4.Referencias}
\input{Referencias.bib}


\end{document}


%\section{Motivación}
%\section{Espacios de polinomios discretos}
%	\subsection{Espacios $W_{n,k}$}
%	\subsection{Definición del grado de una señal finita}
%\section{Polinomios discretos de Legendre}
%	\subsection{Sobre los PDL en la literatura}
%    \subsection{Construcción}
%    \subsection{Simetrías en las entradas de los PDL}
%   \subsection{Cálculo de los PDL}
%    \subsection{Análisis de señales finitas en base a los PDL}
%\section{Análisis espectral}
%	\subsection{Desarrollo de metodología}
%	\subsection{Resultados del análisis numérico de algunos PDL}
%\section*{Referencias}

