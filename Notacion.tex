\chapter{Notaciones y abreviaciones}

\begin{center}
\huge{Lista de notaciones empleadas}
\end{center}

\vspace{0.5cm}


\noindent
 $x=(x_{m})_{m=0}^{n-1}$ \hbox to\linewidth{\hfil Vector de $\IR^{n}$ con sus entradas especificadas \hfil} \\
 $B_{\epsilon}(x)$ \hbox to\linewidth{\hfil Bola de radio $\epsilon$ y centro $x$. \hfil}  \\
 $:=$ \hbox to\linewidth{\hfil Igualdad cierta por definición \hfil}  \\
 $\equiv \hspace{0.2cm} (mod \hspace{0.1cm} n)$  
 \hbox to\linewidth{\hfil Congruencia módulo $n$ \hfil}  \\
 $ // $  \hbox to\linewidth{\hfil Parte entera de una división \hfil}  \\
 $\IN$  \hbox to\linewidth{\hfil Conjunto de los enteros positivos \hfil}  \\
 $\overline{\IN}$ \hbox to\linewidth{\hfil Conjunto de los enteros no negativos \hfil}  \\
 $\IZ$ \hbox to\linewidth{\hfil Conjunto de los números enteros  \hfil}  \\
 $\IR$ \hbox to\linewidth{\hfil Conjunto de los números reales \hfil}   \\
 $\IR^{+}$ \hbox to\linewidth{\hfil Conjunto de los números reales positivos \hfil}   \\
 $\IR^{+}_{0}$ \hbox to\linewidth{\hfil Conjunto de los números reales no negativos \hfil}   \\
 $span(W)$ \hbox to\linewidth{\hfil Subespacio generado por $W$ \hfil}    \\
 $\IR[t]$ \hbox to\linewidth{\hfil Anillo de polinomios con coeficientes reales \hfil}  \\
 $\IC[t]$ \hbox to\linewidth{\hfil Anillo de polinomios con coeficientes complejos \hfil}  \\
 $\partial(f)$ \hbox to\linewidth{\hfil Grado de un polinomio $f$ \hfil}   \\
 $K^{(m)}$ \hbox to\linewidth{\hfil Fading factorial, c.f. \ref{def: fading factorial} \hfil}  
 $\langle \cdot, \cdot \rangle$ \hbox to\linewidth{\hfil Producto punto \hfil}  \\
 $|| \cdot ||$ \hbox to\linewidth{\hfil Norma \hfil} \\
 $\Pi_{W}$ \hbox to\linewidth{\hfil Proyección sobre el subespacio cerrado $W$ \hfil}  \\
 $B^{\perp}$ \hbox to\linewidth{\hfil Complemento ortogonal de un subconjunto $B$ de un espacio con producto punto. \hfil}  \\
 $\leq$ \hbox to\linewidth{\hfil Relación ``ser subespacio de'' \hfil}   \\
 $\measuredangle$ \hbox to\linewidth{\hfil Ángulo \hfil}  \\
 $L^{2}[a,b]$ \hbox to\linewidth{\hfil c.f. \cite{wolfram} \hfil} \\



\vspace{0.5cm}

\begin{center}
\huge{Abreviaciones}
\end{center}

\vspace{0.5cm}

\begin{tabular}{ c c }
 G-S & Polinomio discreto de Legendre \\
 G-S & Gram-Schmidt \\
 C-S & Cauchy-Schwarz \\
 BON & Base ortonormal \\
 DLOP & Abreviación (adoptada de ~\cite{Neuman})
 de ``polinomio ortogonal discreto de Legendre'' \\
 sii & si y sólo si \\
 l.i. & linealmente independiente
\end{tabular}




\newpage