\chapter{Notaciones y abreviaciones}

\begin{center}
\huge{Notaciones}
\end{center}

\vspace{0.5cm}

\TODO{No es necesario ponerlo en forma de lista,
también puedes formar párrafos.
Poner una referencia para cuando el término aparezca
por primera vez en el texto.}


\textbf{Fading factorial:}

\begin{tabular}{ c c }
 $x=(x_{j})_{j=0}^{n-1}$ & Vector de $\IR^{n}$ con sus entradas
 especificadas. \\
 $:=$ & Igualdad cierta por definición \\
 $\equiv \hspace{0.2cm} (mod \hspace{0.1cm} n)$ & congruencia módulo $n$ \\
 $ // $ & $13//2=6$ \\
 $\IN$ & Conjunto de los enteros positivos \\
 $\overline{\IN}$ & Conjunto de los enteros no negativos \\
 $\IZ$ & Conjunto de los números enteros   \\
 $\IR$ & Conjunto de los números reales  \\
 $\IR^{+}$ & Conjunto de los números reales positivos  \\
 $\IR^{+}_{0}$ & Conjunto de los números reales no negativos  \\
 $span(W)$ & (con $W$ subconjunto de un espacio vectorial) el subespacio de $V$ que consta de todas las combinaciones lineales finitas de elementos de $W$.   \\
 $\IR[x]$ & El espacio vectorial real de los polinomios con coeficientes reales \\
 $\partial(f)$ & Grado de un polinomio $f$  \\
 $< , >$ & Usado para denotar un producto punto \\
 $|| \hspace{0.3cm} ||$ & Usado para denotar una norma \\
 $\Pi_{W}$ & Proyección sobre el subespacio cerrado $W$ (\TODO{inserta referencia}) \\
 $A^{B}$ & El conjunto de funciones de $B$ a $A$. \\
 $B^{\perp}$ & Complemento ortogonal de un subconjunto $B$ de un espacio con producto punto. \\
 $\oplus$ & Suma directa  \\
 $\leq$ & \TODO{Uso esto para subespacios?}  \\
 $\measuredangle$ & ángulos. \\
 $\boxplus $ & Suma ortogonal \\
 $\boxminus $ & Diferencia directa de dos sbuconjuntos $A$ y $B$, $A \boxminus B := A \cap B^{\perp}$ \\
\end{tabular}
\vspace{0.5cm}

\begin{center}
\huge{Abreviaciones}
\end{center}

\vspace{0.5cm}

\begin{tabular}{ c c }
 G-S & Gram-Schmidt \\
 C-S & Cauchy-Schwarz \\
 BON & Base ortonormal \\
 DLOP & Abreviación (adoptada de ~\cite{Neuman})
 de ``polinomio ortogonal discreto de Legendre'' \\
 sii & si y sólo si \\
 l.i. & linealmente independiente
\end{tabular}




\newpage