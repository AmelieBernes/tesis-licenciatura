\begin{abstract}


Motivados por la busqueda de un sistema
de representación de $\IR^{n}$ adecuado para
el estudio morfológico de señales finitas
(c.f. lista de deseos \ref{lista de objetivos}), 
para cada $n \in \IN$ mayor a uno definimos
(c.f. capítulo \ref{cap 2}) 
$n$ vectores de $\IR^{n}$, que llamamos -respetando
el nombre original de estos objetos que, como comentamos
en \ref{sec: Sobre los polinomios discretos de Legendre en la literatura},
ya habían sido estudiados y aplicados-
\textbf{polinomios discretos de Legendre} (aka PDL),
cada uno asociado a un grado entero $k$ de $0$ a $n-1$,
que juntos conforman una base ortonormal de $\IR^{n}$,
la \textbf{base de Legendre discreta $n-$dimensional}, y que, 
como mostramos en el capítulo
\ref{chap: Analisis de señales en base a coeficientes respecto a las BLDs}, 
cumple satisfactoriamente los requisitos explicados en la lista
\ref{lista de objetivos}.\\


Hacemos también un estudio de simetrías de los PDL 
(c.f. capítulo \ref{section: sobre simetrias en las entradas de los poliomios discretos de Legendre})
que, junto con las fórmulas dadas en \cite{Neuman}, 
será usado para definir algoritmos 
(c.f. capítulo \ref{Implementación computacional de las bases discretas de Legendre en Python})
para programar a los PDL de cualquier dimensión $n$. \\

Se realizó además,
con una metodología propuesta por nosotros (c.f. sección 
\ref{sec: metodologia para realizar un analisis espectral que considere frecuencias arbitrarias}),
un análisis espectral de los PDL
(c.f. capítulos \ref{chap: resultados numericos analisis espectrales})
que nos llevó
a plantear una conjetura en la que establecemos
una relación entre frecuencias y polinomios discretos de Legendre
que depende sólo de los parámetros de dimensión $n$ y 
frecuencia $k$ de este último, y que es estudiada numéricamente
para algunas dimensiones. \\


Para que los conceptos y herramientas que usamos a lo largo
del desarrollo de este trabajo queden claras, incluimos al final
un apéndice de teoría en el que
plasmamos algunos resultados o definiciones que son estrictamente
necesarios para el desarrollo de este trabajo de tesis.
Preferimos no limitarnos a citar referencias que abordaran estos temas,
pues definiciones presentes en unas son diferentes a las usadas por otras;
además, para algunos resultados clásicos (como el teorema de Gram-Schmidt)
necesitábamos dar formulaciones distintas a las canónicas pero útiles para
nosotros.\\

\begin{comment}
Mostrada con teoría y ejemplos  cómo las
bases de Legendre discretas
son un sistema de representación particularmente útil
para hacer un estudio cuantitivo de morfología
(y, potencialmente, un análisis espectral también)
de señales finitas, esbozamos cómo esto da lugar a
trabajos a futuro en esta línea.\\
\end{comment}

\mbox{}
\vfill
Este trabajo fue escrito en \LaTeX.
El lenguaje de programación de nuestra elección
para el desarrollo de la parte computacional 
fue Python. Los códigos a los que se hace referencia en este
trabajo pueden encontrarse en el repositorio
\TODO{ref}.
La mayoría de las imágenes aquí presentadas fueron,
a su vez, programadas en Python y, en algunas ocasiones,
retocadas (o dibujadas completamente) en Krita. 
El formato adoptado 
se basa en el diseño de los libros de 
Edward Tufte.
\end{abstract}