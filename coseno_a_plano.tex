\section{Caso particular en el que el subespacio en cuestión es un plano}
\label{ap: Caso particular en el que el subespacio en cuestión es un plano}

\TODO{Cambiar título e intro porque moví esto.}
Necesitaremos concentrarnos en el caso
particular en el que el subespacio cerrado $W$ es 
un plano,\sidenote{O sea, un subespacio
de dimensión $2$.} por lo que a continuación elaboramos un poco más
la teoría de la sección 
\ref{angulo entre elementos de un espacio con producto punto}
para este caso particular.

La situación es la siguiente: $V$ es un $\IR-$espacio
de Hilbert, $u$ y $v$ son elementos de $V$,
unitarios y linealmente
independientes entre sí. El espacio que ellos generan
es pues un plano, digamos,


\[
P := span \{ u, v \}.
\]

Dado $x \in V$,
el coseno del ángulo entre $x$ y $W$ es,
según la proposición
\ref{prop: algunos hechos sobre el angulo entre un vector y un subespacio},

\begin{equation}
\label{eq0: 19Marzo}
cos \left( \measuredangle (x, P) \right) = 
\frac{|| \Pi_{P}(x) ||}{||x||};
\end{equation}
para lograr expresar el lado derecho de la igualdad en términos
sólo de $u$, $v$ y $x$ (que son los elementos básicos de
nuestra discusión), conviene primero obtener, a partir 
de estos elementos, una base
ortonormal del espacio $P$.


\begin{obs}
Si $u, v \in V$ son unitarios y linealmente independientes, y $P$
es el plano que generan, entonces
$\{ u, z \}$, donde

\begin{equation}
\label{eq2: 19Marzo}
z:= \frac{v- \langle u, v \rangle u}{||v- \langle u, v \rangle u||}
\end{equation}
es una BON de $P$
\end{obs}
\noindent
\textbf{Demostración.}
Basta aplicar el teorema de Gram-Schmidt 
\ref{Teo:Gram-Schmidt}.
\QEDB
\vspace{0.2cm}

Teniendo una BON de $P$, según el 
corolario 
\ref{cor: proyeccion en terminos de BON}, se tiene la siguiente
expresión para la proyección de $x$ en $P$;

\begin{equation}
\label{eq1: 19Marzo}
\Pi_{P}(x)= \langle x, u \rangle u + \langle x, z \rangle z;
\end{equation}

\noindent
puesto que, según la definición \eqref{eq2: 19Marzo} de 
$z$ este vector es función de $u$ y $v$, fácilmente se
puede derivar, a partir de \eqref{eq1: 19Marzo},
una expresión de $\Pi_{P}(x)$ en función sólo
de $x$, $u$ y $v$. Se plasman las fórmulas 
concretas a continuación.
	\begin{prop}
	\label{prop: formulas 20Marzo}
	Sean $V$ un espacio de Hilbert, $x \in V$,
	$u,v \in V$ linealmente independientes
	y unitarios. Si $P$ es el plano
	que generan $u$ y $v$, entonces,

		\begin{equation}
		\label{eq0: 24ap}
		\Pi_{P}(x)= \frac{\langle x, u \rangle -\langle u, v \rangle \langle x, v \rangle }{1-\langle u, v \rangle^{2}} u + \frac{\langle x, v \rangle -\langle u, v \rangle \langle x, u \rangle }{1-\langle u, v \rangle^{2}} v
		\end{equation}
	y 
		\begin{equation}
		\label{eq3: 19Marzo}
		  || \Pi_{P}(x) ||^{2}=
		  \frac{\langle x, u \rangle^{2} +  \langle x, v \rangle^{2}	
	       -2  \langle x, u \rangle \langle x, v \rangle \langle u, v \rangle	}{1- \langle u, v 		\rangle^{2}}.
		\end{equation}
 
	\end{prop}

\noindent
\textbf{Demostración.}
La demostración consiste de simples manipulaciones aritméticas.
Según \eqref{eq1: 19Marzo},
\begin{align*}
\Pi_{P}(x) = & \langle x, u \rangle u + \langle x, z \rangle z \\
 = & \langle x, u \rangle u
 + \frac{\langle x, v \rangle - \langle u, v \rangle \langle x, u \rangle}{|| v -\langle u,v \rangle u ||^{2}}
(v - \langle u,v \rangle u);\\
\end{align*}

\noindent
puesto que $u$ y $v$ son unitarios, 
tenemos que
\begin{align}
\label{eq3: 23ap}
|| v -\langle u,v \rangle u ||^{2} = & 
\langle v,v \rangle^{2} -2
\langle u,v \rangle^{2} +\langle u,v \rangle^{2}\langle u,u \rangle \notag  \\
= & 1 -\langle u,v \rangle^{2}; 
\end{align}
sustituyendo \eqref{eq3: 23ap} en la última expresión para 
$\Pi_{P}(x)$ llegamos a \eqref{eq3: 19Marzo}. \\

Finalmente, 
\begin{align*}
|| \Pi_{P}(x) ||^{2} = & 
\langle x,u \rangle^{2} + \langle x,z \rangle^{2} \\
= & \langle x,u \rangle^{2} + 
\left(
\frac{\langle x,v \rangle - \langle u,v \rangle
\langle x,u \rangle}{||v -\langle u,v \rangle u ||}
\right)^{2};\\
\end{align*}

\noindent
sustituyendo \eqref{eq3: 23ap} en esta última expresión
llegamos a \eqref{eq3: 19Marzo}.

\QEDB
\vspace{0.2cm}

Usando las expresiones
\eqref{eq: coseno a subespacio}
y \eqref{eq3: 19Marzo} es fácil establecer
la siguiente proposición.

\begin{prop}
Sean $V$ un espacio de Hilbert, $x \in V$,
	$u,v \in V$ linealmente independientes
	y unitarios. Si $P$ es el plano
	que generan $u$ y $v$, entonces,
	
	
\begin{equation}
\label{eq: coseno a plano}
cos (\measuredangle (x, P)) = 
\sqrt{
\frac{\langle x, u \rangle^{2} +  \langle x, v \rangle^{2}	
	       -2  \langle x, u \rangle \langle x, v \rangle \langle u, v \rangle	}{
	       ||x||^{2} \cdot 
	       (1- \langle u, v 	\rangle^{2})  }}.
\end{equation}
\end{prop}